\section{Antragstext}

Um die finanziellen Möglichkeiten des Landesverbandes nicht ohne Not einzuschränken, sollen Firmenspenden nicht verboten werden. Spenden juristischer Personen an die den Landesverband sollen jedoch für die Öffentlichkeit transparent (d.h. sichtbar) sein und zwar mit Angabe des Spenders, der Höhe der Spende, dem Zeitpunkt der Spende, der Parteigliederung, die die Spende erhalten hat und ggf. weiteren Angaben. Bei zweckgebundenen Spenden ist der Zweck ebenfalls zu veröffentlichen. Spenden für nicht satzungsgemäße Zwecke sind von der begünstigten Parteigliederung zurückzuweisen. Die jährlichen Spenden von einer natürlichen oder juristischen Person werden weiterhin auf ein Drittel der Gesamteinnahmen (ohne Spenden) gemäß endgültigem Haushaltsplan des Vorjahres begrenzt (bei derzeit 22.900 Euro für das Jahr 2012\footnote{\url{http://vorstand.piratenpartei-mv.de/wp-content/uploads/2012/01/Haushaltsplan.pdf}} entspräche dies 7633,33 Euro für das Jahr 2013).

\section{Vergleich mit dem Status Quo}

Die Paragraphen 10 und 11 der Finanzordnung des Bundes sehen eine Veröffentlichungspflicht ab einer Summe von 10.000 Euro pro Jahr im Rechenschaftsbericht vor. Alle Einzelspenden über 1.000 Euro werden unverzüglich unter Angabe von Spendernamen, Summe und ggf. Verwendungszweck veröffentlicht.

In dieser Initiative wird \emph{zusätzlich} gefordert:

\begin{itemize}
\item
  Ausnahmslose Veröffentlichung \emph{aller} Spenden juristischer Personen.
\item
  Begrenzung der Spenden auf \emph{maximal ein Drittel der Jahreseinnamen des Vorjahres} und von natürlicher oder juristischer Person.
\end{itemize}
\section{Begründung}

Die Initiative ist weitgehend aus den im Bund abgestimmten Initiativen 1623\footnote{\url{https://lqfb.piratenpartei.de/lf/initiative/show/1623.html}} (September 2011, Zustimmung 91\%) und 826\footnote{\url{https://lqfb.piratenpartei.de/lf/initiative/show/826.html}} (August 2010, Zustimmung 88\%) übernommen. Zur ersten Initiative gab es die Anregung\footnote{\url{https://lqfb.piratenpartei.de/lf/suggestion/show/2064.html}}, sich an den Vorgaben von Transparency International\footnote{\url{http://www.transparency.de/Transparency-International-Deu.1007.0.html}} zu orientieren. Diese legen unter anderem eine Höchstgrenze der jährlichen Spende von einer natürlichen oder juristischen Person auf maximal 50.000 Euro pro Jahr fest. Diese Gesamtsumme wird in dieser Initiative auf einen relativen Anteil des Budgets des Vorjahres angepasst. Hintergrund dabei ist, dass Spenden nur in einem angemessenen Rahmen angenommen werden sollten. Diese Initiative schlägt vor, dass ein Drittel der Einnahmen (ohne Spenden) des Vorjahres als angemessene Obergrenze für Spenden anzusehen.
