\section{Antrag}

Die Piratenpartei Mecklenburg-Vorpommern spricht sich für die Etablierung von Anlaufstellen für Opfer sexueller Gewalt aus, die neben sozialer, medizinischer und psychologischer Beratung die gerichtsfeste Sicherung von Befunden (zum Beispiel Verletzungen und DNA-Spuren) anbieten, die in einem späteren Ermittlungs- und Strafverfahren als Beweis dienen können.

Die zu etablierende Einrichtungen sollen anonym und kostenfrei nutzbar sein und eng mit schon bestehenden Beratungsstellen kooperieren. Auch sollen die Anlaufstellen über qualifiziertes Personal verfügen, um den Opfern sexueller Gewalt die Möglichkeit zu geben, nicht in einer stark belastenden Ausnahmesituation entscheiden zu müssen, ob sie Strafanzeige erstatten wollen, ohne dass durch Überlegenszeit Befunde verlorengehen können.

In Betracht als Anlaufstelle kommen zum Beispiel Schutzambulanzen und gynäkologische Abteilungen von Krankenhäusern. Wir wollen für die konkrete Organisation der Einrichtungen die Erfahrungen aus entsprechenden Modellprojekten (gegenwärtig in Frankfurt/Main\footnote{\url{http://www.frauennotruf-frankfurt.de/fileadmin/redaktion/pdf/FNR-PM-Beweissicherung-nach-Vergewaltigung.pdf}}, im Landkreis Fulda\footnote{\url{http://schutzambulanz-fulda.de/wp-content/uploads/2012/06/120620\_Handout\_allgemein\_\%20und\%20Polizei\_SF\%2002-2012\%20zum\%20Versenden\%20per\%20Mail.pdf}} und in München\footnote{\url{http://www.rechtsmedizin.med.uni-muenchen.de/wissenschaft/klinische\_rechtsmed/ambul\_gewaltopfer/index.html}}) auswerten. Der Opferschutz soll aus Landesmitteln finanziert werden.

\section{Begründung}

tl;dr: Opfer kann ohne Polizei Spuren rechts sicher sicherstellen lassen und später auf diese, sollte es zu einer Anzeige kommen, zurückgreifen, ohne sofort eine Anzeige bei der Polizei aufgeben zu müssen, was dem Abbau einer Barriere gleichkommt. (siehe Fußnote 1, 2)

Viele Opfer sexueller Gewalt fühlen sich in den ersten Stunden oder Tagen nach der Tat nicht in der Lage, Anzeige zu erstatten. Dies kann mannigfaltige Gründen haben: Angst oder falscher Solidarität mit dem Täter oder Furcht davor, der Vernehmung durch die Polizei nicht gewachsen zu sein, um nur einzelne zu nennen. Entscheiden sich die Opfer später doch noch für eine Anzeige, vielleicht bestärkt durch Vertrauenspersonen aus dem persönlichen Umfeld oder nach Konsultation einer Beratungsstelle, sind die physischen Befunde oft bereits unbrauchbar geworden.

Bis jetzt kann nur die Polizei Befunden in den ersten Stunden nach einem Vorfall sicherstellen lassen, wozu eine Anzeige Voraussetzung ist. Bei der opferbeauftragten Untersuchung kann sich das Opfer nach einer Gewaltet untersuchen lassen und zu einem späteren Zeitpunkt Anzeige erstatten. Die entstehenden Kosten werden vom Träger der Einrichtungen, also dem Staat oder Trägervereinen übernommenen. Sobald es zur Anzeige kommt, werden die Kosten durch den Staat übernommen, da die Befunde zu Beweisen werden.

Ein weiterer Vorteil ist die frühere kompetente medizinische Beratung. So können kurz- und langfristige gesundheitliche Folgen einer Gewalttat durch z.B. die Prävention von sexuell übertragbaren Krankheiten wie HIV und Hepatitis begrenzt werden. Die oft lange anhaltenden psychischen Belastungen können durch eine früh einsetzende Intervention abgemildert werden.

Die Ausführung in Mecklenburg-Vorpommern kann sich an die Modellprojekte der andere Bundesländer (siehe Fußnote 1, 2, 3) anlehnen. In den Städten mit Gynäkologischen Kliniken, können diese die Befundsicherung übernehmen. Sollten keine Kliniken vorhanden sein, könnten Praxen oder Schutzambulanzen dies in unserem weiten Bundesland übernehmen.

Die Piratenpartei Mecklenburg-Vorpommern erhofft sich von dieser Maßnahme eine verbesserte Aufklärungsrate auf dem Feld der Delikte gegen die sexuelle Selbstbestimmung sowie eine Verringerung der psychischen Belastung von Betroffenen unmittelbar nach der Tat.

\subsection{Hinweis}

Die Grundidee kommt von Burkhard Masseida aus Hamburg und wurde dort im LQFB und auf dem LPTHH angenommen.

\subsection{Fragen}

\begin{itemize}
\item
  F: Mich interessiert zu diesem Antrag, ob es ähnliche Einrichtungen auch in kleineren Städten gibt als in Frankfurt oder Hamburg.
\item
  A: Ja, zB die Schutzambulanz Fulda, ein im öffentlichen Gesundheitsdienst angesiedeltes Modellprojekt zur Verbesserung der Versorgung von Gewaltopfern in den Landkreisen Fulda, Bad Hersfeld-Rotenburg und dem Vogelsbergkreis.
\end{itemize}
\begin{itemize}
\item
  F: Damit verbunden ist die Frage, was sich unter der Etablierung von Beweissicherungseinrichtungen verstehen lässt. Der Aufbau einer Ambulanz an einem bestimmten Ort oder die Integration beispielsweise in einem Klinikum.
\item
  A: Hier gibt es verschiedene Ansätze. Die Schutzambulanz Fulda ist eine eigene Ambulanz/Praxis. In Frankfurt wird es durch die Klinik für Gynäkologie und Geburtshilfe und dem Institut für Rechtsmedizin, beide am Klinikum der Johann Wolfgang Goethe-Universität übernommen. Wie gesagt, das ganze sind leider auch heute noch Modelle und Versuche und es gibt nicht das eine Konzept. An sich soll die Möglichkeit geschaffen werden.
\end{itemize}
\begin{itemize}
\item
  F: Ich fände es gut, wenn im Antragstext von den guten Vorbildern abgesehen auch erläutert wird, wie so ein Konzept für unser doch ganz anders strukturiertes Bundesland aussehen könnte.
\item
  A: Ist eingebaut.
\end{itemize}
\begin{itemize}
\item
  F: Ich lese aus den Antworten jetzt heraus, dass es nicht darum geht eine bestimmte Institution an einem bestimmten Ort aufzubauen, wo dann jemand sitzt und darauf wartet, dass sich ein Opfer meldet.
\item
  A: Sowohl als auch. Es gibt wie gesagt verschiedene Ansätze für solche Einrichtungen.
\end{itemize}
