\section{Antrag}

Die Piratenpartei Mecklenburg-Vorpommern spricht sich für die zeitnahe, flächendeckende und barrierefreie Einführung eines Notruf- und Informationssystem per Mobilfunk, Internet und Fax aus.

\section{Begründung}

tl;dr: Ältere und benachteiligte Personen haben in Mecklenburg-Vorpommern in einer Notfallsituation, im Vergleich zu einer ähnlichen Situation in Sachsen, Brandenburg und Berlin bei weitem schlechtere Karten. Dies kann durch eine Änderung im Notrufsystem des Landes leicht geändert werden.

Bis heute gibt es in Mecklenburg-Vorpommern gar keine Möglichkeit, im Gegensatz zu Sachsen, Brandenburg und Berlin mit der Möglichkeit einer Notruf-SMS, in einer Notfallsituation barrierefrei einen Notruf abzusenden. Dies ist besonders wichtig für Menschen mit Behinderung und Personen höheren Alters, aber auch wenn ein Handyakku nicht mehr für einen Notruf per Sprache ausreicht. In solchen Situationen ist ein non-verbaler Notruf notwendig. Ein bundesweiter barrierefreier non-verbaler Notruf für Polizei, Feuerwehr und Krankenwagen existiert derzeit nicht, obwohl sogar seit 1991 die europaweit einheitliche Notrufnummer 112 existiert. (1)

Bisher gibt es lediglich in Sachsen, Brandenburg und Berlin spezielle SMS Notrufnummern, die technisch gesehen relativ problemlos auf ganz Deutschland ausgeweitet werden können. Der Mobilfunkanbieter wandelt die SMS in ein Fax um, das dann an die zuständige Polizeidienststelle weitergeleitet wird. Außerdem besteht die Möglichkeit, direkt ein Notruf-Fax zu versenden. Der Service ist aber ausschließlich für Menschen mit Hörbehinderungen gedacht.\footnote{\url{http://www.rettungsdienst.de/nachrichten/barrierefreie-notrufe-schaffen-28129}}

Österreich hat bereits sehr gute Erfahrungen mit einer SMS Notrufnummer für Gehörlose gemacht, die von allen großen Netzbetreibern unterstützt wird. Es gibt derzeit zwar eine Notfall-Fax-Einrichtung (z.B. in Münster) für Gehörlose und Schwerhörige Menschen, aber dieser ist umständlich wird nicht immer so ernst genommen wie es vonnöten wäre, so dass Hilfe teils zu spät eintraf. Länder in denen schon ein Notruf per SMS möglich ist sind: USA, England, Australien, Irland, Singapur, Portugal.

Ein Missbrauch dieser neuen Systeme ist natürlich genauso, wie bei der heutigen verbalen Methode möglich. Dies wäre aber auch im non-verbalen System ein Missbrauch des Notrufs, der verfolgt und geahndet werden würde.

Im Falle einer Katastrophe in Deutschland wird die Bevölkerung per Sirene und Lautsprecherdurchsagen informiert und angewiesen, das Radio und/oder TV einzuschalten um weitere Informationen zu erhalten. Die Bevölkerung wird dazu angehalten ihre Nachbarn und vor allem Hilfsbedürftige zu informieren. Hörbehinderte Menschen bekommen davon jedoch nicht sofort etwas mit und sind somit auf die Aufmerksamkeit ihrer Mitmenschen angewiesen.

Abschließend sollte man bedenken, dass am 15. März 2012 alle demokratischen Abgeordneten in Schwerin für einen Koalitionsantrag, der die Stärkung des barrierefreien Tourismus\footnote{\url{http://www.barrierefreiheit.de/news-details/items/einstimmiger\_beschluss\_im\_landtag\_mv\_zur\_staerkung\_des\_barrierefreien\_tourismus.html}} zum Ziel hatte, stimmten. Hiermit soll der Tourismus gestärkt und die Teilnahme für alle ermöglicht werden. Eine Verbesserung des Notrufsystems wäre somit nicht nur für die Tourismus ein zusätzliches Plus, sondern vor allem für die älteren und/oder eingeschränkten Bewohner des Flächenlandes Mecklenburg-Vorpommerns.

\subsection{Hinweis}

Die Grundidee kommt von Tbrass aus Schleswig-Holstein und wurde von mir übernommen und umgeschrieben.

\subsection{Fragen}

\begin{itemize}
\item
  F: Welches Gesamtkonzept verfolgst du mit deinen Anträgen?
\item
  A: Probleme erkennen, Lösungen finden und Angebote machen.
\end{itemize}
\begin{itemize}
\item
  F: Hast du Gedanken für gesundheitspolitische Grundpositionen der Piraten in MV
\item
  A: Ich würde erst einmal Bochum abwarten. Sollten wir, wie ich hoffe, danach ein Grundsatzprogramm Gesundheit haben, kann man daran etwas für M-V weiter stricken. Jetzt eigene Grundlagen zu bauen, ohne die Linie der Partei zu haben, halte ich für ungünstig. Lass uns nächstes Jahr etwas für M-V bauen.
\end{itemize}
