\section{Antrag}

Die Landesmitgliederversammlung der Piratenpartei Mecklenburg- Vorpommern beschließt, die Beauftragung des Pressesprechers des Landesverbandes und des Koordinators für die Öffentlichkeitsarbeit des Landesverbandes erfolgt jeweils jährlich durch die Landesmitgliederversammlung. Muss ein außerordentlicher Wechsel eines Beauftragten vor der regulären Landesmitgliederversammlung stattfinden, übernimmt der Landesvorstand in Vertretung die Beauftragung bis zur nächsten Landesmitgliederversammlung.

\section{Begründung}

Der Inhalt des Antrags wurde von den Teilnehmern der Klausurtagung der AG Öffentlichkeitsarbeit am 16.09.2012 in Stralsund\footnote{\url{https://ag-oe.piratenpad.de/77}} besprochen und mehrheitlich beschlossen (12 Ja-Stimmen, 2 Enthaltungen). Er soll den Beauftragten die größtmögliche Legitimation für ihre Arbeit geben und sie mit den nötigen Vollmachten ausstatten, um den gesamten Landesverband im Rahmen ihrer Aufgaben vertreten zu können. Er folgt der Praxis zur Wahl von Vertretern für den Finanzrat. Der Antrag vereint dabei die Intentionen der beiden Liquid Feedback- Initiativen i147\footnote{\url{https://lqpp.de/mv/initiative/show/147.html}} und i148\footnote{\url{https://lqpp.de/mv/initiative/show/148.html}} zur Einführung von Beauftragungen als geeignetes Instrument, um Vollmachten zu vergeben und Arbeitsaufgaben verbindlich festzulegen. Die Initiative i147, welche die Beauftragung von langfristigen und landesweit bedeutsamen Aufgaben durch die Landesmitgliederversammlung fordert, hat bei der Abstimmung im Liquid Feedback mit 13 Ja-Stimmen bei 3 Enthaltungen und 8 Nein- Stimmen knapp gewonnen. Die Initiative i148 (12 Ja, 2 Enthaltungen, 10 Nein), welche Beauftragungen grundsätzlich als Aufgabe des Landesvorstandes vorsieht, wird durch die Vertretungsbefugnis des Landesvorstandes teilweise mit in den Antrag aufgenommen. Kurzfristige Beauftragungen, wie Teilnahmen an Veranstaltungen, sind von dem Antrag nicht betroffen. Grundsätzlich sollen Beauftragungen als Mittel nur bei notwendigen Vertretungsvollmachten eingesetzt werden.
