\section{Antrag}

Die Gemeinden in Mecklenburg-Vorpommern haben sich vielfach zu Zweckverbänden zusammengeschlossen, um ihre Aufgaben zu erledigen. Vor allem die Trinkwasserversorgung und die Abwasserentsorgung wird häufig von Zweckverbänden durchgeführt. Diese sollen keine Zwangsverbände sein. Die Gemeinden sollen selbst darüber entscheiden können, ob sie den Zweckverband wieder verlassen wollen. Eine Zustimmung des Zweckverbands oder der Aufsichtsbehörde darf dafür nicht erforderlich sein. Wir wollen das in der Kommunalverfassung klarstellen. Die Folgen des Austritts, vor allem die Auseinandersetzung des Verbandsvermögens, sollen zwischen Zweckverband und Gemeinde vertraglich geregelt werden.

\section{Begründung}

In der Praxis tauchen immer wieder Probleme auf, wenn sich Gemeinden entschließen, aus dem Wasserzweckverband auszutreten und die Aufgaben Trinkwasserversorgung und Abwasserentsorgung in eigener Regie zu erledigen. Nach unserem Verständnis von kommunaler Selbstverwaltung dürfen Gemeinden in ihrer Freiheit, einem Zweckverband beizutreten, nicht beschränkt werden. Gleiches muss für den Austritt gelten. Dass der Austritt zu praktischen Problemen führt (Auseinandersetzung über das Verbandsvermögen, Übergang der Wasseranlagen auf dem Gemeindegebiet in deren Eigentum), darf kein Grund dafür sein, den Austritt nicht zuzulassen. Diese Fragen sind vertraglich zu klären.
