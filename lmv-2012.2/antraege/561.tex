\section{Antrag}

§ 9b Absatz 10 der Satzung des Landesverbandes wird ersatzlos gestrichen:

\begin{quote}
»Die Landesmitgliederversammlung beschließt die Geschäftsordnung der Ständigen Mitgliederversammlung, in der auch die Konstituierung der Ständigen Mitgliederversammlung geregelt ist.«

\end{quote}
\section{Begründung}

Die Regelung wurde erst bei der letzten LMV in der Satzung eingefügt, zusammen mit allen übrigen Regelungen zur SMV. Die sich aus dieser Bestimmung ergebende Abhängigkeit der SMV von der jeweils letzten LMV ist allerdings nicht mit dem gewollten Status der SMV zu vereinbaren.

Mit Schaffung der Grundlagen für die SMV sollte kein neues Organ schaffen, das etwa ``unter'' der LMV steht. Vielmehr sind die übrigen Regelungen zur SMV so angelegt, dass die SMV nur eine besondere Form der LMV ist und ihr im Übrigen, zumindest verfahrensmäßig, gleichsteht. Wenn die SMV der LMV das Verfahren betreffend gleichsteht, muss sie sich ebenso wie diese eine Geschäftsordnung geben und damit konstituieren können. Auch muss es ebenso wie auf der LMV jederzeit möglich sein, einen GO-Änderungsantrag zu stellen und die GO der SMV zu ändern.

\section{Folgewirkung für die LMV 2012.2}

Falls der Antrag die für einen Satzungsänderungsantrag notwendige Zweidrittelmehrheit findet, ist eine Diskussion und Beschlussfassung über eine GO für die SMV auf der LMV 2012.2 nicht mehr notwendig. Die SMV könnte ihre GO selbst diskutieren und beschließen.

Der Antrag wurde als LQFB-Initiative angenommen; Ja: 21 (72\%), Enthaltung: 1; Nein: 8 (28\%).

Ergänzung zu meiner LQFB-Initiative: Eine Mitgliederversammlung muss keine Versammlungsämter (z. B. Versammlungsleiter) bestimmen, um beschlussfähig zu sein. Und selbst wenn solche bestimmt werden, muss dies nicht in geheimer Wahl, also auf einer realen LMV, geschehen.
