\section{Antrag}

Der Vorstand erfüllt seine Aufgaben transparent. In die Geschäftsordnung des Vorstandes sind folgende Regelungen aufzunehmen:

\begin{enumerate}
\item
  Sitzungen des Vorstandes sind mit Angaben zu Ort, Datum, Uhrzeit sowie Tagesordnung möglichst mit einer Frist von einer Woche, spätestens jedoch mit ihrem Beginn, öffentlich bekannt zu machen.
\item
  Der öffentliche Teil von Sitzungen des Vorstandes ist in einem öffentlich zugänglichem Raum (real oder digital) durchzuführen, sodass Gäste teilnehmen können.
\item
  Der wesentliche Inhalt sowohl aller Sitzungen des Vorstandes, auch außerplanmäßiger, ist zeitnah zu dokumentieren und zu veröffentlichen. Themen, die personenbezogene Daten betreffen, sind ausreichend zu anonymisieren, ggf. ist nur das Thema in allgemeiner Form festzuhalten (z. B.: ``Entscheidung über zwei Anträge auf Beitragsminderung'').
\end{enumerate}
\section{Begründung}

Mit dem Parteiprogramm fordert die Piratenpartei von Inhabern öffentlicher Ämter des Staates Transparenzund Nachvollziehbarkeitihres Handelns. Schon um diese Forderung glaubwürdig vertreten zu können, sollten die Inhaber öffentlicher Ämter in der Piratenpartei unsere Transparenzforderung vorleben.
