\section{Antragstext}

Der § 9a Absatz 1 der Satzung soll wie folgt gefasst werden:

\begin{quote}
Der Vorstand besteht aus

a) dem oder der Vorsitzenden,
b) dem oder der stellvertretenden Vorsitzenden,
c) dem Schatzmeister oder der Schatzmeisterin,
d) dem Vorstand oder der Vorständin für die Mitgliederverwaltung und
e) dem Vorstand oder der Vorständin für die Programmkoordination.

\end{quote}
\section{Begründung}

Unsere aktuelle Satzung bedient sich einer rein maskulinen Darstellung des Vorstandes. Ich möchte stattdessen, dass Frauen sich explizit angesprochen und erwähnt fühlen, wenn sie unsere Satzung lesen.

Die hier vorgeschlagene Fassung entspricht dem Parteiengesetz und verzichtet auf Bezeichnungen, die das Amt künstlich erhöhen. Stattdessen wird die konkrete Funktion benannt.

\section{Q\&A}

Warum ist es wichtig, dass die beiden großen Geschlechtergruppen explizit benannt werden?

Sprache und Wirklichkeit interagieren miteinander. So wie die Wirklichkeit unseren Sprachschatz beeinflusst, wirkt auch die Sprache auf unser Verhalten und wie wir die Realität auffassen. Eine rein maskuline Beschreibung der Ämter muss darum zwar nicht zwingend dazu führen, dass nur Männer kandidieren oder gewählt werden, es manifestiert aber durchaus den Zustand, in dem wir uns befinden: Wir sind eine Partei in der Frauen deulich unterrepräsentiert sind. Wir können in unserer Satzung klarstellen, dass dieser aktuelle Zustand weder gewünscht noch selbstverständlich ist.

Gibt es ``die Vorständin''?

Ja: http://www.duden.de/rechtschreibung/Vorstaendin\footnote{\url{http://www.duden.de/rechtschreibung/Vorstaendin}} Die Form wird selten gebraucht, auch, weil vornehmlich Männer Vorstandsposten besetzen. Sie ist aber völlig in Ordnung und klingt nur wegen der Seltenheit ungewohnt.

Warum kein Binnen-I oder Schrägstriche?

Die gewählte Form ist sicher umständlich, aber in jedem Fall völlig korrekt. In einem Fließtext würde ich über das Gendern von derartigen Bezeichnungen auch anders denken, aber wir haben hier eben eine Übersicht und diese wird ohnehin Punkt für Punkt gelesen.
