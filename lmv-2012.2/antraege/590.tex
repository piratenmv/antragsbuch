\section{Antrag}

Die Piratenpartei Mecklenburg-Vorpommern spricht sich für ein verpflichtendes handwerkliches Fortbildungskonzept\footnote{\url{https://www.thieme-connect.com/ejournals/abstract/10.1055/s-0029-1240646}} für Ärzte und medizinisches Personal in der Geburtshilfe und der angegliederten Fachrichtungen aus, um dauerhaft eine vergleichbare Erfahrung wie bei mehr als 700 Geburten pro Jahr vorzuweisen.

Des Weiteren sollen landesweite, Klinik übergreifende Übungszentren für Aus-, Weiter- und Fortbildung eingerichtet werden, in denen Übungsgeräte, Parcours und Lehrmaterial bereitgestellt werden. Dies könnte auch in Kooperation mit anderen Bundesländern geschehen.

Auch sollte die Weiter- und Fortbildungsordnung angepasst werden, so dass diese Übungszentren verpflichtend, in von Experten zu definierenden Abständen, von in der Geburtshilfe tätigen Ärzten zum Zwecke der Weiter- und Fortbildung besucht werden müssen.

Die Finanzierung kann über europäische und länderübergreifende Strukturfonds geschehen, da nur durch Zusammenschlüsse einzelner Kliniken, auch länderübergreifend, die Auslastung der Trainingsgeräte gewährleistet werden kann.

\section{Begründung}

Die Schwangerschaft und Geburt ist auch heute immer noch eine gefährliche Situation für Mutter und Kind. Darüber kann auch die Tendenz in Richtung einer natürlichen Geburt in Geburtshäusern nicht hinwegtäuschen. Screening und Risikoabschätzungen können einen hohen Anteil von Problemen und Gefahren während der Geburt verhindern. Hochschwangere Frauen werden so frühzeitig in gut ausgestattete und mit Geburten erfahrene Kliniken überführt.

Geburtsklinik und die angestellten Ärzte können aber einen ausreichenden Erfahrungslevel aber nur gewährleisten, wenn eine ausreichende Anzahl von Geburten pro Jahr stattfinden. Für eine ausreichende Erfahrung wurde in der Uniklinik Rostock während der ehemaligen DDR mit mindestens 800 Geburten pro Jahr gerechnet. Heute mit modernster Technik sollten es aber immer noch mindestens 700 Geburten im Jahr für eine Geburtsklinik sein. Dies würde aber zurzeit in Mecklenburg Vorpommern nur für knapp 7--8 Kliniken zutreffen. Sobald eine Anzahl von 500 Geburten pro Jahr unterschritten wurde, kann von keinem ausreichenden Training für die Fortbildung mehr ausgegangen werden.

Notfälle, wie das durchführen einer Zangengeburt, die bei ungefähr 5\% der Geburten auftreten, müssen geübt werden und das Team muss eingespielt sein. Dies gilt nicht nur für die Gynäkologie, sondern auch für die Anästhesie, die Kinderärzte, die Hebammen und die jeweilige Pflege. Das Problem hier ist nun, das die 500 Geburten gerade noch für die Wahrung der fachlichen und vor allem handwerklichen Fähigkeiten, also der Fortbildung der erfahrenen Fach, Ober- und Chefärzte ausreicht. Die Weiterbildung des Ärztlichen Nachwuchses bleibt aber so auf der Strecke. Im Umkehrschluss würde eine geringe Anzahl von Geburten und eine gute Weiterbildung bedeuten, dass die Fortbildung auf der Strecke bleibt und im Notfall die Fachkompetenz fehlt.

Die Weiter- und Fortbildung\footnote{\url{http://www.springerlink.com/index/p561230p5163ux2x.pdf}} kann also nur ab einer gewissen Anzahl an Geburten pro Jahr als ausreichend und sinnvoll bezeichnet werden. Aber dank der heutigen technischen Möglichkeiten an Phantomen und Modellen, kann vieles in Trainingsanlagen simuliert und trainiert werden. Diese Modelle sind aber nicht unbedingt billig. Aus diesem Grund sollen auch Landesweite Trainingszentren etabliert werden, so dass verschiedene Kliniken Zugang haben und die Modelle eine hohe Auslastung erreichen.

\subsection{Hinweis}

Die Grundidee kommt von Professor Koepcke während eines Gesprächs. Der Antrag liegt mir am Herzen, da so eine hohe Erfahrung, Fort- und Weiterbildung in der Geburtshilfe in einem Flächenland wie Mecklenburg-Vorpommern sichergestellt und so Mecklenburg-Vorpommern für junge Eltern mit Kinderwunsch weiterhin attraktiv gemacht werden kann.

\subsection{Fragen}

\begin{itemize}
\item
  F: Wie wird bisher die Weiter- und Fortbildung in diesem Bereich geregelt?
\item
  A: Die Weiterbildung ist zur Zeit in der Weiterbildungsordnung der Ärztekammern, bei uns die von Mecklenburg-Vorpommern\footnote{\url{http://www.aek-mv.de/default.aspx?pid=20090604093219625}}, geregelt.
\end{itemize}
\begin{itemize}
\item
  F: Was passiert wenn es diese Pflicht nicht gibt? Die von dir beschriebene mangelnde Erfahrung durch nicht genug Training (500 Geburten/Jahr dürften ca 1 Geburt pro Schicht eines Arztes sein) ist naturgemäß schlecht quatifizirbar. Wie groß ist der Handlungsdruck und wie viel kann da eine Pflicht kompensieren?
\item
  A: Nun ja, wir haben beim letzten LPT gesagt, wir stehen zu kleinen Kliniken, wir wollen die flächendeckende Versorgung um auch das Land am Leben zu erhalten. Wenn wir nur die Klinken mit \textgreater{}700 Geburten pro Jahr haben wollen, würden 13 Geburtskliniken wegfallen. Wenn wir nur die \textless{} 500 schließen würden, wären das immer noch 7 Stück. Dh, wir haben uns zur Sicherung der Kliniken ausgesprochen, nun müssen wir schauen das auch die Qualität da bleibt.
\item
  A: Wie groß der Handlungsdruck ist, lässt sich nicht in Zahlen gießen. Noch haben wir erfahrene Ärzte. Aber diese Erfahrung muss beibehalten werden und wenn man in einer Klinik mit wenig Geburten arbeitet, muss man eben anderweitig in Übung bleiben. Das bedeutet, wir würden hier nichts Abstimmen was heute direkt spürbare Änderungen vollbringt, aber es ist zukünftig wichtig.
\end{itemize}
\begin{itemize}
\item
  F: Wenn das in den Bereich ärztlicher Selbstverwaltung geht, möchte ich da aber nicht politisch reinfunken.
\item
  A: Schon heute wird diese ärztliche Selbstverwaltung teilweise oder ganz in Frage gestellt und von Gesetzen beschnitten. Hier wäre es nur eine logische Weiterschreibung des Programmpunktes des letzten LPTs.
\end{itemize}
