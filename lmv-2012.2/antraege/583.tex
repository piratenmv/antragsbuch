\section{Antrag}

Die Finanzordnung des Landesverbandes Mecklenburg-Vorpommern wird um folgenden Passus ergänzt:

\begin{quote}
Der Landesverband Mecklenburg-Vorpommern nimmt keine Spenden juristischer Personen an.

\end{quote}
\section{Begründung}

Ein Verbot von Firmenspenden an Parteien wäre generell zu begrüßen. Das Parteienfinanzierungssystem gehört insgesamt auf den Prüfstand. So sollte z.B. die Parteienfinanzierung nicht an den Umfang von eingehenden Spenden gekoppelt sein. Eine Wahlkampfkostenerstattung sollte mehr am Gedanken der Chancengleichheit ausgerichtet werden.

Hier bleibt uns zunächst nur die Möglichkeit, die gesellschaftliche Debatte zu diesen Wechselwirkungen zu entfachen. bei uns selbst können wir jedoch beginnen.

Firmen versuchen über Geldflüsse Einfluss auf politische Parteien zu nehmen. Es gibt keinen anderen Grund für die sogenannte ``politische Landschaftspflege'' von Unternehmen. Steuerliche Gründe können es jedenfalls nicht sein - juristische Personen können Parteispenden nicht steuerlich absetzen. Diese Möglichkeit steht nur natürlichen Personen offen (§ 34g Einkommensteuergesetz\footnote{\url{http://www.gesetze-im-internet.de/estg/\_\_34g.html}})

Nützlichkeitserwägungen sollten nicht vor Grundsätzen wie Unabhängigkeit rangieren.
