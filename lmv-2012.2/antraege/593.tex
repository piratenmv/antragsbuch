\section{Antrag}

Die Piratenpartei Mecklenburg-Vorpommern spricht sich für eine landesweite Einführung von „automatisierten externen Defibrillatoren`` (AED)\footnote{\url{http://de.wikipedia.org/wiki/Automatisierter\_Externer\_Defibrillator}} aus. Diese sollten den Forderung der „American Heart Association`` und des „Swiss Resuscitation Council``\footnote{\url{https://lqpp.de/mv/initiative/show/www.todi.ch/AED-Richtlinien.pdf}} folgen.

\begin{itemize}
\item
  Jeder Ort, an dem innerhalb von 5 Jahren wahrscheinlich ein plötzlicher Herztod erfolgen wird
\item
  Öffentliche Plätze mit großen Menschenansammlungen und einer Wahrscheinlichkeit eines Kreislaufstillstandes pro 1'000 Personen-Jahre
\item
  Bahnhöfe, Flughäfen, Konzertsäle, Theater, Industriegebiete (Fabriken), Bürokomplexe, Einkaufszentren, Sportstadien, Grosse Veranstaltungen wie Open Air's, Stadtfeste, etc.
\item
  Risikosituationen, wie zB: Arztpraxen, Spitäler (z.b. Bettenstationen), Rettungswagen, Einsatzambulanzen, Feuerwehr-/Polizeifahrzeuge, Wohnungen von Risikopatienten, Sportplätze, Fitness-Center, Schwimmbäder, Freibäder, Bestimmte Industriezweige, z.b. Elektrizitätswerke
\item
  Wenn das Zeitintervall zwischen Notruf und Verfügbarkeit des manuellen (konventionellen) Defibrillators über 5 Minuten ist, innerhalb eines konventionellen Alarmsystems, z.b. Transportmittel, (Flugzeuge, Schienenverkehrsmittel, Schiffe)
\end{itemize}
Diese AEDs sollten für jeden Bürger sichtbar und zugänglich aufgestellt werde.

Auch setzt sich die Piratenpartei Mecklenburg-Vorpommern für eine Werbe-, Informations- und Ausbildungsinitiative, in der die Angst des Bürgers vor Erster-Hilfe im Allgemeinen und der Nutzung des AEDs im speziellen genommen wird, angelehnt an die Werbekampagne zur AED-Studie in Chicago\footnote{\url{http://www.nejm.org/doi/full/10.1056/NEJMoa020932}}, ein.

\section{Begründung}

tl;dr: Ein Todesfall alle 5min in Deutschland durch den plötzlichem Herztod:. „Mit den heutigen technischen Möglichkeiten der Automatischen Externen Defibrillatoren sollte jeder Laie defibrillieren können\ldots{} Wer dies nicht erkennen oder verstehen will, der versteht nicht wie die Menschen sterben.`` (1986, Peter Safar) Wenn man eine nicht ansprechbare, erwachsene Person auffindet, ist es nicht selten ein plötzlicher Kreislaufstillstand. In dieser Situation ist der Beginn einer sofortigen Reanimation äußerst wichtig. Diese Bemühung besteht nicht nur aus der Herzdruck-Massage und der „Mund-zu-Mund-Beatmung``. Ein möglichst früher Beginn der Defibrillation spielt eine entscheidende Rolle, denn Ursache für den plötzlichen Kreislaufstillstand können Kammerflimmern, pulslose ventrikuläre Tachkardie und Asystolie sein. Während bei Asystolie die Therapieoptionen und Prognose stark eingeschränkt sind, kann bei Kammerflimmern oder pulslose ventrikuläre Tachkardie das Defibrillieren Leben retten.

Die Überlebenschance für einen Patienten liegt in der ersten Minute nach Herzkreislaufstillstand bei 90\%, aber mit jeder Minute um 7--10\% geringer. Nach 10 Minuten besteht nur noch der Bruchteil einer Chance, da das Kammerflimmern in eine Asystolie übergeht. Wenn man bedenkt, das der Patient gefunden, die Rettung alarmiert, der Ort angefahren und die Reanimation begonnen werden muss, kann es schon viel zu spät sein. Deswegen ist nicht nur eine suffiziente Reanimation wichtig, sondern auch eine Möglichkeit den Patienten als Laienhelfer kontrolliert defibrillieren zu können.

AEDs sind auch für nicht ausgebildete Laienhelfer äußerst einfach durch optische und akustische Anweisungen zu bedienen und können durch eine simple Bedienung und durch die integrierte Analysesoftware nicht fehlbedient werden. Seit 2001 wird in der breiten Öffentlichkeit der AED verstärkt beworben und darüber informiert. Es existieren viele Orte mit öffentlich zugänglichen AED, meist an Flughäfen, in Fussballstadien oder in öffentlichen Gebäuden. Das wichtigste ist aber, das diese AED nicht in Sanitätsräumen gelagert werden, sondern in der Öffentlichkeit einsehbar sind, wahrgenommen werden und von zufällig gegenwärtigen Laien als Notfallzeugen eingesetzt werden können.

In Amerika ist der AED schon weit verbreitet, in Europa wird er verstärkt aufgebaut, vor allem in den größeren Industrienationen ist der AED klar im kommen. Einzig Deutschland hat Probleme bei der flächendeckenden Verteilung der Geräte. Rechtlich ist der Einsatz im Rahmen der Ersten Hilfe für Laien unbedenklich.

Gerade Mecklenburg-Vorpommern, mit einer alternden Gesellschaft, die weit auseinander gezogen, teils entfernt von Rettungsdienststellen lebt aber auf der anderen Seite stark durch und vom Tourismus lebt, könnte somit zwei Ziele zur selben Zeit erreichen. Zuerst einmal würden die Sicherheit und die Überlebensrate für die Bevölkerung steigen und eine verbesserte Wahrnehmung von Mitteln der Ersten Hilfe in der Öffentlichkeit würde Ängste nehmen. Als zweites könnte es ein Kriterium für den Tourismus in Mecklenburg-Vorpommern, der Teils auch ein etwas erhöhtes Durchschnittsalter hat sein. Die Gefühlte und gelebte Sicherheit, sollte doch mal ein reanimationspflichtiger Notfall in der Öffentlichkeit eintreten.

\subsection{Hinweis}

Für die Texte habe ich mich durch den Wikipediaeintrag\footnote{\url{http://de.wikipedia.org/wiki/Automatisierter\_Externer\_Defibrillatorund}} einen Eintrag bei Thieme\footnote{\url{http://www.thieme.de/viamedici/aktuelles/artikel/aed.html}} inspirieren lassen.

Nach einem Gespräch mit der Wasserwacht kann ich noch folgende Informationen, die nicht im LQFB stehen, ergänzen:

\begin{itemize}
\item
  Der Akku/die Batterie hält ca 300 Schocks durch
\item
  Die Stromquelle lässt sich meist mit einem Griff austauschen
\item
  Die Stromquelle hält unbenutzt über vier Jahre
\item
  Das Gerät muss nicht jedes Jahr vom TÜV abgenommen/gewartet werden
\end{itemize}
\subsection{Fragen}

\begin{itemize}
\item
  F: Des weiteren würde mich der Kostenfaktor dieser Geräte interessieren.
\item
  A: Hier kannst du als Privatabnehmer schon im Bereich von 500€ beginnen und im Bereich um die 2000€ ankommen. So weit ich aber mitbekommen habe, ist bei AEDs Teuer+Marke=gut nicht mehr so wirklich zutreffend. Sogar ALDI hatte schobn gute, vollkommen ausreichende AEDs im Angebot, solange eben alle medizinischen entsprechenden Standards, Vorschriften und Qualitätsprüfungen erfüllt werden.
\end{itemize}
