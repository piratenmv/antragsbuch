\section{Antrag}

Die PIRATEN Mecklenburg-Vorpommern setzen sich für die Einordnung der Wasserrettung in die Aufgaben des Rettungsdienstes ein, was adäquat zur Gesetzgebung des Rettungsdienstgesetz in Brandenburg (RDG BB) erfolgen kann.

Des Weiteren sollen klare Bestimmungen zu Sicherung- und Rettungsvorkehrungen an Stränden, Flüssen und Binnengewässern, Rettungsorganistations und Leistungsträger übergreifend, beschlossen werden.

\section{Begründung}

Die Wasserrettung wird in Mecklenburg-Vorpommern von drei eingetragenen Vereinen ehrenamtlich geschultert. Dies sind die Wasserwachten des Deutschen Roten Kreuzes, der Deutschen Lebens-Rettungs-Gesellschaft und des Arbeiter-Samariter-Bundes. Im Rettungsdienstgesetz M-V (RDG MV) wird bis jetzt nur die Trägerschaft der Wasserrettung durch Kommunen und kreisfreie Städte geregelt.

Bis jetzt hat jede Organisation eigene Regelungen zu Personal, Qualifikation, Anforderungen und Equipment. Dazu können die Kommunen, die bewachten Badestellen und Strandabschnitte unterhalten und bezahlen, eigene Vorgaben zu den vorgehaltenen Rettungsmitteln geben. Die am besten ausgestatteten Stützpunkte in Mecklenburg-Vorpommern haben die Seebäder, da diese eine Notfallrettung vorhalten müssen, um weiterhin als Seebad angesehen zu werden. Diese finanzieren ihre Wasserrettung unter anderem mit und durch die Kurtaxe. Nichts desto trotz sollte für ganz Mecklenburg-Vorpommern einheitliche Standards geschaffen werden, dies würde die Rettung auch vergleichbar und evaluierbar machen.

Mit der Standardisierung würde wohl weiteres Personal an den Badestellen benötigt. Weniger in den Osteebädern, als mehr an kleineren Binnenseen und Badestellen, die erkennbaren Badebetrieb haben. Auch würde ein Argument der Gemeinden wegfallen, das sie gerne eine Wasserrettung finanzieren würden, aber da es keine verbindlichen Regeln gäbe, wüssten sie nicht wie.

Des Weiteren würde die Eingliederung auch Vorteile für die Kommunen und kreisfreie Städte bedeuten. Denn so müssten Einsätze, bei denen Patienten vor Ort, ähnlich dem Rettungsdienst versorgt werden und ins Krankenhaus müssen, auch von den Krankenkassen bezahlt werden. Dieses Geld würde aber nicht den Hilfsorganisationen oder den Rettungsschwimmern zu gute kommen, sondern an die Kommunen ausgezahlt werden.

Als letzten Punkt wurde die Hoffnung geäußert, dass durch die Eingliederung das Ansehen der Wasserrettung in der Öffentlichkeit ansteigt. So dass man die Wasserrettung nicht mehr nur als bessere Bademeister am Strand betrachtet, sondern wirklich als Rettungdienstleister und Helfer in der Not.

Zusammengefasst kann man sagen, eine Eingliederung der Wasserrettung in das Rettungsdienstgesetz M-V würde nicht nur die Qualität verbessern, die Kommunen etwas finanziell entlasten, sondern auch das Ansehen des Ehrenamtes verbessern.

\subsection{Hinweis}

Dieser Antrag wurde zum LPT12.1 schon einmal gestellt, hat den selben Antragstext, aber die Begründung wurde sehr groß umgestellt. Die Ursprüngliche LQFB-Iniziative\footnote{\url{https://lqpp.de/mv/initiative/show/100.html}} ist 100\% positiv beschieden worden.
