\section{Antrag}

Die Piratenpartei Mecklenburg-Vorpommern spricht sich gegen die Privatisierung von kommunalen Krankenhäusern und von Universitätskliniken aus. Zudem soll die Überführung von privaten Krankenhäusern in öffentliche oder freigemeinnützige Trägerschaft gefördert werden, etwa durch ein gesetzliches Vorkaufsrecht, Zuschüsse des Landes oder zinsverbilligte Kredite.

\section{Begründung}

Die stationäre medizinische Versorgung zählt zur existentiellen Infrastruktur, die vom Staat zu gewährleisten ist. In Mecklenburg-Vorpommern bestehen gegenwärtig sieben öffentliche, 15 gemeinnützige und 19 private Krankenhäuser.

Die Entwicklung im Krankenhaussektor ist seit zehn Jahren von drei Entwicklungen geprägt: der Zunahme von medizinisch nicht notwendigen Leistungen, dem Abbau von Vollkraftstellen und der Zunahme von Krankenhausprivatisierungen.

Diese Entwicklungen hängen mit der Einführung eines Abrechnungssystems nach Fallpauschalen seit 2003 zusammen. Dieses bietet Vorteile gegenüber dem früheren Vergütungssystem nach krankenhausindividuellen Pflegesätzen, schafft aber zugleich Fehlanreize durch einen Kostendruck, der zu medizinisch nicht notwendigen Leistungen, Personalabbau und Tarifflucht führt. Unter diesen Voraussetzungen sehen sich viele Kommunen veranlasst, die Trägerschaft für ihre Krankenhäuser aufzugeben und diese zu privatisieren. Das wiederum verschärft die Konkurrenzsituation zwischen den Krankenhäusern und zugleich den Kostendruck, weil zusätzlich noch die Renditeerwartungen der privaten Eigentümer befriedigt werden müssen. Öffentliche Krankenhäuser arbeiten dagegen nicht gewinnorientiert und sind an das Tarifrecht des Öffentlichen Dienstes gebunden, was tendenziell zu besseren Arbeitsbedingungen für die Mitarbeiter führt. Außerdem unterliegen sie durch ihre Eigentumsstruktur demokratischer Kontrolle.
