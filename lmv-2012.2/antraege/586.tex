\section{Antragstext}

In die Präambel der Satzung soll folgende Erklärung aufgenommen werden:

\begin{quote}
\minisec{Erklärung}

Wir Piraten sind Teil einer globalen Gemeinschaft von Menschen, unabhängig von Alter, Geschlecht, Abstammung und gesellschaftlicher Stellung. Wir schätzen die Verschiedenheit der Menschen, wir sind offen für alle mit neuen Ideen. Aber wir sind nicht offen für jede Idee.

Wer Menschen wegen ihrer Herkunft, ihrer geschlechtlichen Identität, ihrer sexuellen Orientierung, ihrer Religion oder einer Behinderung diskriminiert oder physische und psychische Gewalt gegen sie ausübt, wird mit uns keinen Dialog führen und hat keinen Platz bei uns.

Wir wissen, dass sich die Menschen nicht in Rassen einteilen lassen. Wir sind davon überzeugt, dass ein Nationalismus, der andere Nationen als nicht gleichwertig ansieht, das Zusammenleben in unserer auf Vielfalt beruhenden Gesellschaft bedroht. Wir sind uns angesichts der historischen und aktuellen faschistischen Gewalt in Deutschland unserer Verantwortung bewusst.

Die Piratenpartei Mecklenburg-Vorpommern erklärt das Vertreten von Rassismus und nationalem Chauvinismus sowie die Leugnung und Verharmlosung der faschistischen Gewalt für unvereinbar mit einer Mitgliedschaft.

\end{quote}
\section{Begründung}

Der Text orientiert sich an der ursprünglichen Unvereinbarkeitserklärung\footnote{\url{https://lqpp.de/mv/initiative/show/67.html}}, enthält aber einige Änderungen dazu. Der Text ist stärker auf die Piraten bezogen, Fremdwörter sind zum Teil aufgelöst oder erklärt worden. Der Text enthält einige Begründungselemente. Der Begriff der strukturellen Gewalt wurde vermieden, religiöse Diskriminierung ist dafür aufgenommen.
