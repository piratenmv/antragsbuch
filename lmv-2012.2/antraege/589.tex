\section{Antragstext}

In die Präambel der Satzung soll folgende Erklärung aufgenommen werden:

\begin{quote}
\minisec{Erklärung}

Wir sind eine Gemeinschaft von Menschen verschiedenen Alters, Sprache, Kultur und Abstammung sowie gesellschaftlicher Stellung, offen für alle mit neuen Ideen.

Gewaltfreiheit und der respektvolle Umgang untereinander, unabhängig von Herkunft, Religion, Geschlecht oder äußerem Erscheinungsbild bilden unseren Anspruch an die politische Diskussion.

Wir sind uns der historischen Rolle Deutschlands bewusst und stehen heute umso mehr für ein friedliches Zusammenleben der Gesellschaften in unserem Land und Europa.

Die Piratenpartei Mecklenburg-Vorpommern tritt für eine nachhaltige Entwicklung von Gesellschaft und Natur in unserem Bundesland und darüber hinaus ein. Wir begrüßen die Beteiligung jedes Einzelnen und laden all diejenigen zum politischen Gespräch ein, die sich mit unseren Grundsätzen verbunden fühlen.

\end{quote}
\section{Begründung}

Die Piratenpartei steht für bestimmte Grundsätze wie Bürgerbeteiligung, Transparenz, Gleichberechtigung, Meinungsvielfalt, Mitmachgedanke usw. . Eine Satzungspräambel soll über diese Grundsätze formuliert werden. Wichtig ist die Vermittlung des Gedankens, warum man bei den Piraten mitmachen bzw. sich allgemein politisch/gesellschaftlich engagieren sollte. Dieser Antrag steht in Konkurrenz zu anderen Vorschlägen zur Ausgestaltung der Satzungspräambel.
