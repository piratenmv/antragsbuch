\section{Antragstext}

Um die finanziellen Möglichkeiten des Landesverbandes nicht ohne Not einzuschränken, sollen Firmenspenden nicht verboten werden. Spenden juristischer Personen an die den Landesverband sollen jedoch für die Öffentlichkeit transparent (d.h. sichtbar) sein und zwar mit Angabe des Spenders, der Höhe der Spende, dem Zeitpunkt der Spende, der Parteigliederung, die die Spende erhalten hat und ggf. weiteren Angaben. Bei zweckgebundenen Spenden ist der Zweck ebenfalls zu veröffentlichen. Spenden für nicht satzungsgemäße Zwecke sind von der begünstigten Parteigliederung zurückzuweisen.

\section{Vergleich mit dem Status Quo}

Die Paragraphen 10 und 11 der Finanzordnung des Bundes sehen eine Veröffentlichungspflicht ab einer Summe von 10.000 Euro pro Jahr im Rechenschaftsbericht vor. Alle Einzelspenden über 1.000 Euro werden unverzüglich unter Angabe von Spendernamen, Summe und ggf. Verwendungszweck veröffentlicht.

In dieser Initiative wird \emph{zusätzlich} gefordert:

\begin{itemize}
\item
  Ausnahmslose Veröffentlichung \emph{aller} Spenden juristischer Personen.
\end{itemize}
\section{Begründung}

Die Initiative ist weitgehend aus den im Bund abgestimmten Initiativen 1623\footnote{\url{https://lqfb.piratenpartei.de/lf/initiative/show/1623.html}} (September 2011, Zustimmung 91\%) und 826\footnote{\url{https://lqfb.piratenpartei.de/lf/initiative/show/826.html}} (August 2010, Zustimmung 88\%) übernommen.
