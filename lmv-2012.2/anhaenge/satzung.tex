\subsubsection{§ 1 - Name, Sitz und Tätigkeitsgebiet}

(1) \textsuperscript{1}Der Landesverband Mecklenburg-Vorpommern der
Piratenpartei Deutschland ist ein untergeordneter Gebietsverband auf
Landesebene gemäß der Satzung der Piratenpartei Deutschland
(Bundessatzung\textsuperscript{\href{\#cite\_note-0}{{[}1{]}}}).
\textsuperscript{2}Der Sitz des Landesverbandes und Ort der
Landesgeschäftsstelle ist Rostock.

(2) \textsuperscript{1}Der Landesverband Mecklenburg-Vorpommern der
Piratenpartei Deutschland führt einen Namen und eine Kurzbezeichnung.
\textsuperscript{2}Der Name lautet: \textbf{Piratenpartei Deutschland,
Landesverband Mecklenburg-Vorpommern}. \textsuperscript{3}Die offizielle
Abkürzung des Landesverbandes Mecklenburg-Vorpommern der Piratenpartei
Deutschland lautet: PIRATEN. \textsuperscript{4}Die Verwendung des
verkürzten Namens ``Piratenpartei MV'' ist zulässig.

(3) \textsuperscript{1}Untergeordnete Gliederungen des Landesverbandes
Mecklenburg-Vorpommern der Piratenpartei Deutschland führen den Namen
Piratenpartei Deutschland verbunden mit ihrer Organisationsstellung und
dem Namen der Gliederung. \textsuperscript{2}Den untergeordneten
Gliederungen wird die Verkürzung auf ``Piratenpartei'' in Verbindung mit
dem Gliederungsnamen erlaubt.

(4) \textsuperscript{1}Das Tätigkeitsgebiet des Landesverbandes
Mecklenburg-Vorpommern der Piratenpartei Deutschland ist das Bundesland
Mecklenburg-Vorpommern.

(5) \textsuperscript{1}Die im Landesverband Mecklenburg-Vorpommern der
Piratenpartei Deutschland organisierten Mitglieder werden
geschlechtsneutral als Piraten bezeichnet.

\subsubsection{§ 2 - Mitgliedschaft}

(1) \textsuperscript{1}Mitglied des Landesverbandes ist jedes Mitglied
der Piratenpartei Deutschland mit angezeigtem Wohnsitz in
Mecklenburg-Vorpommern.

(2) \textsuperscript{1}Der Landesverband führt ein Piratenverzeichnis.

(3) \textsuperscript{1}Untergliederungen können ein eigenes
Piratenverzeichnis führen.

\subsubsection{§ 3 - Erwerb der Mitgliedschaft}

(1) \textsuperscript{1}Der Erwerb der Mitgliedschaft der Piratenpartei
Deutschland wird durch die
Bundessatzung\textsuperscript{\href{\#cite\_note-1}{{[}2{]}}} geregelt.

(2) \textsuperscript{1}Jegliche Änderung am Bestand der Mitgliedsdaten
muss allen übergeordneten Gliederungen mitgeteilt werden.

\subsubsection{§ 4 - Rechte und Pflichten der Piraten}

\textsuperscript{1}Um eine Gleichbehandlung aller Piraten im
Landesverband zu gewährleisten, werden die Rechte und Pflichten der
Piraten des Landesverbandes allein durch die
Bundessatzung\textsuperscript{\href{\#cite\_note-2}{{[}3{]}}} geregelt.
\textsuperscript{2}Eine hiervon abweichende Regelung durch
untergeordnete Gliederungen ist unzulässig.

\subsubsection{§ 5 - Beendigung der Mitgliedschaft}

(1) \textsuperscript{1}Die Beendigung der Mitgliedschaft ist der
niedrigsten Gliederung anzuzeigen, die ein Piratenverzeichnis führt.

(2) \textsuperscript{1}Die Beendigung der Mitgliedschaft in der
Piratenpartei Deutschland wird durch die
Bundessatzung\textsuperscript{\href{\#cite\_note-3}{{[}4{]}}} geregelt.

(3) \textsuperscript{1}Die Beendigung der Mitgliedschaft im
Landesverband erfolgt durch Wechsel des Wohnsitzes in ein anderes
Bundesland oder durch Beendigung der Mitgliedschaft in der Piratenpartei
Deutschland.

\subsubsection{§ 6 - Ordnungsmaßnahmen}

\textsuperscript{1}Die Regelungen zu den Ordnungsmaßnahmen, die in der
Bundessatzung\textsuperscript{\href{\#cite\_note-4}{{[}5{]}}} getroffen
werden, gelten entsprechend auch auf Landesebene.

\subsubsection{§ 7 - Gliederung}

(1) \textsuperscript{1}Der Landesverband gliedert sich in Kreisverbände
und Ortsverbände. \textsuperscript{2}Kreisverbände können sich über das
Gebiet mehrerer aneinander angrenzender Kreise und kreisfreier Städte
erstrecken, Ortsverbände über das Gebiet mehrerer aneinander
angrenzender Gemeinden.

(2) \textsuperscript{1}Auf Verlangen von mindestens drei
gründungswilligen Piraten lädt der Landesvorstand alle Piraten mit
angezeigtem Wohnsitz im Gebiet des künftigen Kreisverbands zu einer
Gründungsversammlung ein. \textsuperscript{2}Ort und Zeit der
Gründungsversammlung werden von den gründungswilligen Piraten bestimmt,
wobei die Ladungsfrist mindestens vier Wochen beträgt.
\textsuperscript{3}Die Gründungsversammlung ist beschlussfähig, wenn
mindestens sieben stimmberechtigte Piraten erschienen sind.
\textsuperscript{4}Der Kreisverband ist errichtet, wenn auf der
Gründungsversammlung dessen Satzung beschlossen worden ist.
\textsuperscript{5}Für den Beschluss ist eine Mehrheit von 2/3 der
abgegebenen Stimmen erforderlich. \textsuperscript{6}Über die
Versammlung ist ein Protokoll anzufertigen und zu veröffentlichen.

(3) \textsuperscript{1}Für die Gründung von Ortsverbänden gilt Absatz 2
entsprechend, solange der zuständige Kreisverband keine andere Regelung
trifft.

\subsubsection{§ 8 - Bundespartei und Landesverbände}

\textsuperscript{1}Der Landesverband verpflichtet sich, den Regelungen
des Bundessatzung\textsuperscript{\href{\#cite\_note-5}{{[}6{]}}}
bezüglich des Verhältnisses von Bundespartei und Landesverbänden Folge
zu leisten und seine untergeordnete Gliederungen zu ebensolchem
Verhalten anzuhalten.

\subsubsection{§ 9 - Organe des Landesverbands}

\textsuperscript{1}Organe sind die Landesmitgliederversammlung, das
Landesschiedsgericht und der Vorstand.

\subsubsection{§ 9a - Der Vorstand}

(1) \textsuperscript{1}Der Vorstand besteht aus dem Vorsitzenden, dem
stellvertretenden Vorsitzenden, dem Schatzmeister, dem politischen
Geschäftsführer und dem Generalsekretär.

(2) \textsuperscript{1}Der Vorstand vertritt den Landesverband nach
innen und außen. \textsuperscript{2}Er führt die Geschäfte auf Grundlage
der Beschlüsse der Parteiorgane.

(3) \textsuperscript{1}Die Mitglieder des Vorstands werden von einer
Landesmitgliederversammlung mindestens jährlich in geheimer Wahl
gewählt. \textsuperscript{2}Der Vorstand bleibt bis zur Wahl eines neuen
Vorstands im Amt.

(4) \textsuperscript{1}Der Vorstand tritt in seiner Amtsperiode
mindestens zweimal zusammen. \textsuperscript{2}Er wird vom Vorsitzenden
oder bei dessen Verhinderung vom stellvertretendem Vorsitzenden mit
einer Frist von zwei Wochen unter Angabe der Tagesordnung und des
Tagungsortes einberufen. \textsuperscript{3}Bei außerordentlichen
Anlässen kann die Einberufung auch kurzfristiger erfolgen.

(5) \textsuperscript{1}Auf Antrag eines Zehntels der Piraten kann der
Vorstand zum Zusammentritt aufgefordert und mit aktuellen
Fragestellungen befasst werden. \textsuperscript{2}Die aktuelle
Mitgliederzahl ist regelmäßig zu veröffentlichen.

(6) \textsuperscript{1}Der Vorstand beschließt über alle
organisatorischen und politischen Fragen im Sinne der Beschlüsse der
Landesmitgliederversammlung.

(7) \textsuperscript{1}Der Vorstand gibt sich eine Geschäftsordnung und
veröffentlicht diese angemessen. \textsuperscript{2}Sie umfasst u.a.
Regelungen zu:

\begin{enumerate}
\item
  Verwaltung der Mitgliedsdaten und deren Zugriff und Sicherung
\item
  Aufgaben und Kompetenzen der Vorstandsmitglieder
\item
  Dokumentation der Sitzungen
\item
  virtuellen oder fernmündlichen Vorstandssitzungen
\item
  Form und Umfang des Tätigkeitsberichts
\item
  Beurkundung von Beschlüssen des Vorstandes
\end{enumerate}
(8) \textsuperscript{1}Die Führung der Landesgeschäftsstelle wird durch
den Vorstand beauftragt und beaufsichtigt.

(9) \textsuperscript{1}Der Vorstand liefert zur
Landesmitgliederversammlung einen schriftlichen Tätigkeitsbericht ab.
\textsuperscript{2}Dieser umfasst alle Tätigkeitsgebiete der
Vorstandsmitglieder, wobei diese in Eigenverantwortung des Einzelnen
erstellt werden. \textsuperscript{3}Wird der Vorstand insgesamt oder ein
Vorstandsmitglied nicht entlastet, so kann die
Landesmitgliederversammlung oder der neue Vorstand gegen ihn Ansprüche
gelten machen. \textsuperscript{4}Tritt ein Vorstandsmitglied zurück,
hat dieser unverzüglich einen Tätigkeitsbericht zu erstellen und dem
Vorstand zuzuleiten.

(10) \textsuperscript{1}Tritt ein Vorstandsmitglied zurück bzw. kann
dieses seinen Aufgaben nicht mehr nachkommen, so geht seine Kompetenz
wenn möglich auf ein anderes Vorstandsmitglied über.
\textsuperscript{2}Der Vorstand gilt als nicht handlungsfähig, wenn mehr
als zwei Vorstandsmitglieder zurückgetreten sind oder ihren Aufgaben
nicht mehr nachkommen können oder wenn der Vorstand sich selbst für
handlungsunfähig erklärt. \textsuperscript{3}In einem solchen Fall wird
von dem dienstältesten Vorstand der direkt untergeordneten
Gliederungsebene zur Geschäftsführung eine kommissarische Vertretung
bestimmt. \textsuperscript{4}Die kommissarische Vertretung endet mit der
Neuwahl des gesamten Vorstandes auf einer unverzüglich einberufenen
außerordentlichen Landesmitgliederversammlung.

(11) \textsuperscript{1}Tritt der gesamte Vorstand geschlossen zurück
oder kann seinen Aufgaben nicht mehr nachkommen, so führt der
dienstälteste Vorstand der direkt untergeordneten Gliederungsebene
kommissarisch die Geschäfte bis eine von ihm unverzüglich einberufenen
außerordentlichen Landesmitgliederversammlung einen neuen Vorstand
gewählt hat.

\subsubsection{§ 9b - Die Landesmitgliederversammlung}

(1) \textsuperscript{1}Die Landesmitgliederversammlung ist die
Mitgliederversammlung auf Landesebene.

(2) \textsuperscript{1}Die Landesmitgliederversammlung tagt mindestens
einmal jährlich als Realversammlung. \textsuperscript{2}Die Einberufung
erfolgt aufgrund Vorstandsbeschluss. \textsuperscript{3}Der Vorstand
lädt jedes Mitglied persönlich mindestens vier Wochen vor der
Landesmitgliederversammlung in Textform (vorrangig per E-Mail,
nachrangig per Brief) ein. \textsuperscript{4}Die Einladung hat Angaben
zum Tagungsort, Tagungsbeginn, vorläufiger Tagesordnung und der Angabe,
wo weitere, aktuelle Veröffentlichungen gemacht werden, zu enthalten.
\textsuperscript{5}Spätestens eine Woche vor der
Landesmitgliederversammlung sind die Tagesordnung in aktueller Fassung,
die geplante Tagungsdauer und alle bis dahin dem Vorstand eingereichten
Anträge im Wortlaut zu veröffentlichen.

(3) \textsuperscript{1}Eine außerordentliche Landesmitgliederversammlung
wird unverzüglich einberufen, wenn mindestens eins der folgenden
Ereignisse eintritt:

\begin{enumerate}
\item
  Der Vorstand ist handlungsunfähig.
\item
  Ein Zehntel der stimmberechtigten Piraten des Landesverbandes
  Mecklenburg-Vorpommern beantragt es.
\item
  Der Landesvorstand beschließt es mit einer Zweidrittelmehrheit.
\end{enumerate}
\textsuperscript{2}Es ist ein Grund für die Einberufung zu benennen.
\textsuperscript{3}Die außerordentliche Landesmitgliederversammlung darf
sich nur mit dem benannten Grund der Einberufung befassen.
\textsuperscript{4}In dringenden Fällen kann mit einer verkürzten Frist
von mindestens zwei Wochen eingeladen werden.

(4) \textsuperscript{1}Die Landesmitgliederversammlung nimmt den
Tätigkeitsbericht des Vorstandes entgegen und entscheidet daraufhin über
seine Entlastung.

(5) \textsuperscript{1}Über die Landesmitgliederversammlung, deren
Beschlüsse und Wahlen wird ein Ergebnisprotokoll gefertigt, das von der
Protokollführung, der Versammlungsleitung und der Wahlleitung
unterschrieben und anschließend veröffentlicht wird.
\textsuperscript{2}Die Entscheidungen der Landesmitgliederversammlung
werden mit einfacher Mehrheit der abgegebenen gültigen Stimmen
beschlossen. Bei Stimmengleichheit gilt ein Antrag als abgelehnt.
\textsuperscript{3}Stimmenthaltungen werden als ungültige Stimmen
gewertet.

(6) \textsuperscript{1}Die Landesmitgliederversammlung wählt mindestens
zwei Rechnungsprüfer, die den finanziellen Teil des Tätigkeitsberichtes
des Vorstandes vor der Beschlussfassung über ihn prüfen.
\textsuperscript{2}Das Ergebnis der Prüfung wird der
Landesmitgliederversammlung verkündet und zu Protokoll genommen.
\textsuperscript{3}Danach sind die Rechnungsprüfer aus ihrer Funktion
entlassen.

(7) \textsuperscript{1}Die Landesmitgliederversammlung wählt mindestens
zwei Kassenprüfer. \textsuperscript{2}Diesen obliegen die Vorprüfung des
finanziellen Tätigkeitsberichtes für die folgende
Landesmitgliederversammlung und die Vorprüfung, ob die Finanzordnung und
das Parteiengesetz eingehalten wird. \textsuperscript{3}Sie haben das
Recht, Einsicht in alle finanzrelevanten Unterlagen zu verlangen und auf
Wunsch Kopien persönlich ausgehändigt zu bekommen.
\textsuperscript{4}Sie sind angehalten, etwa zwei Wochen vor der
Landesmitgliederversammlung die letzte Vorprüfung der Finanzen
durchzuführen. \textsuperscript{5}Ihre Amtszeit endet durch Austritt,
Rücktritt, Entlassung durch die Landesmitgliederversammlung oder mit
Wahl ihrer Nachfolger.

(8) \textsuperscript{1}Die Landesmitgliederversammlung tagt daneben
online und nach den Prinzipien von Liquid Democracy als Ständige
Mitgliederversammlung. \textsuperscript{2}Jeder Pirat im Landesverband
Mecklenburg-Vorpommern hat das Recht, an der Ständigen
Mitgliederversammlung teilzunehmen. \textsuperscript{3}Das Stimmrecht
richtet sich nach § 4 Abs. 4 der Bundessatzung.

(9) \textsuperscript{1}Die Ständige Mitgliederversammlung kann für den
Landesverband verbindliche Stellungnahmen und Positionspapiere
beschließen. \textsuperscript{2}Entscheidungen über die Parteiprogramme,
die Satzung, die Beitragsordnung, die Schiedsgerichtsordnung, die
Auflösung sowie die Verschmelzung mit anderen Parteien (§ 9 Abs. 3
Parteiengesetz) sind ausgeschlossen, insoweit kann die Ständige
Mitgliederversammlung nur Empfehlungen abgeben.

(10) \textsuperscript{1}Die Landesmitgliederversammlung beschließt die
Geschäftsordnung der Ständigen Mitgliederversammlung, in der auch die
Konstituierung der Ständigen Mitgliederversammlung geregelt ist.

\subsubsection{§ 10 - Bewerberaufstellung für die Wahlen zu
Volksvertretungen}

(1) \textsuperscript{1}Die Bewerberaufstellung für die Wahlen zu
Volksvertretungen erfolgt nach den Regularien der einschlägigen Gesetze
sowie den Vorgaben der
Bundessatzung\textsuperscript{\href{\#cite\_note-6}{{[}7{]}}}.

(2) \textsuperscript{1}Die Aufstellung kann sowohl als
Mitgliederversammlung des zuständigen Stimm- bzw. Wahlkreises als auch
im Rahmen einer anderen Mitgliederversammlung stattfinden, sofern
gewährleistet wird, dass alle Stimmberechtigten in angemessener Zeit und
Form eingeladen wurden und nur die Stimmberechtigten an der Wahl
teilnehmen. \textsuperscript{2}Die Einladung muss dabei explizit auf die
Bewerberaufstellung hinweisen.

\subsubsection{§ 11 - Satzungs- und Programmänderung}

(1) \textsuperscript{1}Änderungen der Landessatzung und des Programms
können nur von einer Landesmitgliederversammlung mit einer
Zweidrittelmehrheit der abgegebenen gültigen Stimmen beschlossen werden.
\textsuperscript{2}Besteht das dringende Erfordernis einer
Satzungsänderung zwischen zwei Landesmitgliederversammlungen, so kann
die Satzung auch geändert werden, wenn mindestens zwei Dritteln der
Piraten dem Änderungsantrag schriftlich zustimmen.

(2) \textsuperscript{1}Über einen Antrag auf Satzungs- oder
Programmänderung auf einer Landesmitgliederversammlung kann nur
abgestimmt werden, wenn er mindestens zwei Wochen vor Beginn der
Landesmitgliederversammlung beim Vorstand eingegangen ist.

(3) \textsuperscript{1}Der Landesverband übernimmt das Grundsatzprogramm
der Piratenpartei Deutschland. \textsuperscript{2}Die
Landesmitgliederversammlung stellt ein landes- und kommunalpolitisches
Programm auf und schreibt dieses fort. \textsuperscript{3}Die
Landesmitgliederversammlung kann auf dieser Grundlage ein eigenes
Wahlprogramm für Kommunal- und Landtagswahlen beschließen.
\textsuperscript{4}Alle Programme müssen auf den Werten des
Grundsatzprogramms basieren.

\subsubsection{§ 12 - Auflösung und Verschmelzung}

\textsuperscript{1}Die Auflösung oder Verschmelzung regelt die
Bundessatzung\textsuperscript{\href{\#cite\_note-7}{{[}8{]}}}.

\subsubsection{§ 13 - Parteiämter}

\textsuperscript{1}Die Regelung der
Bundessatzung\textsuperscript{\href{\#cite\_note-8}{{[}9{]}}} zu den
Parteiämtern findet Anwendung.

\subsection{Abschnitt B: Finanzordnung}

\subsubsection{§ 16 Finanzordnung}

\textsuperscript{1}Die Finanzordnung der
Bundessatzung\textsuperscript{\href{\#cite\_note-9}{{[}10{]}}} findet
entsprechende Anwendung.

\subsection{Abschnitt C: Schiedsgerichtsordnung}

\subsubsection{§ 15 Landesschiedsgericht}

\textsuperscript{1}Für das Landesschiedsgericht gilt die
Bundesschiedsgerichtsordnung\textsuperscript{\href{\#cite\_note-10}{{[}11{]}}}.

\subsection{Abschnitt D: Organisatorisches}

\subsubsection{§ 16 Wahlordnung}

\textsuperscript{1}Die Landesmitgliederversammlung regelt das Verfahren
von Wahlen und Abstimmungen in einer
Wahlordnung\textsuperscript{\href{\#cite\_note-11}{{[}12{]}}}.

\subsubsection{§ 17 Schlussbestimmungen}

\textsuperscript{1}Diese Satzung tritt mit Verabschiedung durch die
Landesmitgliederversammlung in Kraft.

\subsection{Referenzen}

\begin{enumerate}
\item
  \href{\#cite\_ref-0}{↑} \href{/Bundessatzung}{Bundessatzung}
\item
  \href{\#cite\_ref-1}{↑}
  \href{/Bundessatzung\#.C2.A7\_3\_-\_Erwerb\_der\_Mitgliedschaft}{§ 3
  Bundessatzung (``Erwerb der Mitgliedschaft'')}
\item
  \href{\#cite\_ref-2}{↑}
  \href{/Bundessatzung\#.C2.A7\_4\_-\_Rechte\_und\_Pflichten\_der\_Piraten}{§
  4 Bundessatzung (``Rechte und Pflichten der Piraten'')}
\item
  \href{\#cite\_ref-3}{↑}
  \href{/Bundessatzung\#.C2.A7\_5\_-\_Beendigung\_der\_Mitgliedschaft}{§
  5 Bundessatzung (``Beendigung der Mitgliedschaft'')}
\item
  \href{\#cite\_ref-4}{↑}
  \href{/Bundessatzung\#.C2.A7\_6\_-\_Ordnungsma.C3.9Fnahmen}{§ 6
  Bundessatzung (``Ordnungsmaßnahmen'')}
\item
  \href{\#cite\_ref-5}{↑}
  \href{/Bundessatzung\#.C2.A7\_8\_-\_Bundespartei\_und\_Landesverb.C3.A4nde}{§
  8 Bundessatzung (``Bundespartei und Landesverbände'')}
\item
  \href{\#cite\_ref-6}{↑}
  \href{/Bundessatzung\#.C2.A7\_10\_-\_Bewerberaufstellung\_f.C3.BCr\_die\_Wahlen\_zu\_Volksvertretungen}{§
  10 Bundessatzung (``Bewerberaufstellung für die Wahlen zu
  Volksvertretungen'')}
\item
  \href{\#cite\_ref-7}{↑}
  \href{/Bundessatzung\#.C2.A7\_13\_-\_Aufl.C3.B6sung\_und\_Verschmelzung}{§
  13 Bundessatzung (``Auflösung und Verschmelzung'')}
\item
  \href{\#cite\_ref-8}{↑}
  \href{/Bundessatzung\#.C2.A7\_15\_-\_Partei.C3.A4mter}{§ 15
  Bundessatzung (``Parteiämter'')}
\item
  \href{\#cite\_ref-9}{↑}
  \href{/Bundessatzung\#Abschnitt\_B:\_Finanzordnung}{Abschnitt B
  Bundessatzung (``Finanzordnung'')}
\item
  \href{\#cite\_ref-10}{↑}
  \href{/Bundessatzung\#Abschnitt\_C:\_Schiedsgerichtsordnung}{Abschnitt
  C Bundessatzung (``Schiedsgerichtsordnung'')}
\item
  \href{\#cite\_ref-11}{↑} \href{/MV:Wahlordnung}{Wahl- und
  Abstimmungsordnung der Piraten in Mecklenburg-Vorpommern}
\end{enumerate}
