\subsection{Allgemeines}

\subsubsection{§ 1 Befugnisse}

(1) Nimmt ein Pirat gar nicht oder nicht an der gesamten Versammlung
teil, so entstehen hieraus keine rückwirkenden Rechte; insbesondere
ergibt sich daraus keine Rechtfertigung für eine Anfechtung von
Wahlergebnissen oder Beschlüssen.

(2) Ämter und Befugnisse der Versammlung enden mit dem Ende der
Versammlung.

\subsubsection{§ 2 Akkreditierung}

(1) Stimmberechtigte Mitglieder der Versammlung im Sinne dieser
Geschäftsordnung sind alle akkreditierten Piraten.

(2) Alle im Sinne der Satzung stimmberechtigten Piraten werden von einem
Vertreter des Landesverbands akkreditiert. Dabei erhält jeder
stimmberechtigte Pirat eine Stimmkarte und einen Stimmzettelblock.

(3) Die für die Akkreditierung zuständigen Personen führen eine Liste
der akkreditierten Piraten.

(4) Beim vorzeitigen Verlassen des Parteitags hat ein akkreditiertes
Mitglied sich bei den dafür zuständigen Personen zu deakkreditieren. Ein
vorübergehendes Verlassen des Parteitags bedarf keiner Deakkreditierung.

\subsubsection{§ 3 Grundlegende Regeln für Wahlen und Abstimmungen}

(1) Alle Abstimmungen und Wahlen finden grundsätzlich mit einfacher
Mehrheit der gültig abgegebenen Stimmen und offen mit Handzeichen statt,
sofern nicht diese Geschäftsordnung, die Satzung oder ein Gesetz anderes
bestimmt.

(2) Für offene Wahlen und Abstimmungen erhält jeder Stimmberechtigte
zwei Stimmkarten, die durch Farbe, Symbol und Beschriftung als »Ja« und
»Nein« gekennzeichnet sind. Bei Abstimmungen wird gleichzeitig, bei
Bedarf auch nacheinander, nach Ja- und Nein-Stimmen gefragt, es ist die
jeweils gewünschte Stimmkarte zu zeigen. Enthaltungen werden nicht
gezählt.

(3) Jeder Stimmberechtigte kann eine geheime Wahl \textbf{\{GO-Antrag
auf geheime Wahl\}} oder geheime Abstimmung \textbf{\{GO-Antrag auf
geheime Abstimmung\}} beantragen. Geschäftsordnungsanträge werden immer
offen abgestimmt.

(4) Bei einer geheimen Wahl oder Abstimmung wird mit einem nummerierten
Stimmzettel gewählt. Die Nummer des Stimmzettels wird durch den
Wahlleiter bekannt gegeben. Die Wahlgangnummer und die
Stimmzettelenummer werden in jedem Wahlgang übereinstimmend verwendet.

(4a) Bei Abstimmungen über nur einen Antrag und bei Wahlen mit nur einem
Kandidaten muss genau eine der folgenden Optionen ausgewählt werden:

\begin{itemize}
\item
  1 für ``Ja''
\item
  2 für ``Nein''
\end{itemize}
(4b) Bei Abstimmungen über mehrere Anträge und bei Wahlen mit mehreren
Kandidaten findet eine Akzeptanzwahl statt. Jeder Stimmberechtigte hat
so viele Stimmen, wie Anträge bzw. Kandidaten zur Auswahl stehen, darf
für jeden Antrag bzw. Kandidaten jedoch nicht mehr als eine Stimme
abgeben. Es dürfen die Nummern auf dem Stimmzettel ausgewählt werden,
die vom Wahlleiter den Anträgen bzw. Kandidaten zugeordnet wurden. Ein
leerer Stimmzettel lehnt alle Anträge bzw. Kandidaten ab.

(4c) Anders ausgefüllte Stimmzettel sind ungültig. Enthaltung ist durch
Abgeben keines oder eines ungültigen Stimmzettels möglich.

(5) Das Ergebnis einer offenen Wahl oder Abstimmung wird vom
Versammlungsleiter nach Augenmaß festgestellt und mitgeteilt. Bei
unklaren Verhältnissen oder auf Antrag der Versammlung beauftragt die
Versammlungsleitung die Wahlleitung mit der Auszählung.
\textbf{\{GO-Antrag auf Auszählung\}}

(6) Wurden Stimmen ausgezählt, z.B. bei einer geheimen Wahl oder
Abstimmung, teilt der Wahlleiter der Versammlung das Ergebnis nach
Abschluss der Auszählung mit. Dieses besteht aus der Anzahl der auf jede
mögliche Option entfallenen Stimmen, bei geheimen Wahlen und
Abstimmungen auch aus der Anzahl der Stimmberechtigten für diese Wahl
oder Abstimmung und der Anzahl der ungültigen Stimmen und Enthaltungen.

(7) Alle Piraten, insbesondere jedoch die Wahlhelfer, sind verpflichtet,
Vorkommnisse, die die Rechtmäßigkeit der Wahl oder Abstimmung in Frage
stellen, sofort dem Wahlleiter bekannt zu machen, der unverzüglich die
Versammlung darüber in Kenntnis zu setzen hat.

(8) Bei Unklarheit des Ergebnisses findet eine Wiederholung der Wahl
oder Abstimmung statt. Um das sicherzustellen, kann die Wiederholung
beantragt werden \textbf{\{GO-Antrag auf Wiederholung der
Wahl/Abstimmung\}}.

(9) Findet die Wiederholung einer Wahl oder Abstimmung nicht unmittelbar
nach der ursprünglichen Wahl statt, so muss die Beteiligung an der Wahl
oder Abstimmung (gemessen an der Summe der zustimmenden und ablehnenden
Stimmen) bei mindestens 90\% der ursprünglichen Wahl oder Abstimmung
liegen, damit das neue Ergebnis rechtskräftig wird.

(10) Die Wahlleitung kann akkreditierten Piraten, die sich außerhalb des
Sitzungssaales befinden, nach eigenem Ermessen eine Beteiligung an den
Wahlen und Abstimmungen des Landesparteitags ermöglichen.

\subsubsection{§ 4 Ordnungsmaßnahmen}

(1) Ordnungsmaßnahmen werden gegen Anwesende verhängt, die gegen die
Geschäftsordnung verstoßen, den Ablauf des Parteitags grob stören oder
die grundsätzliche Ordnung des Parteitags verletzen.

(2) Ordnungsmaßnahmen sind während der gesamten Versammlung gültig. Sie
können vom verhängenden Parteitagsorgan jederzeit während der
Versammlung revidiert werden.

(3) Die Maßnahme des Ordnungsrufs wird durch die Versammlungsleitung
verhängt und dient der Verwarnung.

(4) Die Maßnahme des Verweises wird durch die Versammlungsleitung
verhängt und dient der verschärften Verwarnung. Die Maßnahme ist mit dem
Namen des Betroffenen oder falls zutreffend der Mitgliedsnummer zu
Protokoll zu geben.

(5) Die Maßnahme des Ausschlusses vom Parteitag wird auf Antrag der
Versammlungsleitung selbst durch die Versammlung verhängt.

\subsection{Versammlungsämter}

\subsubsection{§ 5 Versammlungsämter}

(1) Die Versammlung bestimmt eine Versammlungsleitung, eine Wahlleitung
und eine Protokollführung.

(2) Die Amtszeit von Versammlungsämtern beginnt mit der Bestimmung des
jeweiligen Versammlungsamtes und endet mit dem Ende der Versammlung,
Rücktritt oder durch Abberufung durch die Versammlung.

(3) Bei Rücktritt von einem Parteitagsamt ist unverzüglich eine
Nachfolgebesetzung zu bestimmen.

(4) Bis zur Bestimmung einer Versammlungsleitung und Protokollführung
durch die Versammlung setzt der Landesvorstand eine kommissarische
Versammlungsleitung und eine kommissarische Protokollführung ein.

\subsubsection{§ 6 Versammlungsleitung}

(1) Die Versammlung wird durch einen Versammlungsleiter geleitet, der
möglichst zu Beginn von dieser gewählt wird. Der Versammlungsleiter
fungiert ebenfalls als Leiter im Sinne des § 8 VersammlG.

(2) Der Versammlungsleiter kann mehrere Versammlungsleiterhelfer
festlegen, sofern es keinen Widerspruch gibt. Versammlungsleiterhelfer
können dem Versammlungsleiter bei Aufgaben helfen bzw. Aufgaben
übernehmen sowie den Versammlungsleiter auf dessen Wunsch vertreten. Die
Vertretung ist als Versammlungsleiterwechsel im Protokoll zu vermerken.

(3) Dem Versammlungsleiter obliegt die Einhaltung der Tagesordnung inkl.
Zeitplan. Dazu teilt er Rederecht inkl. Redezeit zu bzw. entzieht diese,
wobei eine angemessene inhaltliche wie personale Diskussion und
Beteiligung der einzelnen Piraten sichergestellt werden muss. Jedem
stimmberechtigten Pirat kann auf Verlangen eine angemessene Redezeit
eingeräumt werden. Sind Gäste zugelassen, so kann der Versammlungsleiter
diesen ein Rederecht einräumen, sofern es keinen Widerspruch gibt.

(4) Der Versammlungsleiter kündigt Beginn und Ende von
Sitzungsunterbrechungen sowie den Zeitpunkt der Neuaufnahme der
Versammlung nach einer Vertagung an.

(5) Grundsätzlich stellt der Versammlungsleiter die Ergebnisse von
Wahlen und Abstimmungen fest, sofern dafür nicht ausdrücklich der
Wahlleiter vorgesehen ist. Er kann den Wahlleiter grundsätzlich, für
weitere Wahlen (z.B. zu Versammlungsämtern) oder auch für bestimmte
einzelne Abstimmungen beauftragen, ihn bei der Feststellung von
Abstimmungsergebnissen zu unterstützen.

(6) Die Versammlungsleitung nimmt während der Versammlung Anträge
entgegen, die sie nach kurzer Prüfung auf Zulässigkeit und Dringlichkeit
der Versammlung angemessen bekannt macht.

(7) Kommt es im Laufe der Versammlung zu einer formalen Verklemmung, ist
die Versammlungsleitung berechtigt, diese per Entscheid aufzulösen.

\subsubsection{§ 7 Wahlleitung}

(1) Die Versammlung wählt zur Durchführung von Wahlen zu Ämtern, die
über das Ende der Versammlung hinaus bestehen, mindestens einen
Wahlleiter. Diese dürfen nicht Kandidat für ein Amt sein, dessen Wahl
sie durchzuführen haben.

(2) Wahlleiter können vom Versammlungsleiter beauftragt werden, ihn bei
der Feststellung weiterer Wahl- oder Abstimmungsergebnisse zu
unterstützen.

(3) Die Durchführung von Wahlen umfasst \\ 1. die Ankündigung der Wahl,
\\ 2. Hinweise auf die Modalitäten der Wahl, \\ 3. die Eröffnung und die
Beendigung der Wahl, \\ 4. das Sicherstellen der Einhaltung der
Wahlordnung und Satzung, insbesondere der geheimen Wahl, \\ 5. das
Entgegennehmen der Stimmergebnisse aus den einzelnen Wahllokalen und
deren Aufsummierung, \\ 6. Feststellung der Anzahl abgegeben, der
gültigen, der ungültigen und der jeweils auf die Kandidaten entfallenen
Stimmen und der daraus resultierenden Wahl, \\ 7. Frage an die gewählten
Kandidaten, ob diese jeweils ihre Ämter antreten und \\ 8. Erstellung
eines Wahlprotokolls.

(4) Die Wahlleiter ernennen Wahlhelfer. Je zwei Wahlhelfer werden zur
Entgegennahme der Stimmzettel einer Wahlurnen zugeordnet. Die Wahlhelfer
beaufsichtigen die Abgabe der Stimmzettel, zählen die Ergebnisse aus und
melden sie dem Wahlleiter. Wahlhelfer dürfen nicht ein Kandidat für ein
Amt sein, dessen Wahl sie durchzuführen haben. Wahlhelfer stehen unter
der Aufsicht des Wahlleiters und können auch von der Versammlung mit
Mehrheit abgelehnt werden. \textbf{\{GO-Antrag auf Ablehnung des
Wahlhelfers XY\}}

(5) Die Wahlleitung fertigt ein Wahlprotokoll über alle Wahlen der
Versammlung an, das vom Wahlleiter und mindestens zwei Wahlhelfern zu
unterschreiben und somit zu beurkunden ist.

\subsubsection{§ 8 Protokollführung}

(1) Die Protokollführung ist verantwortlich für das Erstellen eines
schriftlichen Protokolls der Versammlung.

(2) Das Protokoll der Versammlung enthält mindestens

\begin{itemize}
\item
  jeden Wechsel des Versammlungsleiters,
\item
  gestellte Anträge (nicht GO-Anträge) im Wortlaut,
\item
  Feststellungen der Versammlungsleitung, wie Ergebnisse von
  Abstimmungen und Meinungsbilder,
\item
  Ergebnisse aller Abstimmungen über die Anträge,
\item
  das Wahlprotokoll (falls Wahlen stattfinden).
\end{itemize}
(3) Es wird durch Unterschrift eines Versammlungsleiters, des
Wahlleiters und des am Ende der Versammlung amtierenden Vorsitzenden
oder dessen Stellvertreters beurkundet.

(4) Es ist den Piraten (im Sinne der Satzung) durch Veröffentlichung auf
üblichen Kommunikationswegen unverzüglich zugänglich zu machen.

\subsection{Wahlen}

\subsubsection{§ 9 Kandidaturen}

(1) Der Wahlleiter ruft vor der Wahl zur Kandidatenaufstellung auf und
gibt den Kandidaten Zeit, sich zu melden.

(2) Vor der Schließung der Kandidatenaufstellung ist diese vom
Wahlleiter bekannt zu geben. Daraufhin ist ein letzter Aufruf zu
starten. Meldet sich innerhalb angemessener Zeit kein neuer Kandidat, so
wird die Liste geschlossen.

(3) Wurde die Kandidatenliste geschlossen, so kann sich keiner mehr
aufstellen oder seine Kandidatur zurückziehen.

\subsubsection{§ 10 Wahlen}

(1) Die Wahlen der Vorstandsmitglieder und des Schiedsgerichts sind
geheim. Andere Wahlen finden grundsätzlich offen statt. Auf Verlangen
eines Stimmberechtigten wird eine Wahl geheim durchgeführt.
\textbf{\{GO-Antrag auf geheime Wahl\}}

(2) Kandidieren mehrere Bewerber, so findet eine Akzeptanzwahl statt.
Gewählt ist der Kandidat, welcher die meisten Stimmen und eine absolute
Mehrheit der sich nicht enthaltenden Abstimmenden erhält.

(3) Haben zwei oder mehrere Kandidaten für ein zu besetzendes Amt exakt
die gleiche (höchste) Stimmenanzahl, wird unter diesen Kandidaten ein
weiterer Wahlgang gemäß § 9 \textbf{{[}Kandidaturen{]}} Abs. 2
durchgeführt. Steht danach immer noch kein Sieger fest, wird per Los
entschieden.

(4) Sind mehrere Ämter gleicher Bezeichnung in einem Wahlgang zu wählen
(z.B. Beisitzer oder Kassenprüfer), kann dies in einem Wahlgang oder
getrennt geschehen \textbf{\{GO-Antrag auf getrennte Wahl\}}.

(5) Werden mehrere Ämter gleicher Bezeichnung in einem Wahlgang gewählt,
findet eine Akzeptanzwahl statt. Gewählt sind die Kandidaten in der
Reihenfolge ihrer Stimmenanteile, bis die zu besetzende Zahl der Ämter
erreicht ist. Bei Stimmgleichheit an der Schwelle wird eine Stichwahl
durchgeführt, danach entscheidet das Los. Erreichen in einem Wahlgang
nicht genug Bewerber die erforderliche Mehrheit, findet ein weiterer
Wahlgang statt. Die Versammlung kann beschließen, die Wahlliste wieder
zu öffnen.

(6) Werden getrennte Wahlgänge durchgeführt, bestimmt der Wahlleiter die
Abstimmungsreihenfolge. Die Versammlung kann eine davon abweichende
Reihenfolge bestimmen. \textbf{\{GO-Antrag auf Änderung der Reihenfolge
der Wahlgänge\}}

(7) Gibt es nur einen Kandidaten, so wird mit ``Ja'' oder ``Nein''
abgestimmt. Der Kandidat ist gewählt, falls mehr Ja- als Nein-Stimmen
abgegeben wurden.

(8) Bild- und Tonaufnahmen sind auch während geheimer Stimmabgabe
zulässig.

\subsubsection{§ 11 Wahlen zur Aufstellung einer Bundesliste zu Wahlen}

(1) Ein Kandidat für die Bundesliste gilt als gewählt, sofern er die
mehrheitliche Zustimmung erhält.

(2) Die Rangfolge in der Bundesliste wird durch die Stimmenzahl
festgelegt.

(3) Die Versammlung entscheidet über die maximale Anzahl der Stimmen,
die pro Kandidat vergeben werden kann und über die Gesamtzahl der
maximal zu vergebenden Stimmen pro Wähler, die entweder exakt der Anzahl
der Kandidaten entsprechen muss, oder mindestens dem doppelten.
Entscheidet die Versammlung, dass maximal eine Stimme pro Kandidat
vergeben werden darf oder dass die Maximalzahl der Stimmen gleich der
Anzahl der Kandidaten ist, so muss die Abstimmung über den Listenplatz
der Kandidaten in einem zweiten Wahlgang erfolgen.

(4) Zwischen Kandidaten mit gleicher Stimmenzahl wird in einem weiteren
Wahldurchgang eine Stichwahl durchgeführt. Abweichend von Abs. 3 erhält
jeder Wähler eine Stimme, die er einem der Kandidaten geben kann. Abs. 2
gilt entsprechend.

\subsection{Anträge}

\subsubsection{§ 12 Abstimmungen über Anträge}

(1) Gibt es drei oder mehr Anträge, die sich gegenseitig ausschließen,
so wird mittels Auswahl durch Zustimmung (Akzeptanzverfahren) die Zahl
der Anträge zunächst auf zwei reduziert. Dabei werde alle
konkurrierenden Anträge zur Abstimmung gestellt und nur die Zahl der
Ja-Stimmen für jeden Antrag gezählt, wobei jeder Berechtigte beliebig
vielen Anträgen zustimmen kann. Für die zwei Anträge mit den höchsten
Stimmanteilen gilt dann das Verfahren nach Absatz 2. Bei
Stimmengleichheit an der Schwelle wird unter Ausschluss der sicher
weiterkommenden und sicher auszuschließenden Anträge das Verfahren nach
den Absätzen 1 oder 2 erneut angewandt, bei wiederholter
Stimmengleichheit entscheidet das Los.

(2) Gibt es zwei Anträge, die sich gegenseitig ausschließen, so wird
zuvor in einer Stichwahl ermittelt, welcher Antrag ausscheidet und
welcher einzig zur Abstimmung stehen soll. Ja-Stimmen zählen für den
ersten Antrag, Nein-Stimmen für den zweiten Antrag. Der Antrag mit
weniger Stimmen gilt als abgelehnt und scheidet aus. Bei
Stimmengleichheit wird die Abstimmung wiederholt, bei erneuter
Stimmengleichheit entscheidet das Los. Der erfolgreiche Antrag steht
dann zur Gesamtabstimmung nach Absatz 3.

(3) Steht nur ein Antrag zur Abstimmung oder ist durch die Verfahren
nach den Absätzen 1 und 2 ein Antrag zur Gesamtabstimmung ausgewählt
worden, so wird entsprechend § 3 dieser Geschäftsordnung abgestimmt. Bei
dieser Abstimmung müssen die gegebenenfalls durch diese
Geschäftsordnung, die Satzung oder ein Gesetz geforderten Mehrheiten
erreicht werden.

\subsubsection{§ 13 Allgemeine Anträge an die Versammlung}

(1) Zu Beginn der Beratung eines neuen Antrags hat der Antragsteller
eines jeden aufgerufenen Antrags das Recht, seinen Antrag in kompakter
Rede vorzustellen (Antragsbegründung). Anschließend folgt die
Aussprache. Die Reihenfolge der Wortbeiträge in der Aussprache wird von
der Versammlungsleitung festgelegt.

(2) Redebeiträge können zeitlich begrenzt werden wobei dem Antragsteller
relativ zu einzelnen weiteren Redebeiträgen mehr Zeit einzuräumen ist.

(3) Fragen an einen Redner können im Anschluss an den Wortbeitrag
gestellt werden. Sie müssen deutlich als solche gestellt werden und den
Adressaten enthalten. Auf Fragen kann der Adressat antworten, Fragen
dienen nicht der Erörterung oder der Darstellung der Meinung des
Fragenden.

(4) Zur Einhaltung der Tagesordnung kann die Versammlungsleitung die
Zahl der Fragen begrenzen, die Liste der Wortmeldungen schließen und
Redezeiten begrenzen, nachdem darauf deutlich hingewiesen worden ist.

\subsubsection{§ 14 Anträge auf Änderung der Satzung}

(1) Es gelten die Regelungen aus § 13 \textbf{{[}allgemeine Anträge an
die Versammlung{]}} entsprechend.

(2) Bei Abstimmungen über die Änderung der Satzung ist eine
Zwei-Drittel-Mehrheit (d.h. doppelt so viele Ja wie Nein Stimmen)
erforderlich.

\subsubsection{§ 15 Anträge auf Änderung des Programms}

(1) Es gelten die Regelungen aus § 13 \textbf{{[}allgemeine Anträge an
die Versammlung{]}} entsprechend.

(2) Abgelehnte oder zurückgezogene Programmanträge können auf Wunsch des
Antragstellers sofort als Positionspapier abgestimmt werden.

(3) Bei Abstimmungen über die Änderung des Parteiprogramms ist eine
Zwei-Drittel-Mehrheit (d.h. doppelt so viele Ja wie Nein Stimmen)
erforderlich.

\subsubsection{§ 16 Anträge zur Geschäftsordnung}

(1) Nur die in dem Abschnitt \textbf{Geschäftsordnungsanträge} benannten
Geschäftsordnungsanträge sind als solche zulässig.

(2) Insofern in dieser Geschäftsordnung nicht anders geregelt, kann
jeder Pirat jederzeit einen zulässigen GO-Antrag stellen. Dazu hebt er
beide Hände und begibt sich an das dafür vorgesehene Saalmikrofon. Die
Wortmeldung zu einem GO-Antrag hat Vorrang vor anderen Wortmeldungen.
Sie unterbricht weder einen laufenden Wortbeitrag noch eine eröffnete
Wahl (also ab Beginn der vom Wahlleiters eröffneten Stimmabgabe bis zu
deren Ende) oder Abstimmung.

(3) Versucht ein Teilnehmer, einen nicht zulässigen GO-Antrag oder einen
GO-Antrag in einer nicht zulässigen Form zu stellen, entzieht ihm der
Versammlungsleiter unverzüglich das Wort.

(4) Um Missverständnisse zu vermeiden, sollen komplexere GO-Anträge als
Text beim Versammlungsleiter oder dem von ihm damit beauftragten Piraten
eingereicht werden.

(5) Wurde ein GO-Antrag gestellt, so kann jeder Pirat entsprechend Abs.
2 einen GO-Alternativantrag stellen. \textbf{\{GO-Alternativantrag\}}.
Andere Anträge sind bis zum Beschluss über den Antrag oder dessen
Rückziehung nicht zulässig.

(6) Jeder Pirat kann daraufhin eine Für- oder Gegenrede für einen Antrag
halten. Die Beendigung der Aussprache liegt einzig im Ermessen des
Versammlungsleiters.

(7) Unterbleibt eine Gegenrede und wurde kein Alternativantrag gestellt,
so ist der Antrag angenommen. Gibt es mindestens eine Gegenrede oder
gibt es mindestens einen Alternativantrag, so wird über den Antrag bzw.
die Anträge abgestimmt. Im letzteren Fall gilt § 12
\textbf{{[}Abstimmungen über Anträge{]}} Abs. 2 entsprechend; eine
Gesamtabstimmung entsprechend § 12 \textbf{{[}Abstimmungen über
Anträge{]}} Abs. 3 findet nicht statt.

\subsection{Geschäftsordnungsanträge}

\subsubsection{§ 17 Zulassung des Gastredners}

(1) Jeder Pirat kann das Rederecht für einen Gast beantragen. Der Gast
ist namentlich zu benennen.

\subsubsection{§ 18 Ablehnung eines Wahlhelfers}

(1) Wahlhelfer können von der Versammlung mit Mehrheit abgelehnt werden.
Der Wahlhelfer ist namentlich zu benennen und der Antrag zu begründen.

(2) Dem Wahlhelfer ist das Recht einzuräumen sich angemessen zu
verteidigen.

\subsubsection{§ 19 Geheime Wahl}

(1) Ein GO-Antrag auf geheime Wahl ist ohne Abstimmung angenommen.

\subsubsection{§ 20 Geheime Abstimmung}

(1) Ein GO-Antrag auf geheime Abstimmung ist angenommen, wenn mindestens
10 Piraten zustimmen.

\subsubsection{§ 21 Wiederholung der Wahl/Abstimmung}

(1) Mit einem GO-Antrag auf Wiederholung der Wahl/Abstimmung kann von
mindestens 10 Piraten die Wiederholung der vorangegangen Wahl oder
Abstimmung beantragt werden. Der Antrag ist zu begründen.

\subsubsection{§ 22 Auszählung einer Abstimmung}

(1) Stimmt die Mehrheit für den GO-Antrag auf Auszählung einer
Abstimmung, sollten die Wahlhelfer diese Auszählung unterstützen.

\subsubsection{§ 23 Getrennte Wahlgänge}

(1) Nach einem angenommenen GO-Antrag auf getrennte Wahlgänge legt der
Wahlleiter die Reihenfolge der Wahlgänge fest.

\subsubsection{§ 24 Änderung der Reihenfolge der Wahlgänge}

(1) Finden getrennte Wahlgänge statt, so kann die Versammlung mit einem
GO-Antrag auf Änderung der Reihenfolge der Wahlgänge eine abweichende
Reihenfolge der Wahlgänge bestimmen.

\subsubsection{§ 25 GO-Alternativantrag}

(1) Wurde ein GO-Antrag gestellt, so kann jeder Pirat einen
GO-Alternativantrag stellen. Andere Anträge sind bis zum Beschluss über
den Antrag oder dessen Rückziehung nicht zulässig.

\subsubsection{§ 26 Schließung der Redeliste}

(1) Wurde ein GO-Antrag auf Schließung der Redeliste angenommen, so
müssen sich alle Redner unverzüglich melden.

(2) Der GO-Antrag auf Schließung der Redeliste ist nicht zulässig, wenn
er von einem Piraten gestellt wurde der bereits eine Rede in der
aktuellen Debatte gehalten hat oder selbst in der Redeliste eingereiht
ist.

\subsubsection{§ 27 Wiedereröffnung der Redeliste}

(1) Jeder Pirat kann einen begründeten GO-Antrag auf Wiedereröffnung der
Redeliste stellen, falls die Redeliste geschlossen ist.

(2) Ein GO-Antrag auf Wiedereröffnung der Redeliste wird erst
abgestimmt, sobald alle Redner auf der geschlossenen Redeliste an der
Reihe waren.

(3) Wurde ein GO-Antrag auf Wiedereröffnung der Redeliste angenommen, so
wird die Redeliste für einen kurzen Moment wiedereröffnet. Alle Redner
müssen sich unverzüglich melden. Die Redeliste gilt danach wieder als
geschlossen.

\subsubsection{§ 28 Begrenzung der Redezeit}

(1) Ein GO-Antrag auf Begrenzung der Redezeit muss die gewünschte
maximale Dauer (in Minuten) zukünftiger Redebeiträge enthalten und die
Angabe machen, wie lange diese Beschränkung gelten soll (z.B. bis zur
Beschlussfassung über oder Vertagung des aktuellen Antrages).

(2) Der GO-Antrag auf Begrenzung der Redezeit ist nicht zulässig, wenn
er von einem Piraten gestellt wurde der bereits eine Rede in der
aktuellen Debatte gehalten hat oder selbst in der Redeliste eingereiht
ist.

\subsubsection{§ 29 Einholung eines Meinungsbildes}

(1) Meinungsbilder sind ein Mittel zur Überprüfung der Meinung der
Versammlung zum gerade behandelten Antrag. Meinungsbilder die inhaltlich
keinen erkennbaren Zusammenhang mit dem gerade behandelten Antrag haben,
werden nicht entgegengenommen.

(2) Ein GO-Antrag auf Einholung eines Meinungsbildes gilt ohne
Abstimmung als angenommen.

(3) Ein Meinungsbild wird (auch bei knappem Ergebnis) nicht ausgezählt.

(4) Ein GO-Antrag auf Meinungsbild muss schriftlich beim
Versammlungsleiter oder dem von ihm beauftragten Piraten gestellt
werden.

\subsubsection{§ 30 Unterbrechung der Sitzung}

(1) Ein GO-Antrag auf Unterbrechung der Sitzung kann die Dauer der
Unterbrechung beinhalten. Falls die Dauer nicht bestimmt ist, obliegt es
dem Versammlungsleiter die Dauer zu bestimmen.

\subsubsection{§ 31 Änderung der Tagesordnung}

(1) Eine Änderung der Tagesordnung kann sein \\ 1. das Hinzufügen eines
Punktes, \\ 2. das Entfernen eines Punktes, \\ 3. das Heraustrennen
eines Punktes aus einem anderen Punkt der Tagesordnung, \\ 4. das Ändern
der Reihenfolge von Punkten.

(2) Ein GO-Antrag auf Änderung der Tagesordnung muss schriftlich beim
Versammlungsleiter oder dem von ihm beauftragten Piraten von mindestens
einem akkreditierten Piraten gestellt werden.

(3) Ein GO-Antrag auf Änderung der Tagesordnung muss sämtliche zur
Änderung vorgesehenen Tagesordnungspunkte enthalten. Bei Hinzufügung,
Verschiebung, Heraustrennung und der Änderung der Reihenfolge von
Tagesordnungspunkten müssen eindeutige Angaben enthalten sein, wann die
betreffenden Anträge behandelt werden sollen.

\subsubsection{§ 32 Änderung der Geschäftsordnung}

(1) Ein GO-Antrag auf Änderung der Geschäftsordnung muss schriftlich
beim Versammlungsleiter oder dem von ihm beauftragten Piraten von
mindestens einem akkreditierten Piraten gestellt werden.

(2) Ein GO-Antrag auf Änderung der Geschäftsordnung muss eindeutig
kenntlich machen, was an welcher Stelle in der Geschäftsordnung geändert
werden soll. Ansonsten kann der Antrag aus formalen Gründen abgelehnt
werden.
