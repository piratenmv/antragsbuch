\section{Antrag}

Die Piratenpartei Mecklenburg-Vorpommern setzt sich dafür ein, dass dem ehemaligen Schulfach ``Schulgartenkunde'' wieder mehr Bedeutung im Rahmen der allgemeinbildenden Schulen eingeräumt wird. Es soll versucht werden, wieder flächendeckend einen Schulgarten pro allgemeinbildender Schule zu etablieren.\\Des weiteren sollte ein Budget zur Anschubfinanzierung von geplanten Neuanlage von Schulgärten in Kooperationen mit Eltern und Gartenbauvereinen im Haushaltsplan des Ministeriums für Bildung, Wissenschaft und Kultur Mecklenburg-Vorpommern geschaffen werden.

\section{Begründung}

Der Schulgarten hat eine lange Tradition, die bis in die Antike zurückreicht. Im ökologisch geschützten Raum eines Schulgartens war und ist es möglich, unter Anweisung praktische Erfahrungen im Umgang mit Pflanzen und Lebewesen zu sammeln, die man sonst nur aus Büchern kannte\\Auf diese Weise wurden nicht nur die Grundlagen in gesunder Ernährung, nein auch in Gärtnern ohne Gift, Artenschutz und Tierhaltung gelehrt. Auch wenn dies heute, in Zeiten von Massentierhaltung und Gengemüse aus dem EU-Ausland, nicht mehr die übergeordnete Rolle spielt, erinnern sich noch viele an die Zeit die sie an ihren Beeten verbracht haben und an das gelernte. Diese Erfahrungen werden auch heute noch oft an die eigenen Kinder weitergeben.\\Auf Anfrage an des Ministeriums für Bildung, Wissenschaft und Kultur Mecklenburg-Vorpommern, konnte dieses mir leider nicht mitteilen, wie viele Schulen noch einen Schulgarten besitzen, da dies nicht erfasst wird. Auch gibt es keinerlei finanzielle oder anderweitige Unterstützung. Das Fach ``Schulgarten'' ist auf der Stundentafel in das Fach ``Sachkundeunterricht'' aufgegangen und wird je nach schulinternem Lehrplan weniger oder gar nicht mehr behandelt.\\Auf diese Weise würde den Schülern von Klein auf der Wert von regionalen Erzeugnissen und selbst angebauter Nahrung vermittelt, der Unterschied vom Geschmack von Selbstanbau und Großmarkt gezeigt und ein gesünderer Lebenswandel vermittelt. Durch Kooperationen mit den Eltern und lokalen und regionalen Gartenbauvereinen, würde die regionale Identität der Kinder gestärkt, ein Gefühl des Zusammenhaltens erzeugt und im Endeffekt ein besseres Lernklima geschaffen.

\subsection{Anregungen}

\subsubsection{Soll Schulgarten wieder ein reguläres Schulfach werden?"}

Nein, seine Inhalte sind in andere Fächer aufgegangen. Die genaue Antwort aus dem Bildungsministerium ist hier zu finden. (Antwort aus dem Bildungsministerium zum Thema Schulfach\footnote{\url{http://pastebin.com/SFCEyLUQ}})

\subsubsection{Können wir klären, ob die sachlichen Voraussetzungen noch gegeben sind?}

Nein, da ``Die Anzahl der Schulen mit einem Schulgarten (wird) statistisch nicht erfasst {[}wird{]}.'' Das beutetet, die Verwaltung und das Ministerium für Bildung weiß nicht, wie viele Schulgärten jetzt existieren und ob es dafür noch Flächen gibt oder gäbe. Des weiteren gibt es auch zZ kein Budget dafür.
