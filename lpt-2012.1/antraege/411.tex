\section{Antrag}

\subsection{Bürgermeister per Zustimmungswahl}

Die Piratenpartei Mecklenburg-Vorpommern setzt sich für eine Wahl der Oberbürgermeister und Bürgermeister per Zustimmungswahl ein. Bei dieser einfachen und leicht verständlichen Methode haben die Wähler die Möglichkeit, für beliebig viele Kandidaten zu stimmen. Wählbar sind alle Kandidaten, die die dafür notwendigen Grundvoraussetzungen erfüllen. Gewählt ist der Kandidat mit den meisten Stimmen. Die Vorteile der Zustimmungswahl sind vielfältig. Der beliebteste Kandidat gewinnt die Wahl, und die strukturelle Benachteiligung von kleinen Parteien wird verringert. Konsensfindung und die Diskussionen an Sachthemen werden gefördert und mögliche Verzerrungen des Wählerwillens durch das Stichwahlsystem werden ausgeschlossen.

\section{Begründung}

Der Antrag wurde aus dem Antragsbuch des LPT 2012.1 aus Brandenburg (Antrag WP030\footnote{\url{http://wiki.piratenbrandenburg.de/Antragsfabrik/Bürgermeister\_per\_Zustimmungswahl}}) übergenommen. Dort wurde der Antrag angenommen. Die Grundlage des Antrages entstammt dem Wahlprogramm der Piratenpartei Nordrhein-Westfalen. Der Antrag wurde durch die AG TDBD entsprechend auf das Land Brandenburg angepasst und dann zum Landesparteitag 2012.1 eingebracht.
