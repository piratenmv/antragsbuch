\section{Antrag}

\subsection{CCS-Technologie}

Der Transport von industriell verunreinigtem CO2 sowie dessen Endlagerung im Untergrund oder in Gewässern bergen eine große Anzahl an potenziellen Gefahren sowie ökologischen und finanziellen Nachteilen die bisher noch nicht vollständig zu überblicken sind. Einige dieser Gefahren sind Erdbeben und Erdrutsche, welche für anliegende Städte und Ortschaften Landschafts-, Gebäude-, Straßen- und Personenschäden bedeuten würden. Die Abscheidung, der Transport und die CO2-Endlagerung mindern die Effizienz der fossilen Kraftwerke, wodurch die Stromerzeugung teurer werden würde und sehr viele Steuergelder aufgewendet werden müssten. Aus diesen und weiteren Gründen lehnen wir den Transport von industriell verunreinigtem CO2, sowie dessen Endlagerung im Untergrund oder in Gewässern ab. Eine Abscheidung von CO2 für andere Nutzungsarten wird nicht abgelehnt.

\section{Begründung}

Dieser Antrag wurde aus dem Antragsbuch zum LPT2012.1 in Brandenburg übernommen (WP081). Der Antrag wurde dort angenommen.

\subsection{Begriffsklärung}

CCS (engl. Carbon (Dioxide) Capture and Storage, deut. Kohlenstoff (dioxid)-Abscheidung und Speicherung) bezeichnet die Abscheidung von Kohlenstoffdioxid (CO2) insbesondere aus industriellen Prozessen (z.B. Energiegewinnung aus fossilen Brennstoffen) mit dem Ziel der anschließenden Endlagerung, vorrangig in unterirdischen und unterseeischen Speicherstätten. Durch die Endlagerung soll verhindert werden, dass das CO2 in die Atmosphäre gelangt, wo es als Treibhausgas wirken könnte. Eine Abscheidung von CO2 zur weiteren Nutzung, z.B. für chemische Synthesen, darf nicht als Teil des CCS-Verfahrens verstanden werden. Mit diesem Antrag wird nicht das gesamte Technologiefeld abgelehnt werden. Eine Speicherung von CO2 in Form von Biomasse wird nicht abgelehnt.
