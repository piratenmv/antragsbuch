\section{Antragstext}

Die Piratenpartei fordert die Abschaffung der in § 28 Straßen- und Wegegesetz Mecklenburg-Vorpommern geregeltem Gebühren sowie die Freistellung von kommunalen Verwaltungs- und Bearbeitungsgebühren bei Anträgen für Informationsstände von gemeinnützige Organisation sowie politischer Parteien zur politischen Willensbildung. Ausnahmen können durch erhöhte Nachfrage bei geringem Platzangebot (bspw. im Rahmen von Weihnachtsmärkten) begründet werden.

\section{Begründung}

Nach Artikel 21 Grundgesetz wirken die Parteien bei der politischen Willensbildung des Volkes mit. Informationsstände sind daher gerade für kleine Parteien ein wichtiges Medium um den Bürgern über die Inhalte der Partei zu informieren. Eine Gebühr für solche Stände belastet gerade kleine Parteien finanziell stärker als Parteien mit großem Budget. Auch für gemeinnützige Organisationen sind Gebühren eine große finanzielle Belastung. Dass kostenlose Stände durchaus möglich sind, zeigt bswp. die Hansestadt Rostock.
