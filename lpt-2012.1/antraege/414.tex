\section{Antrag}

\subsection{Lichtverschmutzung reduzieren}

Die Piratenpartei Mecklenburg-Vorpommern wollen die Lichtüberflutung des städtischen und außerstädtischen öffentlichen Raumes im Interesse der Umwelt im Sinne des natürlichen Tages- und Nachtrhythmus von Tier, Mensch und Natur vermindern, ohne die Sicherheit von Wegen zu beeinträchtigen. Für die nächtliche Straßenbeleuchtung sind Lichtquellen mit entsprechend dem Stand der Technik reduzierten UV-Anteil zu bevorzugen, um die Beeinflussung von Insekten und Vögeln zu vermindern.

\section{Begründung}

Der Wegfall einer klaren Tag- und Nachttrennung hat erheblichen Einfluß auf die biologischen Aktivitäten nicht nur des Menschen, sondern auch von Tieren. Darüber hinaus beeinträchtigt die Abstrahlung in den Himmel die Orientierung von fliegenden Tieren.

\section{Anmerkung}

Der Antrag wurde aus dem Antragsbuch des LPT 2012.1 aus Brandenburg (Antrag WP036\footnote{\url{http://wiki.piratenbrandenburg.de/Antragsfabrik/Lichtverschmutzung}}) übergenommen. Dort wurde der Antrag angenommen.
