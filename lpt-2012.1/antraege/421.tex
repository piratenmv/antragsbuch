\section{Antrag}

Die Piratenpartei fordert die Ergänzung der pädagogischen Systembetreuung durch technisch speziell geschultes Personal. Zum einen sollen so Lehrkräfte entlastet werden, zum anderen kann nur so eine angemessene technische Betreuung gewährleistet werden. Lehrer, die die Aufgabe der pädagogischen Systembetreuung übernehmen sind entsprechend zu entlohnen. Die Piratenpartei erachtet es als notwendig an den Schulen in Mecklenburg-Vorpommern auf 90 bis 100 Rechner eine volle Stelle, mindestens aber eine halbe Stelle pro Einrichtung zur Verfügung zu stellen.

\section{Begründung}

a) In Bezug auf LQFB-Initiative 56; Abgeschlossen und angenommen mit 100\% \\b) Begründung aus LQFB-Initiative 56: "Wie bereits in Initiative 30 ausgeführt benötigen unsere Schulen eine angemessene Betreuung der IT. Dies dient zum einem der Korrektur des aktuell desolaten Zustands, zum anderen ist es Voraussetzung zur Umsetzung anderer Zielstellungen, die wir im Bereich Schule und Bildung angedacht haben. Die Wartung und Betreuung bezieht sich sowohl auf die Systeme innerhalb der Klassenräume (oder aber auch Bibliotheken, Einzelplatzrechner, etc.) als auch auf die Schulverwaltung. Als Grundlage dieser Forderung dient eine Empfehlung der Gesellschaft für Informatik aus dem Jahre 2001.\\c) LQFB-Initiative 30: `Landkreiseigene Administratoren für Schulnetzwerke'; Abgeschlossen und angenommen mit 100\%
