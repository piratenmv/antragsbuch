\section{Antrag}

Der Landesverband möge beschließen, das Wort „Landesparteitag`` in der Satzung durch „Landesmitgliederversammlung`` zu ersetzen.

\section{Begründung}

Obwohl „Parteitag`` allgemein ein satzungsmäßiges Treffen einer Partei beschreibt, bemerkt Wikipedia\footnote{\url{http://de.wikipedia.org/wiki/Parteitag}}: \emph{„Auf den meisten Parteitagen sind nicht alle Parteimitglieder, sondern aus organisatorischen Gründen nur eine festgelegte Anzahl von Delegierten anwesend.``} Die Piratenpartei setzt kein Delegiertensystem ein. Dies ist einzigartig in der deutschen Parteienlandschaft. Durch die Bezeichnung „Mitgliederversammlung`` kann diese besondere Natur unterstrichen werden. Die Satzung des Berliner Landesverbandes\footnote{\url{http://wiki.piratenpartei.de/BE:Satzung}} folgt diesem Beispiel.

Natürlich ist ein Name nur ein Name und die Presse mag noch immer von „Landesparteitag`` und „Delegierten`` sprechen, jedoch sollten wir uns der besonderen Situation bewusst sein und uns klar machen, dass es sich um eine Versammlung der Mitglieder handelt.

\section{Anmerkung}

Betroffen wären in der aktuellen Satzung\footnote{\url{http://vorstand.piratenpartei-mv.de/dokumente/satzung/}}:

\begin{itemize}
\item
  § 9, Absatz 1
\item
  § 9a Absatz 3, 6, 9, 10, 11
\item
  § 9b Absatz 1, 2, 3, 4, 5, 6, 7 sowie der Titel
\item
  § 11 Absatz 1, 2, 3
\end{itemize}
