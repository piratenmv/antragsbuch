\section{Antrag}

Der Landesparteitag möge folgende Ergänzung zum aktuellen Wahlprogramm unter Punkt ``Bessere politische und finanzielle Rahmenbedingungen'', Anstrich ``Finanzierung von Bildung und Forschung'' beschließen:

Wir fordern eine angemessene konsumtive und investive aufgabenbezogene Grundfinanzierung für unsere Hochschulen im Land Mecklenburg-Vorpommern, damit die Hochschulen die ihnen zugewiesenen landeshoheitlichen Aufgaben ausreichend wahrnehmen können. Die Grundfinanzierung ist regelmäßig mit steigenden Studierendenzahlen sowie mit zu erwartenden Kostensteigerungen beispielsweise in den Bereichen der Personal- und Energiekosten anzupassen. Genauso ist auf sinkende Bedarfe zu reagieren.

\section{Begründung}

Im Rahmen der Zielvereinbarungen zwischen den Hochschulen und dem Land Mecklenburg-Vorpommern ist die Steigerung der Grundfinanzierung im sogenannten Finanzkorridor der Hochschulen für die Jahre 2011 bis 2015 mit einer jährlichen Steigerung von 1,5\% festgeschrieben worden. Die jährliche Steigerung von 1,5\% ist unzureichend. Die Hochschulen können Faktoren wie Tarifabschlüsse, Preissteigerungen für wissenschaftliche Geräte und Bewirtschaftungskosten kaum beeinflussen. Die Kosten aufgrund dieser Faktoren liegen jedoch regelmäßig über jährlich 1,5\%. Die vereinbarte Steigerung der Grundfinanzierung um jährlich 1,5\% in den letzten Jahren führt praktisch zu einer schleichenden Absenkung der finanziellen Ausstattung der Hochschulen. Im Ergebnis müssen die Hochschulen Personal einsparen und können ihre Aufgaben in der gebotenen Qualität nicht erbringen.

Demgegenüber stehen seit 1992/93 stetig steigende Studierendenzahlen in Mecklenburg-Vorpommern. Im Jahr 2011 wurde erstmals trotz anders lautender Prognose die Marke von über 40.000 Studierenden erreicht. Zusätzlich müssen die Hochschulen trotzdem Personaleinsparvorgaben im Rahmen des Landes-Personalkonzepts 2004 erbringen. Eine Kopplung der jährlichen Steigerung der Grundfinanzierung an zu erwartende Preissteigerungen und Studierendenzahlen kann zu zusätzliche Kosten für den öffentlichen Landeshaushalt führen. Diese Kosten können aus Steuermehreinnahmen gedeckt werden. Steuermehreinnahmen sind grundsätzlich aufgrund eines steigenden Lohn- und Preisniveaus zu erwarten. Aktuell ist im Jahr 2011 ein verbesserter Finanzierungssaldo des Landeshaushalts gegenüber der ursprünglichen Planung von 477 Mio.EUR zu verzeichnen (Pressemeldung FM M-V vom 07.02.2012\footnote{\url{http://www.regierung-mv.de/cms2/Regierungsportal\_prod/Regierungsportal/de/fm/\_Service/Presse/Aktuelle\_Pressemitteilungen/index.jsp?\&pid=33330}}). Weiterhin werden laut Steuerschätzung Steuermehreinnahmen für die Jahre 2012 und 2013 in Höhe von jeweils 40 bis 50 Mio.EUR erwartet (Pressemeldung FM M-V vom 10.05.2012\footnote{\url{http://www.regierung-mv.de/cms2/Regierungsportal\_prod/Regierungsportal/de/fm/\_Service/Presse/Aktuelle\_Pressemitteilungen/index.jsp?\&pid=35101}}). Ein Teil dieser Mehreinnahmen soll dazu aufgewendet werden, die Grundfinanzierung der Hochschulen abzusichern.
