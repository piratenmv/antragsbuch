\section{Antrag}

Die Piratenpartei Mecklenburg-Vorpommern will die Beteiligungsmöglichkeiten der Bürger in den Gemeinden und Landkreisen verbessern. Demokratie findet von unten nach oben statt und beginnt in den Kommunen. Demokratie muss nicht versteckt werden! An der kommunalen Selbstverwaltung kann sich nur derjenige sinnvoll beteiligen, der über alles informiert ist, was in der Verwaltung passiert. Hier wollen wir ansetzen.

Was wollen wir erreichen?

\begin{itemize}
\item
  Wir wollen, dass öffentliche Sitzungen von Kreistagen und Gemeindevertretungen im Internet übertragen und die Aufzeichnungen dort auch später noch abgerufen werden können. Die Landkreise in Mecklenburg-Vorpommern sind inzwischen so groß, dass es nicht mehr jedem möglich ist, an den Sitzungen des Kreistags als Zuhörer teilzunehmen. Außerdem müssen viele Menschen zu den Sitzungszeiten arbeiten oder sich um ihre Familie kümmern. Hier können Übertragungen Abhilfe schaffen.
\item
  Wir wollen, dass die Kreistage, Gemeindevertretungen und Ausschüsse grundsätzlich öffentlich tagen. Die Tagesordnungen sollen mit allen Anträgen und Beschlussvorlagen schon im Vorfeld veröffentlicht werden. Auch alle gefassten Beschlüsse sind umgehend zu veröffentlichen.
\item
  Nicht-öffentliche Sitzungen müssen die absolute Ausnahme bleiben. Wir fordern einen klaren und abschließenden Katalog in der Kommunalverfassung, in welchen Fällen die Öffentlichkeit ausgeschlossen werden darf. Beschlüsse aus nicht-öffentlichen Sitzungen sind nicht geheim, sondern müssen gleichfalls veröffentlicht werden.
\item
  Verträge zwischen der öffentlichen Hand und privaten Dritten müssen öffentlich sein. Wir fordern die Veröffentlichung aller Verträge, die die Gemeinden, Landkreise und kommunalen Verbände mit dritten Personen schließen. Wer sich auf das Betriebsgeheimnis berufen will, soll keine
\item
  öffentlichen Aufträge annehmen. Das gilt auch, wenn die öffentliche Hand durch privatrechtlich organisierte Gesellschaften handelt, die ganz oder teilweise im Eigentum der Kommunen stehen. Personenbezogene Daten sind dabei zu anonymisieren.
\end{itemize}
