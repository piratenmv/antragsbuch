\section{Antrag}

\subsection{Ablehnung von Körperscannern}

Die Piratenpartei Mecklenburg-Vorpommern sieht das an verschiedenen Flughäfen durchgeführte Experiment mit den sogenannten Körperscannern, umgangssprachlich ``Nacktscanner'' genannt, als gescheitert an und fordert einen kompletten Verzicht auf diese überflüssige Technik. Tests haben gezeigt, dass diese Geräte nicht zu mehr Sicherheit führen. Statt zu einer Beschleunigung der Passagierabfertigung beizutragen, wird diese noch massiv verzögert. Auch die existierenden Datenschutzbedenken konnten nicht ausgeräumt werden. Angesichts der Tatsache, dass die Geräte mindestens das zehnfache herkömmlicher Metalldetektoren kosten, gibt es daher keinen einzige vernünftigen Grund, der für den Einsatz dieser Geräte spricht.

\section{Anmerkung}

Der Antrag wurde aus dem Antragsbuch des LPT 2012.1 aus Brandenburg (Antrag WP015\footnote{\url{http://wiki.piratenbrandenburg.de/Antragsfabrik/Ablehnung\_von\_Körperscannern}}) übergenommen. Dort wurde der Antrag angenommen. Die Grundlage des Antrages entstammt dem Wahlprogramm der Piratenpartei Saarland. Der Antrag wurde durch die AG TDBD entsprechend auf das Land Brandenburg angepasst und dann zum Landesparteitag 2012.1 eingebracht.
