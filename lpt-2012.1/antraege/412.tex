\section{Antrag}

\subsection{Depublizieren abschaffen}

Die Piratenpartei Mecklenburg-Vorpommern spricht sich gegen das sogenannte ``Depublizieren'' von Internetinhalten der öffentlich- rechtlichen Rundfunkanstalten aus. Unter Verwendung von Gebührengeldern produzierte Inhalte müssen den Gebührenzahlern zeitlich unbegrenzt im Internet zur Verfügung gestellt werden können. Die Piratenpartei Mecklenburg-Vorpommern setzt sich daher dafür ein, dass der entsprechende Passus aus dem Rundfunkstaatsvertrag wieder gestrichen wird und wird keinem Rundfunkänderungsstaatsvertrag zustimmen, in dem dieser Passus enthalten ist.

\section{Begründung}

Dieser Antrag wurde aus dem Antragsbuch des LPT2012.1 aus Brandenburg übernommen (WP017). Der Antrag wurde dort angenommen.
