\section{Antrag}

Die Piratenpartei Mecklenburg-Vorpommern unterstützt die Volksinitiative »Für den Erhalt einer bürgernahen Gerichtsstruktur in Mecklenburg-Vorpommern«.

\section{Begründung}

Das Justizministerium plant die Schließung und Zusammenlegung von mehreren Amtsgerichten, Arbeitsgerichten und Verwaltungsgerichten in Mecklenburg-Vorpommern. Wir meinen, dass die Antwort auf den demographischen Wandel nicht darin bestehen kann, dass sich der Staat aus der Fläche zurückzieht. Rechtsschutz bedeutet auch, dass der Bürger Gerichte in einer zumutbaren Nähe findet und dort auf Richter trifft, die mit den Verhältnissen in ihrem Gerichtsbezirk durch eigene Anschauung vertraut sind. Gerade Amtsgerichte nehmen wichtige gesellschaftliche Aufgaben wahr (zum Beispiel Familiensachen, Grundbuchamt, Nachlassgericht, Betreuungssachen, Vereinsregister), die eine örtliche Nähe zu den Betroffenen erfordern. Die gewachsenen Strukturen sollten deshalb nicht vermeintlichen Einsparungen geopfert werden. Auch kleine Gerichte können mit intelligenten Strukturen effektiv arbeiten.
