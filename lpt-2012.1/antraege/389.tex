\section{Antrag}

Die Piratenpartei Mecklenburg-Vorpommern setzt sich dafür ein, eine Fortbildungsverpflichtung und einen Fortbildungsnachweis von professionell beruflich Pflegenden einzuführen, um den modernen pflegerischen und medizinischen Anforderungen gerecht zu werden. Die Form des Nachweises kann dabei in einem Punktesystem ähnlich dem Modellprojekt „Registrierung beruflich Pflegender`` erfolgen. Die Fortbildungen sind so zu etablieren, dass es eine Freistellungs- und Finanzierungspflicht seitens der Arbeitgeber bis zum Erreichen der Mindestfortbildungspunkte/-zeit gibt. Die Fortbildung kann auch im Rahmen zertifizierter innerbetrieblicher Veranstaltungen erfolgen.

\section{Begründung}

Die professionelle berufliche Pflege in Form der Gesundheits- und Krankenpflege, als auch der Altenpflege ist einem enormen Arbeits- als auch Wissensdruck ausgesetzt. Die ständig zunehmenden wissenschaftlichen Erkenntnisse im Bereich der Medizin und der Pflege sowie die wirtschaftlichen und sozialen Herausforderungen machen eine ständige und fundierte Fortbildung unausweichlich.\\Diese Fortbildung ist im Interesse der Qualitätsweiterentwicklung der Patientenversorgung nur durch geregelte Fortbildungen zu bewältigen. Fortbildungen werden von vielen Pflegenden nur sehr unregelmäßig oder in einem geringen Umfang wahrgenommen. Dies ist unter anderem deswegen der Fall, weil innerbetrieblichen Angebote fehlen, nicht sehr umfangreich sind oder die Freistellung für externe Fortbildungen nur durch die Investition von Freizeit zu erlangen sind. Die beruflich Pflegenden sind ihrem Berufsstand nach grundsätzlich dazu angehalten, sich in regelmäßigen Abständen fortzubilden. Anders als in anderen staatlich anerkannten Heilberufen, gibt es weder eine Fortbildungspflicht, noch einen Fortbildungskatalog. Als Positivbeispiele ist hierbei die Ärzteschaft zu nennen, die in einem klaren Regelwerk Fortbildungspunkte nachweisen muss oder die staatlich examinierten Rettungsassistenten, die im Rahmen ihres Ausbildungsgesetzes eine jährliche Fortbildungspflicht von 30 Stunden zu absolvieren haben.\\Fehlende Fortbildungsmöglichkeiten und Verpflichtungen haben bereits dazu geführt, dass sich einzelne Pflegekräfte und Pflegewissenschaftler im Rahmen der Initiative „Freiwillige Registrierung beruflich Pflegender`` (Freiwillige Registrierung beruflich Pflegender\footnote{\url{http://www.regbp.de/}}) einer Selbstverpflichtung zur Fortbildung unterworfen haben. Hierbei müssen 40 Fortbildungspunkte in zwei Jahren nachgewiesen werden. Um dem zunehmenden Pflegekraftfachmangel entgegen zu wirken, muss der Wert dieser Berufsbilder erhalten werden und den beruflich Pflegenden gesetzlich die Möglichkeit gegeben werden, sich fortbilden zu können und im Sinne der Patienten auch zu müssen.

\section{Hinweis}

Im Original ist dieser Antrag von Thomas Weijers für den AK Gesundheit als Antrag WP027 zu finden. Dieser Antrag ist für mich ein Anliegen, da man gerade in einer überalternden Gesellschaft in der immer mehr Bürger lange Zeit im Krankenhaus verbringen oder zu Hause gepflegt werden, man sich auf eine gut und aktuell ausgebildeten Pflege verlassen muss. Dies ist wie oben erwähnt bei Ärzten und Rettungsassistenten schon lange umgesetzt.
