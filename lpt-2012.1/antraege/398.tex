\section{Antrag}

§ 7 der Satzung wird wie folgt gefasst:

(1) Der Landesverband gliedert sich in Kreisverbände und Ortsverbände. Kreisverbände können sich über das Gebiet mehrerer aneinander angrenzender Kreise und kreisfreier Städte erstrecken, Ortsverbände über das Gebiet mehrerer aneinander angrenzender Gemeinden.

(2) Auf Verlangen von mindestens drei gründungswilligen Piraten lädt der Landesvorstand alle Piraten mit angezeigtem Wohnsitz im Gebiet des künftigen Kreisverbands zu einer Gründungsversammlung ein. Ort und Zeit der Gründungsversammlung werden von den gründungswilligen Piraten bestimmt, wobei die Ladungsfrist mindestens vier Wochen beträgt. Die Gründungsversammlung ist beschlussfähig, wenn mindestens zehn stimmberechtigte Piraten erschienen sind. Der Kreisverband ist errichtet, wenn auf der Gründungsversammlung dessen Satzung beschlossen worden ist. Für den Beschluss ist eine Mehrheit von 2/3 der abgegebenen Stimmen erforderlich, Enthaltungen gelten dabei als Ablehnung. Über die Versammlung ist ein Protokoll anzufertigen und zu veröffentlichen.

(3) Für die Gründung von Ortsverbänden gilt Absatz 2 entsprechend, solange der zuständige Kreisverband keine andere Regelung trifft.

\section{Begründung}

Angesichts verschiedener Überlegungen zur Gründung von Kreisverbänden ist die Frage aufgekommen, nach welchem Verfahren die Gründung überhaupt erfolgt und ob sogar der Landesparteitag zustimmen muss.

Der Vorschlag soll diese Unsicherheiten abbauen und Mindeststandards bestimmen:

\begin{itemize}
\item
  Drei-Piraten-Prinzip für die Gründungsinitiative
\item
  alle Piraten aus dem künftigen Verband werden zur Gründung geladen, damit die gesamte Basis abstimmen kann = keine Gründung im kleinen Kreis
\item
  mindestens zehn interessierte Piraten bei der Gründung notwendig, damit der Verband eine notwendige Basis hat und arbeitsfähig ist
\item
  Kreisverbände über mehrere Kreise hinweg heißen trotzdem Kreisverband
\end{itemize}
