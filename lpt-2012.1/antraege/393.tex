\section{Antrag}

Die Piratenpartei Mecklenburg-Vorpommern setzt sich dafür ein, dass der Artikel 14 Absatz 6 des Gesetz über den öffentlichen Gesundheits- und Veterinärdienst, die Ernährung und den Verbraucherschutz sowie die Lebensmittelüberwachung (Gesundheitsdienst- und Verbraucherschutzgesetz - GDVG) (Artikel 14 GdVG BY\footnote{\url{http://by.juris.de/by/GesDVerbrSchG\_BY\_Art14.htm}}) sinngemäß in das Gesetz über den Öffentlichen Gesundheitsdienst im Land Mecklenburg-Vorpommern (Gesetz über den Öffentlichen Gesundheitsdienst - ÖGDG M-V) (ÖGDG MV\footnote{\url{http://mv.juris.de/mv/gesamt/OeGDG\_MV.htm}}) übernommen wird.

\section{Begründung}

tl;dr: Änderungen im BKiSchG gut, aber wir wollen mehr a) Meldepflicht, um Ärzte gar nicht die Entscheidung zu überlassen, ob sie gewichtige Anhaltspunkte melden oder nicht, b) Meldestrukturen vereinheitlichen, kürzen und Kompetenzen schaffen\\Das am 01.01.2012 in Kraft getretene Bundeskinderschutzgesetz (BKiSchG) (BKiSchG\footnote{\url{http://www.bundesaerztekammer.de/downloads/BKiSchG\_22.12.2011\_final.pdf}}) ist für Kinder und Jugendliche, aber auch Ärzte ein Schritt in die richtige Richtung. Die wichtigsten Neuregelungen für Ärzte durch das Gesetz sind die Schaffung einer bundeseinheitlichen Befugnisnorm zur Einschaltung des Jugendamtes bei Verdacht auf Vernachlässigung oder Misshandlung eines Kindes (§ 4 KKG). Damit wird für Ärzte eine größere Rechtssicherheit im Umgang mit der ärztlichen Schweigepflicht gemäß § 203 StGB einerseits und der Einschaltung Dritter auf der Grundlage eines rechtfertigen Notstandes nach § 34 StGB geschaffen.\\Denn die Schweigepflicht ist eine essentielle Voraussetzung für den Arztberuf. Eine gesetzliche Grundlage stellt dafür die ärztliche Berufsordnung dar. Danach dürfen Ärzte ein ihnen anvertrautes Geheimnis nicht unbefugt offenbaren. In der Strafprozessordnung (StPO) und der Zivilprozessordnung (ZPO) ist zusätzlich festgelegt, dass diese Geheimnisse weder Gerichten noch der Polizei mitzuteilen sind. Die Schweigepflicht ist auch bei Minderjährigen einzuhalten, d.h. deren Erziehungsberechtige sind nicht zu informieren. § 203 Absatz 1 Nummer 1 Strafgesetzbuch (StGB) stellt denjenigen unter Strafe, der unbefugt ein fremdes Geheimnis, namentlich ein zum persönlichen Lebensbereich gehörendes Geheimnis offenbart, das ihm als Arzt oder Zahnarzt anvertraut worden oder sonst bekanntgeworden ist.\\Der Arzt darf aber die Schweigepflicht unter folgenden Bedingungen brechen, ohne sich strafbar zu machen: a) Bestehen eines rechtfertigenden Notstandes (§ 34 StGB), b) mutmaßliche Einwilligung des Patienten. Nun ist auch der Bruch zum Schutz von Kindern und Jugendlichen besser geregelt und bietet Rechtssicherheit für Ärzte.\\Das neue Gesetzt regelt aber nicht die Meldepflicht. In mehreren Bundesländern existiert diese, was dem Schutz von Kindern und Jugendlichen dient. So ist in Bayern nach Artikel 14 Absatz 6 des GDVG bei gewichtigen Anhaltspunkten für eine Misshandlung, Vernachlässigung oder einen sexuellen Missbrauch eine unverzügliche Meldung an das Jugendamt vorzunehmen. Dieses Gesetz regelt eine Ausnahme von § 203 StGB, in diesem Fall wird das Arztgeheimnis nicht unbefugt, sondern berechtigt verletzt. Dies bedeutet aber nicht, dass vom Arzt eine Strafanzeige gestellt werden muss, da Kindesmisshandlung nicht zu den Straftaten gehört, die nach § 138 StGB zwingend anzuzeigen sind.\\Das BKiSchG erläutert nur den Weg, der gegangen werden kann. Zu Beginn der Kette steht ein Gespräch mit dem Kind oder Jugendlichen und dem Personensorgeberechtigten. Danach besteht ein Anspruch auf Beratung durch eine Fachkraft des Jugendamtes und die Erlaubnis zur pseudonymisierten Datenübermittlung. Zuletzt existiert die Befugnis zur Einschaltung des Jugendamtes unter Mitteilung der erforderlichen Daten.\\Die Übernahme des Absatzes 6 aus Artikel 14 der Bayrischen GDVG würde die Situation in Mecklenburg-Vorpommern für Kinder und Jugendliche deswegen in zwei entscheidenden Punkten verbessern:\\a) Die klare Meldepflicht des Arztes gegenüber dem Jugendamt hätte den Vorteil, dass Ärzte gar nicht mehr überlegen müssen, ob sie die ``Kette'' starten wollen. Denn im die neugeregelten Abläufe im KBiSchG sind nicht verpflichtend.\\b) Dazu kommt, dass verschiedene Meldekreise verkürzt würden. Die bis jetzt übliche Herangehensweise in Mecklenburg-Vorpommern (Leitfaden MV\footnote{\url{http://www.gewalt-gegen-kinder-mv.de/images/stories/tk-leitfaden\_gewalt-gegen-kinder.pdf}}), das Familiengericht, das Jugendamt oder die Polizei einzuschalten, würde insoweit eingeschränkt, dass vom Arzt nur noch das Jugendamt einzubeziehen ist. So kann weniger liegenbleiben, es gibt weniger Kompetenzgerangel und eine Institution mit Kompetenz kann lenken, leiten und im Rahmen des Kindeswohls entscheiden.

\section{Hinweis}

Dieser Antrag entstand durch Anregung und Mitarbeit von Stefan im Antrag ``U- und J-Untersuchungen für Kinder und Jugendliche'' und mit Texten der Webseite der Bundesärztekammer und aus Wolfgang Keils ``BASIC Rechtsmedizin''. Ich finde diesen Antrag sehr wichtig, da er durch die klare Regelung der Meldepflicht und die Verkürzung der Wege das Leid von Kindern und Jugendlichen schneller und effektiver zu beenden. Und das sind Aufaddiert bei Kindesvernachlässigung, Kindesmisshandlung und sexuellen Missbrauch in MV immerhin 389 Kinder im Jahr 2006. (Leitfaden MV\footnote{\url{http://www.gewalt-gegen-kinder-mv.de/images/stories/tk-leitfaden\_gewalt-gegen-kinder.pdf}})
