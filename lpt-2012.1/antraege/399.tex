\section{Antrag}

Es wird folgender § 2 Absatz 3 in die Satzung eingefügt:

\begin{quote}
Untergliederungen können ein eigenes Piratenverzeichnis führen.

\end{quote}
§ 5 Absatz 1 der Satzung wird wie folgt gefasst:

\begin{quote}
Die Beendigung der Mitgliedschaft ist der niedrigsten Gliederung anzuzeigen, die ein Piratenverzeichnis führt.

\end{quote}
\section{Begründung}

Die Aufnahme von Piraten ist bereits jetzt nach § 3 Absatz 1 Satz 3 Nummer 1 der Bundessatzung Aufgabe der niedrigsten Gliederung. Untergliederungen sollten deshalb die Option (nicht die Pflicht!) bekommen, auf Wunsch die Mitgliederverwaltung für ihren Bereich zu übernehmen. Durch § 3 Absatz 2 der Landessatzung ist sichergestellt, dass in diesem Fall Änderungen der Mitgliedsdaten auch dem Landesverband zugänglich gemacht werden. Die Neufassung von § 5 Absatz 1 gibt den Untergliederungen, die ein Piratenverzeichnis führen und die Mitgliederverwaltung übernommen haben, folgerichtig die Zuständigkeit auch für den Empfang von Austrittserklärungen, Adressveränderungen und anderen Umständen, die zum Ende der Mitgliedschaft in der Untergliederung führen.
