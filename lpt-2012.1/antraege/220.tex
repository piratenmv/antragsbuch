\section{Antrag}

§ 9a Absatz 3 Landessatzung wird wie folgt geändert:

\begin{quote}
Der Landesvorstand wird mindestens in jedem zweiten Kalenderjahr gewählt. Die Amtszeit endet mit der Wahl eines neuen Landesvorstands. Die Wiederwahl ist zulässig.

\end{quote}
\section{Begründung}

Die bisherige Regelung schreibt vor, den Landesvorstand jedes Kalenderjahr zu wählen. Damit wäre eine fast zweijährige Amtszeit theoretisch schon jetzt möglich. Dieser Satz schafft Klarheit. Die Formulierung des ersten Satzes ist direkt aus dem Parteiengesetz genommen, die beiden anderen Sätze stammen aus der Landessatzung.

Eine längere Amtszeit bedeutet mehr Kontinuität und spart der Partei den Aufwand, jedes Jahr einen Wahlparteitag zu machen.
