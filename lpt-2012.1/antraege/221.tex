\section{Antrag}

Der Landesparteitag möge beschließen, an bereiter Stelle den folgenden Text in die Satzung aufzunehmen:

\subsection{Liquid Democracy}

(1) Die Piratenpartei Deutschland, Landesverband Mecklenburg-Vorpommern nutzt zur Willensbildung über das Internet eine geeignete Software. Diese muss die ``Anforderungen für den Liquid Democracy Systembetrieb'' erfüllen, welche vom Vorstand beschlossen werden.\\Die Mindestanforderungen sind:\\a) Jedes Mitglied muss die Möglichkeit haben, Anträge im System zu stellen. Zulassungsquoren und Antragskontingente sind zulässig, müssen jedoch für alle Mitglieder gleich sein.\\b) Das System muss ohne Moderatoren auskommen.\\c) In das System eingebrachte Anträge dürfen nicht gegen den Willen des Antragsstellers von anderen Mitgliedern verändert oder gelöscht werden können.\\d) Jedem Mitglied muss es innerhalb eines bestimmten Zeitraums möglich sein, Alternativanträge einzubringen.\\e) Das eingesetzte Abstimmungsverfahren darf Anträge, zu denen es ähnliche Alternativanträge gibt, nicht prinzipbedingt bevorzugen oder benachteiligen. Mitgliedern muss es möglich sein, mehreren konkurrierenden Anträgen gleichzeitig zuzustimmen. Der Einsatz eines Präferenzwahlverfahrens ist hierbei zulässig.\\f) Es muss möglich sein, die eigene Stimme mindestens themenbereichsbezogen durch Delegation an ein anderes Mitglied zu übertragen. Diese Delegationen müssen jederzeit widerrufbar sein und übertragenes Stimmgewicht muss weiter übertragen werden können. Selbstgenutztes Stimmgewicht darf nicht weiter übertragen werden.

(2) Der Vorstand stellt den dauerhaften und ordnungsgemäßen Betrieb des Systems sicher.

(3) Jedem Mitglied ist Einsicht in den abstimmungsrelevanten Datenbestand des Systems zu gewähren. Während einer Abstimmung darf der Zugriff auf die jeweiligen Abstimmdaten anderer Mitglieder vorübergehend gesperrt werden.

(4) Die Organe sind gehalten, das Liquid Democracy System zur Einholung von Meinungsbildern zur Grundlage ihrer Beschlüsse zu nutzen. Das Schiedsgericht ist davon ausgenommen.

(5) Die Organe der Partei sind angehalten, die Anträge, die im Liquid Democracy System positiv beschieden wurden, vorrangig zu behandeln.

(6) Teilnahmeberechtigt ist jeder Pirat, der nach der Satzung stimmberechtigt ist. Jeder Pirat erhält genau einen persönlichen Zugang, der nur von ihm genutzt werden darf.

\section{Begründung:}

Der Antrag ist aus der Satzung der Piratenpartei Deutschland Berlin übernommen und im Punkt (1) (2) und (8) geändert und angepasst.\\Liquid Democracy ist ein wesentliches Prinzip piratiger Politik und sollte sich auch konkret in der politischen Arbeit widerspiegeln.
