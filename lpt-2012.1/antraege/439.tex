\section{Antrag}

Der Landesparteitag empfiehlt, unabhängig vom Recht, der Kandidatur auf den Mitgliederversammlungen zur Aufstellung der Kandidaten, alle die sich auf Listenplätze und Direktkandidatur bewerben, sich an folgenden Prozedere zu orientieren:

\begin{itemize}
\item
  Ihre Kandidatur bis Ende August 2012 bekannt zu geben.
\end{itemize}
\begin{itemize}
\item
  In den Monaten September bis November 2012 Vorstellungen auf den Versammlungen, Stammtischen der jeweiligen Gliederungen und Gruppen
\end{itemize}
\begin{itemize}
\item
  Bis Ende Januar 2013 Mitgliederversammlungen zur Aufstellung der Direktkandidaten
\end{itemize}
\begin{itemize}
\item
  Im Februar/ März 2013 der Landesparteitag zur Aufstellung der Landesliste.
\end{itemize}
Der Landesparteitag beschließt diesen Zeitplan.

\section{Begründung}

Im Antrag \#377 wird das Ziel 31.10.2012 genannt, mit der Begründung des Wahlkampfes und des kennen Lernens der Kandidaten. Ich würde es besser finden, das man die Kandidaten vor den Wahlen zu den Listen kennen lernt.

Der Wahlkampf wird sowieso erst in den Wochen vor der Bundestagswahl personalisiert geführt. So dass auch für die Kandidaten und Mitglieder noch ausreichend Zeit bleibt.
