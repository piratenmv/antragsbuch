\section{Antrag}

Alle Staatsgewalt geht vom Volke aus (Artikel 3 Absatz 1 Satz 1 der Landesverfassung). Die Piratenpartei Mecklenburg-Vorpommern fordert deshalb ein Landtagswahlrecht, welches den Willen des Souveräns bestmöglich abbildet. Die Piraten wollen mehr Demokratie, nicht weniger. Wir fordern:

1. Die Wahlperiode des Landtags wird wieder auf vier Jahre verkürzt.\\2. Die auf die Zweitstimmen entfallenden Mandate werden durch eine offene Listenwahl bestimmt. Die Wähler erhalten wie bei der Kommunalwahl die Möglichkeit, eine bestimmte Person aus der Liste zu wählen.\\3. Die Sperrklausel (Fünf-Prozent-Hürde) wird abgeschafft.\\4. Das aktive Wahlrecht besteht ab dem vollendeten 16. Lebensjahr.

\section{Begründung}

1. Wahlen müssen periodisch stattfinden. Die Legitimation der Volksvertretung muss regelmäßig erneuert werden. Die Einflussmöglichkeit des Souveräns verringert sich, wenn Legislaturperioden vergrößert werden. Auch bei Wahlperioden von vier Jahren, wie sie bis 2006 in Mecklenburg-Vorpommern vorgeschrieben waren und in Hamburg und Bremen noch sind, bleibt ein effektives Arbeiten von Landtag und Landesregierung möglich.

2. Nach dem jetzigen Wahlsystem hat der Wähler keinen Einfluss auf die personelle Zusammensetzung des Parlaments, soweit die Landtagsmandate über die Zweitstimme vergeben werden. Das betrifft die Hälfte der Abgeordneten. Hier bestimmen ausschließlich die Parteien selbst über die Vergabe der Mandate, indem sie die Landeslisten aufstellen. Wir wollen dagegen offene Listen, damit der Wähler durch die Wahl bestimmter Personen auf einer Liste die Reihenfolge der gewählten Abgeordneten beeinflussen können.

3. Die Sperrklausel ist ein erheblicher Eingriff in die Gleichheit der Wahl. Sie führt dazu, dass alle (Zweit-)Stimmen, die auf eine Partei mit weniger als fünf Prozent der Zweitstimmen entfallen, wertlos werden.

Wir halten diese Sperrklausel für überflüssig und demokratiefeindlich. Im Europa- und Kommunalwahlrecht ist sie bereits abgeschafft. Durch die Begrenzung der Abgeordnetensitze auf 71 besteht schon jetzt eine mathematische Sperrklausel, die Parteien mit Stimmanteilen im Promillebereich von der Sitzvergabe ausschließt.

Es trifft nicht zu, dass ohne eine Sperrklausel eine Regierungsbildung erschwert würde. Sämtliche Landesregierungen seit 1994 stützen sich auf eine absolute Mehrheit der Zweitstimmen im Parlament. Im Gegenteil erlaubt die Fünf-Prozent-Hürde parlamentarische Mehrheiten, die sich nicht auf eine Mehrheit der Zweitstimmen stützen. Das halten wir für viel bedenklicher. Ein geordneter Parlamentsbetrieb muss durch das Parlamentsrecht gesichert werden, nicht durch das Wahlrecht.

Die Sperrklausel führt zu taktischem Wählen. Eine maßgebliche Zahl von Wählern wird davon abgehalten, ihre präferierte Partei zu wählen, weil sie ihre Stimme nicht verschenken wollen.

Auch eine Fünf-Prozent-Hürde konnte nicht verhindern, dass eine rechtsextreme und demokratiefeindliche Partei in den Landtag eingezogen ist. Wir meinen, dass verfassungsfeindliche Parteien verboten werden müssen und das Wahlrecht im Übrigen kleine Parteien nicht diskriminieren darf.

4. Jugendliche sind als Schüler und Auszubildende in mehrerer Hinsicht von landespolitischen Entscheidungen betroffen. Wir wollen deshalb, dass sie auf diese Entscheidungen auch über Wahlen Einfluss nehmen können. Es besteht kein Grund, Jugendliche ab dem 16. Lebensjahr zur Kommunalwahl zuzulassen, nicht aber zur Landtagswahl.
