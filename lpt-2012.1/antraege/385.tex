\section{Antrag}

Die Piratenpartei Mecklenburg-Vorpommern setzt sich dafür ein, dass in Mecklenburg-Vorpommern bei Eingang eines Notrufs, dieser automatisch, ohne menschliches Zutun, geortet wird, so dass den Rettungskräften in kürzester Zeit eine sehr genaue Ortsangabe zur Verfügung steht.\\Sobald die Peilung gestartet wird, soll eine nicht abstellbare Hinweis-SMS über Grund und Dauer der Ortung an den Georteten versendet werden. Die angepeilte Position wird mit allen anderen erhobenen Daten an die im Einsatz befindlichen Kräfte weitergegeben.\\Die Peilung, Verarbeitung und Speicherung der Daten muss zu jeder Zeit dem Datenschutz genüge tun und sollte von diesem von der Planung an bis zum laufenden Einsatz überprüft werden.

\section{Begründung}

Operatoren in den Telefonzentralen einer Einsatzleitung, die die berühmten W-Fragen stellen, müssen viel Fingerspitzengefühl, Können und Erfahrung besitzen, um aus einer absolut unter Stress stehenden Person alle in dieser Situation wichtigen Informationen zu bekommen. Eine dieser Fragen könnte man mit dieser Forderung vereinfachen, wenn nicht sogar abschaffen, was Zeit für die Anderen schaffen würden: Die Frage nach dem Ort.\\In unserer heutigen Zeit, wo fast jeder ein Handy besitzt, Funkzellenortungen bei einem einmal eingerichteten Prozess fast keine Zeit mehr benötigen und auf 550m, wenn nicht sogar bis systembedingt (Timing Advance I\footnote{\url{http://de.wikipedia.org/wiki/Timing\_Advance}}, Timing Advance II\footnote{\url{http://winfwiki.wi-fom.de/index.php/Verfahren\_der\_Geo-Lokalisierung\_f\%C3\%BCr\_Location\_Based\_E-Business\_im\_Vergleich\#Timing\_Advance}}) auf minimal 227m präzise sind, stellt sich die Frage, warum man diese Fähigkeit nicht für den Rettungsdienst nutzen sollte. Man bedenke, wer weiß bei Überland oder Autofahrten genau, wo er sich zu einem Zeitpunkt befindet?\\Würde man jeden Anruf in der Leitstelle automatisch orten, hätte man oft sogar vor dem Ende des Gesprächs eine meist genauere Ortsangabe, als die meisten von einer Unfallstelle angeben könnten. So kann man die Einsatzkräfte nicht nur gezielter an den Ort des Unglücks leiten, sonder früh spezielle örtliche Gegebenheiten eruieren und weitergeben, was vor allem bei Helikoptereinsätzen wichtig sein kann.\\Wichtigster Punkt bei der Planung, muss aber der Datenschutz sein. Es darf nicht Möglich sein, willkürlich Personen zu orten. Des Weiteren muss die so erzeugte Ortung sicher der Meldung zugeordnet werden, so das es zu keiner Verwechslung kommen kann. Der Landesdatenschutz sollte die ganze Entwicklung und den Betrieb begleiten.\\Dieser Antrag würde bei einem Notruf nicht nur Zeit sparen, es würde auch helfen Fehler zu verringern, Prozesse effektiver zu gestalten, und am Ende, durch erfolgreichere Einsätze, Leben retten.

\section{Hinweis}

Dieser Antrag wurde beim Landesdatenschutzbeauftragten eingereicht und in seiner jetzigen Form als interessant, seit Jahren in den Grundprinzipien diskutiert und machbar eingestuft worden.\\Zitat Antwortmail: ``Die datenschutzrechtliche Beurteilung des Verfahrens bei den Rettungsleitstellen, einschließlich der technisch-organisatorischen Maßnahmen, würde dann dem Landesbeauftragten für den Datenschutz obliegen.''\\Anfrage zur Genauigkeit ergab (mit etwas Mühe und einigen Tagen Telefonaten und E-Mails), das ein großer, flächendeckender Anbieter (Name mir bekannt) in M-V das T-A-Verfahren automatisch nutzt.
