\subsection{Allgemeiner Teil.}

\subsubsection{§ 1}

(1) Diese Wahl- und Abstimmungsordnung gilt für alle Versammlungen der
Piraten in Mecklenburg-Vorpommern. Sie gilt, vorbehaltlich besonderer
Bestimmungen der Wahlgesetze, auch für Versammlungen zur Aufstellung von
Kandidaten.

(2) Im Sinne dieser Ordnung ist:

\begin{enumerate}
\item
  Wahl: eine Entscheidung über Personalfragen;
\item
  Abstimmung: eine Entscheidung über Sachfragen;
\item
  Pirat: wer nach § 1 Absatz 5 der Satzung des Landesverbandes Pirat
  ist;
\item
  einfache Mehrheit: die Mehrheit der anwesenden Stimmen;
\item
  geheim: eine Wahl oder Abstimmung, bei der die Stimmen der
  stimmberechtigten Piraten diesen nicht zugeordnet werden können;
\item
  öffentlich: eine Wahl oder Abstimmung, wenn sie für jedermann
  zugänglich ist.
\end{enumerate}
\subsubsection{§ 2}

(1) Wahlen und Abstimmungen sind frei, gleich und allgemein und finden
öffentlich statt.

(2) Wahl- und abstimmungsberechtigt ist, wer dem beschließenden Gremium
beim Zusammentritt als stimmberechtigtes Mitglied angehören kann.

(3) Piraten können sich bei einer nicht geheimen Wahl oder Abstimmung
vertreten lassen. Sie benötigen dazu die Vollmacht des betreffenden
stimmberechtigten Piraten, die zu Beginn der Sitzung dem Wahl- oder
Abstimmungsleiter vorgelegt werden muss.

\subsubsection{§ 3}

(1) Werden Stimmzettel verwendet, müssen sie einheitlich sein. Bei
geheimen Wahlen und Abstimmungen müssen Stimmzettel verwendet werden.

(2) Ungültig sind Stimmzettel, die den Willen des wählenden Piraten
nicht zweifelsfrei erkennen lassen.

\subsubsection{§ 4}

(1) Wählen oder abstimmen können auch aus wichtigem Grunde abwesende,
stimmberechtigte Piraten, wenn die Wahl oder Abstimmung in der
Tagesordnung festgelegt ist, sie rechtzeitig bekannt gemacht wurde und
der Vorstand dies zugelassen hat.

(2) Zulässig sind Möglichkeiten der Wahl oder Abstimmung, wenn dies über
Wege geschieht, die eine geheime Wahl oder Abstimmung ermöglichen und
zulassen, und der wählende oder abstimmende Pirat dabei seine Identität
nachweisen kann, ohne dass die Geheimheit der Wahl betroffen ist.

\subsubsection{§ 5}

(1) Bei der Bestimmung des Ergebnisses werden nicht abgegebene Stimmen
als nicht anwesend gewertet, sofern nichts anderes bestimmt ist.
Enthaltungen gelten als nicht abgegebene Stimmen.

(2) Die Ergebnisse einer Wahl oder Abstimmung müssen so veröffentlicht
werden, dass alle für die Wahl oder Abstimmung stimmberechtigten Piraten
sie einsehen können.

\subsubsection{§ 6}

(1) Jedem zu einer Wahl oder Abstimmung stimmberechtigten Pirat steht
das Recht zur Anfechtung der Wahl oder Abstimmung zu, wenn die
Verletzung von Bestimmungen der Satzung, des Parteiengesetzes, der
Wahlgesetze, des Verfassungsrechts oder eines anderen gültigen Gesetzes
oder Beschlusses als möglich erscheint. Sie ist bis zum 14. Tage nach
der Wahl oder Abstimmung zulässig und muss beim zuständigen Vorstand
eingereicht werden.

(2) Hält der Vorstand die Anfechtung für begründet, erklärt er die Wahl
oder Abstimmung für ungültig.

(3) Gegen die die Anfechtung versagende Entscheidung des Vorstandes ist
die Beschwerde beim Schiedsgericht zulässig.

\subsection{Besonderer Teil.}

\subsubsection{Erster Abschnitt. Wahlen.}

\paragraph{§ 7}

(1) Wahlen finden geheim statt, soweit diese Wahl- und
Abstimmungsordnung nicht etwas anderes bestimmt.

(2) Wahlen können nur stattfinden, wenn sie in der Tagesordnung
angekündigt worden sind, soweit diese Wahl- und Abstimmungsordnung nicht
etwas anderes bestimmt. Die Tagesordnung muss den stimmberechtigten
Piraten spätestens sieben Tage vor der Wahl in Textform zugehen. Bei
Nominierungen zu öffentlichen Ämtern gelten die entsprechenden
gesetzlichen Fristen.

\paragraph{§ 8}

(1) Die Kandidaten sollen gemeinsam in einem Wahldurchgang gewählt
werden.

(2) Gibt es bei Vorstandswahlen für ein Amt nur einen Kandidaten, findet
§ 13 entsprechende Anwendung mit der Maßgabe, dass Enthaltungen als
abgegebene Stimmen gelten. Bewerben sich mehrere Kandidaten um ein Amt,
findet eine Mehrheitswahl statt, bei der Enthaltungen als abgegebene
Stimmen gelten. Gewählt ist, wer die einfache Mehrheit auf sich
vereinigt.

(3) Direktkandidaten für Wahlen zum Europäischen Parlament, zum Bundes-
oder Landtag oder zu kommunalen Vertretungen werden entsprechend Absatz
2 gewählt. Listenkandidaten werden durch eine Akzeptanzwahl aufgestellt,
die in zwei Wahlgängen erfolgt. Im ersten Wahlgang werden so viele
künftige Listenkandidaten gewählt, wie in einer vorhergehenden
Abstimmung beschlossen wurde. Dieser Wahlgang entfällt, wenn sich
weniger Kandidaten für die Liste bewerben, als nach dem Beschluss
aufgestellt werden können. Im zweiten Wahlgang wird die Reihenfolge der
Kandidaten festgelegt, indem jeder stimmberechtigte Pirat eine um eins
erhöhte der Kandidatenzahl entsprechende Stimmenzahl bekommt, die er auf
einen oder mehrere Kandidaten verteilen kann. Die Aufstellung der
Listenkandidaten erfolgt nach absteigender Stimmenzahl.

(4) Versammlungsleiter, Moderatoren, Wahlleiter, Protokollführer und
andere zur Wahl gestellte Personen werden entsprechend Absatz 2 gewählt.
Solche Wahlen finden grundsätzlich nicht geheim statt und müssen nicht
in der Tagesordnung angekündigt werden.

\paragraph{§ 9}

Gibt es bei einer Wahl durch Stimmengleichheit kein eindeutiges
Ergebnis, ist für diese Kandidaten eine Stichwahl durchzuführen. Führt
diese ebenfalls zu keinem Ergebnis, entscheidet das Los.

\paragraph{§ 10}

Für Nachwahlen gelten die gleichen Bestimmungen wie für Wahlen. Die
Wahlperioden bleiben davon unberührt.

\subsubsection{Zweiter Abschnitt. Abstimmungen.}

\paragraph{§ 11}

Soweit in diesem Abschnitt nichts anderes bestimmt ist, finden die
Vorschriften zu den Wahlen entsprechende Anwendung.

\paragraph{§ 12}

Abstimmungen finden nicht geheim statt. Auf Antrag eines Zehntels der
anwesenden, stimmberechtigten Piraten kann mit einfacher Mehrheit geheim
abgestimmt werden. Über den Antrag wird in geheimer Abstimmung
entschieden.

\paragraph{§ 13}

(1) Abstimmungsfragen müssen so gestellt werden, dass sie mit Ja oder
Nein beantwortet werden können.

(2) Die zur Abstimmung gestellte Frage ist positiv beschieden, wenn die
Ja-Stimmen die Nein-Stimmen überwiegen.

(3) Die Anzahl der abgegebenen Stimmen muss mindestens fünfzig vom
Hundert der anwesenden Stimmen betragen.
