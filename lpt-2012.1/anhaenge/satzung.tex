\subsubsection{§ 1 - Name, Sitz und Tätigkeitsgebiet}

(1) \textsuperscript{1}Der Landesverband Mecklenburg-Vorpommern der
Piratenpartei Deutschland ist ein untergeordneter Gebietsverband auf
Landesebene gemäß der Satzung der Piratenpartei Deutschland
(Bundessatzung\textsuperscript{\href{\#cite\_note-0}{{[}1{]}}}).
\textsuperscript{2}Der Sitz des Landesverbandes und Ort der
Landesgeschäftsstelle ist Rostock.

(2) \textsuperscript{1}Der Landesverband Mecklenburg-Vorpommern der
Piratenpartei Deutschland führt einen Namen und eine Kurzbezeichnung.
\textsuperscript{2}Der Name lautet: \textbf{Piratenpartei Deutschland,
Landesverband Mecklenburg-Vorpommern}. \textsuperscript{3}Die offizielle
Abkürzung des Landesverbandes Mecklenburg-Vorpommern der Piratenpartei
Deutschland lautet: PIRATEN. \textsuperscript{4}Die Verwendung des
verkürzten Namens ``Piratenpartei MV'' ist zulässig.

(3) \textsuperscript{1}Untergeordnete Gliederungen des Landesverbandes
Mecklenburg-Vorpommern der Piratenpartei Deutschland führen den Namen
Piratenpartei Deutschland verbunden mit ihrer Organisationsstellung und
dem Namen der Gliederung. \textsuperscript{2}Den untergeordneten
Gliederungen wird die Verkürzung auf ``Piratenpartei'' in Verbindung mit
dem Gliederungsnamen erlaubt.

(4) \textsuperscript{1}Das Tätigkeitsgebiet des Landesverbandes
Mecklenburg-Vorpommern der Piratenpartei Deutschland ist das Bundesland
Mecklenburg-Vorpommern.

(5) \textsuperscript{1}Die im Landesverband Mecklenburg-Vorpommern der
Piratenpartei Deutschland organisierten Mitglieder werden
geschlechtsneutral als Piraten bezeichnet.

\subsubsection{§ 2 - Mitgliedschaft}

(1) \textsuperscript{1}Mitglied des Landesverbandes ist jedes Mitglied
der Piratenpartei Deutschland mit angezeigtem Wohnsitz in
Mecklenburg-Vorpommern.

(2) \textsuperscript{1}Der Landesverband führt ein Piratenverzeichnis.

\subsubsection{§ 3 - Erwerb der Mitgliedschaft}

(1) \textsuperscript{1}Der Erwerb der Mitgliedschaft der Piratenpartei
Deutschland wird durch die
Bundessatzung\textsuperscript{\href{\#cite\_note-1}{{[}2{]}}} geregelt.

(2) \textsuperscript{1}Jegliche Änderung am Bestand der Mitgliedsdaten
muss allen übergeordneten Gliederungen mitgeteilt werden.

\subsubsection{§ 4 - Rechte und Pflichten der Piraten}

\textsuperscript{1}Um eine Gleichbehandlung aller Piraten im
Landesverband zu gewährleisten, werden die Rechte und Pflichten der
Piraten des Landesverbandes allein durch die
Bundessatzung\textsuperscript{\href{\#cite\_note-2}{{[}3{]}}} geregelt.
\textsuperscript{2}Eine hiervon abweichende Regelung durch
untergeordnete Gliederungen ist unzulässig.

\subsubsection{§ 5 - Beendigung der Mitgliedschaft}

(1) \textsuperscript{1}Die Beendigung der Mitgliedschaft ist dem
Landesvorstand anzuzeigen.

(2) \textsuperscript{1}Die Beendigung der Mitgliedschaft in der
Piratenpartei Deutschland wird durch die
Bundessatzung\textsuperscript{\href{\#cite\_note-3}{{[}4{]}}} geregelt.

(3) \textsuperscript{1}Die Beendigung der Mitgliedschaft im
Landesverband erfolgt durch Wechsel des Wohnsitzes in ein anderes
Bundesland oder durch Beendigung der Mitgliedschaft in der Piratenpartei
Deutschland.

\subsubsection{§ 6 - Ordnungsmaßnahmen}

\textsuperscript{1}Die Regelungen zu den Ordnungsmaßnahmen, die in der
Bundessatzung\textsuperscript{\href{\#cite\_note-4}{{[}5{]}}} getroffen
werden, gelten entsprechend auch auf Landesebene.

\subsubsection{§ 7 - Gliederung}

\textsuperscript{1}Die Gliederung des Landesverbands regelt die
Bundessatzung\textsuperscript{\href{\#cite\_note-5}{{[}6{]}}}.
\textsuperscript{2}Zusammenschlüsse von Untergliederungen gleicher Ebene
sind zulässig.

\subsubsection{§ 8 - Bundespartei und Landesverbände}

\textsuperscript{1}Der Landesverband verpflichtet sich, den Regelungen
des Bundessatzung\textsuperscript{\href{\#cite\_note-6}{{[}7{]}}}
bezüglich des Verhältnisses von Bundespartei und Landesverbänden Folge
zu leisten und seine untergeordnete Gliederungen zu ebensolchem
Verhalten anzuhalten.

\subsubsection{§ 9 - Organe des Landesverbands}

\textsuperscript{1}Organe sind der Landesparteitag, das
Landesschiedsgericht und der Vorstand.

\subsubsection{§ 9a - Der Vorstand}

(1) \textsuperscript{1}Der Vorstand besteht aus dem Vorsitzenden, dem
stellvertretenden Vorsitzenden, dem Schatzmeister, dem politischen
Geschäftsführer und dem Generalsekretär.

(2) \textsuperscript{1}Der Vorstand vertritt den Landesverband nach
innen und außen. \textsuperscript{2}Er führt die Geschäfte auf Grundlage
der Beschlüsse der Parteiorgane.

(3) \textsuperscript{1}Die Mitglieder des Vorstands werden von einem
Landesparteitag mindestens jährlich in geheimer Wahl gewählt.
\textsuperscript{2}Der Vorstand bleibt bis zur Wahl eines neuen
Vorstands im Amt.

(4) \textsuperscript{1}Der Vorstand tritt in seiner Amtsperiode
mindestens zweimal zusammen. \textsuperscript{2}Er wird vom Vorsitzenden
oder bei dessen Verhinderung vom stellvertretendem Vorsitzenden mit
einer Frist von zwei Wochen unter Angabe der Tagesordnung und des
Tagungsortes einberufen. \textsuperscript{3}Bei außerordentlichen
Anlässen kann die Einberufung auch kurzfristiger erfolgen.

(5) \textsuperscript{1}Auf Antrag eines Zehntels der Piraten kann der
Vorstand zum Zusammentritt aufgefordert und mit aktuellen
Fragestellungen befasst werden. \textsuperscript{2}Die aktuelle
Mitgliederzahl ist regelmäßig zu veröffentlichen.

(6) \textsuperscript{1}Der Vorstand beschließt über alle
organisatorischen und politischen Fragen im Sinne der Beschlüsse des
Landesparteitages.

(7) \textsuperscript{1}Der Vorstand gibt sich eine Geschäftsordnung und
veröffentlicht diese angemessen. \textsuperscript{2}Sie umfasst u.a.
Regelungen zu:

\begin{enumerate}
\item
  Verwaltung der Mitgliedsdaten und deren Zugriff und Sicherung
\item
  Aufgaben und Kompetenzen der Vorstandsmitglieder
\item
  Dokumentation der Sitzungen
\item
  virtuellen oder fernmündlichen Vorstandssitzungen
\item
  Form und Umfang des Tätigkeitsberichts
\item
  Beurkundung von Beschlüssen des Vorstandes
\end{enumerate}
(8) \textsuperscript{1}Die Führung der Landesgeschäftsstelle wird durch
den Vorstand beauftragt und beaufsichtigt.

(9) \textsuperscript{1}Der Vorstand liefert zum Landesparteitag einen
schriftlichen Tätigkeitsbericht ab. \textsuperscript{2}Dieser umfasst
alle Tätigkeitsgebiete der Vorstandsmitglieder, wobei diese in
Eigenverantwortung des Einzelnen erstellt werden.
\textsuperscript{3}Wird der Vorstand insgesamt oder ein
Vorstandsmitglied nicht entlastet, so kann der Landesparteitag oder der
neue Vorstand gegen ihn Ansprüche gelten machen.
\textsuperscript{4}Tritt ein Vorstandsmitglied zurück, hat dieser
unverzüglich einen Tätigkeitsbericht zu erstellen und dem Vorstand
zuzuleiten.

(10) \textsuperscript{1}Tritt ein Vorstandsmitglied zurück bzw. kann
dieses seinen Aufgaben nicht mehr nachkommen, so geht seine Kompetenz
wenn möglich auf ein anderes Vorstandsmitglied über.
\textsuperscript{2}Der Vorstand gilt als nicht handlungsfähig, wenn mehr
als zwei Vorstandsmitglieder zurückgetreten sind oder ihren Aufgaben
nicht mehr nachkommen können oder wenn der Vorstand sich selbst für
handlungsunfähig erklärt. \textsuperscript{3}In einem solchen Fall wird
von dem dienstältesten Vorstand der direkt untergeordneten
Gliederungsebene zur Geschäftsführung eine kommissarische Vertretung
bestimmt. \textsuperscript{4}Die kommissarische Vertretung endet mit der
Neuwahl des gesamten Vorstandes auf einem unverzüglich einberufenem
außerordentlichen Parteitag.

(11) \textsuperscript{1}Tritt der gesamte Vorstand geschlossen zurück
oder kann seinen Aufgaben nicht mehr nachkommen, so führt der
dienstälteste Vorstand der direkt untergeordneten Gliederungsebene
kommissarisch die Geschäfte bis ein von ihm unverzüglich einberufener
außerordentlicher Parteitag einen neuen Vorstand gewählt hat.

\subsubsection{§ 9b - Der Landesparteitag}

(1) \textsuperscript{1}Der Landesparteitag ist die Mitgliederversammlung
auf Landesebene.

(2) \textsuperscript{1}Der Landesparteitag tagt mindestens einmal
jährlich. \textsuperscript{2}Die Einberufung erfolgt aufgrund
Vorstandsbeschluss. \textsuperscript{3}Der Vorstand lädt jedes Mitglied
persönlich mindestens vier Wochen vor dem Landesparteitag in Textform
(vorranging per E-Mail, nachrangig per Brief) ein.
\textsuperscript{4}Die Einladung hat Angaben zum Tagungsort,
Tagungsbeginn, vorläufiger Tagesordnung und der Angabe, wo weitere,
aktuelle Veröffentlichungen gemacht werden, zu enthalten.
\textsuperscript{5}Spätestens eine Woche vor dem Parteitag sind die
Tagesordnung in aktueller Fassung, die geplante Tagungsdauer und alle
bis dahin dem Vorstand eingereichten Anträge im Wortlaut zu
veröffentlichen.

(3) \textsuperscript{1}Ein außerordentlicher Landesparteitag wird
unverzüglich einberufen, wenn mindestens eins der folgenden Ereignisse
eintritt:

\begin{enumerate}
\item
  Der Vorstand ist handlungsunfähig.
\item
  Ein Zehntel der stimmberechtigten Piraten des Landesverbandes
  Mecklenburg-Vorpommern beantragt es.
\item
  Der Landesvorstand beschließt es mit einer Zweidrittelmehrheit.
\end{enumerate}
\textsuperscript{2}Es ist ein Grund für die Einberufung zu benennen.
\textsuperscript{3}Der außerordentliche Parteitag darf sich nur mit dem
benannten Grund der Einberufung befassen. \textsuperscript{4}In
dringenden Fällen kann mit einer verkürzten Frist von mindestens zwei
Wochen eingeladen werden.

(4) \textsuperscript{1}Der Landesparteitag nimmt den Tätigkeitsbericht
des Vorstandes entgegen und entscheidet daraufhin über seine Entlastung.

(5) \textsuperscript{1}Über den Landesparteitag, dessen Beschlüsse und
Wahlen wird ein Ergebnisprotokoll gefertigt, das von der
Protokollführung, der Versammlungsleitung und der Wahlleitung
unterschrieben und anschließend veröffentlicht wird.

(6) \textsuperscript{1}Der Landesparteitag wählt mindestens zwei
Rechnungsprüfer, die den finanziellen Teil des Tätigkeitsberichtes des
Vorstandes vor der Beschlussfassung über ihn prüfen.
\textsuperscript{2}Das Ergebnis der Prüfung wird dem Landesparteitag
verkündet und zu Protokoll genommen. \textsuperscript{3}Danach sind die
Rechnungsprüfer aus ihrer Funktion entlassen.

(7) \textsuperscript{1}Der Landesparteitag wählt mindestens zwei
Kassenprüfer. \textsuperscript{2}Diesen obliegen die Vorprüfung des
finanziellen Tätigkeitsberichtes für den folgenden Landesparteitag und
die Vorprüfung, ob die Finanzordnung und das PartG eingehalten wird.
\textsuperscript{3}Sie haben das Recht, Einsicht in alle
finanzrelevanten Unterlagen zu verlangen, und auf Wunsch Kopien
persönlich ausgehändigt zu bekommen. \textsuperscript{4}Sie sind
angehalten, etwa zwei Wochen vor dem Landesparteitag die letzte
Vorprüfung der Finanzen durchzuführen. \textsuperscript{5}Ihre Amtszeit
endet durch Austritt, Rücktritt, Entlassung durch den Landesparteitag
oder mit Wahl ihrer Nachfolger.

\subsubsection{§ 10 - Bewerberaufstellung für die Wahlen zu
Volksvertretungen}

(1) \textsuperscript{1}Die Bewerberaufstellung für die Wahlen zu
Volksvertretungen erfolgt nach den Regularien der einschlägigen Gesetze
sowie den Vorgaben der
Bundessatzung\textsuperscript{\href{\#cite\_note-7}{{[}8{]}}}.

(2) \textsuperscript{1}Die Aufstellung kann sowohl als
Mitgliederversammlung des zuständigen Stimm- bzw. Wahlkreises als auch
im Rahmen einer anderen Mitgliederversammlung stattfinden, sofern
gewährleistet wird, dass alle Stimmberechtigten in angemessener Zeit und
Form eingeladen wurden und nur die Stimmberechtigten an der Wahl
teilnehmen. \textsuperscript{2}Die Einladung muss dabei explizit auf die
Bewerberaufstellung hinweisen.

\subsubsection{§ 11 - Satzungs- und Programmänderung}

(1) \textsuperscript{1}Änderungen der Landessatzung und des Programms
können nur von einem Landesparteitag mit einer Zweidrittelmehrheit
beschlossen werden. \textsuperscript{2}Besteht das dringende Erfordernis
einer Satzungsänderung zwischen zwei Landesparteitagen, so kann die
Satzung auch geändert werden, wenn mindestens zwei Dritteln der Piraten
dem Änderungsantrag schriftlich zustimmen.

(2) \textsuperscript{1}Über einen Antrag auf Satzungs- oder
Programmänderung auf einem Landesparteitag kann nur abgestimmt werden,
wenn er mindestens zwei Wochen vor Beginn des Landesparteitages beim
Vorstand eingegangen ist.

(3) \textsuperscript{1}Der Landesverband übernimmt das Grundsatzprogramm
der Piratenpartei Deutschland. \textsuperscript{2}Vom Landesparteitag
kann ein eigenes Wahlprogramm für Kommunal- und Landtagswahlen
verabschiedet werden. \textsuperscript{3}Dieses muss auf den Werten des
Grundsatzprogrammes basieren.

\subsubsection{§ 12 - Auflösung und Verschmelzung}

\textsuperscript{1}Die Auflösung oder Verschmelzung regelt die
Bundessatzung\textsuperscript{\href{\#cite\_note-8}{{[}9{]}}}.

\subsubsection{§ 13 - Parteiämter}

\textsuperscript{1}Die Regelung der
Bundessatzung\textsuperscript{\href{\#cite\_note-9}{{[}10{]}}} zu den
Parteiämtern findet Anwendung.

\bigskip

\subsection{Abschnitt B: Finanzordnung}

\subsubsection{§ 16 Finanzordnung}

\textsuperscript{1}Die Finanzordnung der
Bundessatzung\textsuperscript{\href{\#cite\_note-10}{{[}11{]}}} findet
entsprechende Anwendung.

\subsection{Abschnitt C: Schiedsgerichtsordnung}

\subsubsection{§ 15 Landesschiedsgericht}

\textsuperscript{1}Für das Landesschiedsgericht gilt die
Bundesschiedsgerichtsordnung\textsuperscript{\href{\#cite\_note-11}{{[}12{]}}}.

\subsection{Abschnitt D: Organisatorisches}

\subsubsection{§ 16 Wahlordnung}

\textsuperscript{1}Der Landesparteitag regelt das Verfahren von Wahlen
und Abstimmungen in einer
Wahlordnung\textsuperscript{\href{\#cite\_note-12}{{[}13{]}}}.

\subsubsection{§ 17 Schlussbestimmungen}

\textsuperscript{1}Diese Satzung tritt mit Verabschiedung durch den
Landesparteitag in Kraft.

\subsection{Referenzen}

\begin{enumerate}
\item
  \href{\#cite\_ref-0}{↑} \href{/Bundessatzung}{Bundessatzung}
\item
  \href{\#cite\_ref-1}{↑}
  \href{/Bundessatzung\#.C2.A7\_3\_-\_Erwerb\_der\_Mitgliedschaft}{§ 3
  Bundessatzung (``Erwerb der Mitgliedschaft'')}
\item
  \href{\#cite\_ref-2}{↑}
  \href{/Bundessatzung\#.C2.A7\_4\_-\_Rechte\_und\_Pflichten\_der\_Piraten}{§
  4 Bundessatzung (``Rechte und Pflichten der Piraten'')}
\item
  \href{\#cite\_ref-3}{↑}
  \href{/Bundessatzung\#.C2.A7\_5\_-\_Beendigung\_der\_Mitgliedschaft}{§
  5 Bundessatzung (``Beendigung der Mitgliedschaft'')}
\item
  \href{\#cite\_ref-4}{↑}
  \href{/Bundessatzung\#.C2.A7\_6\_-\_Ordnungsma.C3.9Fnahmen}{§ 6
  Bundessatzung (``Ordnungsmaßnahmen'')}
\item
  \href{\#cite\_ref-5}{↑}
  \href{/Bundessatzung\#.C2.A7\_7\_-\_Gliederung}{§ 7 Bundessatzung
  (``Gliederung'')}
\item
  \href{\#cite\_ref-6}{↑}
  \href{/Bundessatzung\#.C2.A7\_8\_-\_Bundespartei\_und\_Landesverb.C3.A4nde}{§
  8 Bundessatzung (``Bundespartei und Landesverbände'')}
\item
  \href{\#cite\_ref-7}{↑}
  \href{/Bundessatzung\#.C2.A7\_10\_-\_Bewerberaufstellung\_f.C3.BCr\_die\_Wahlen\_zu\_Volksvertretungen}{§
  10 Bundessatzung (``Bewerberaufstellung für die Wahlen zu
  Volksvertretungen'')}
\item
  \href{\#cite\_ref-8}{↑}
  \href{/Bundessatzung\#.C2.A7\_13\_-\_Aufl.C3.B6sung\_und\_Verschmelzung}{§
  13 Bundessatzung (``Auflösung und Verschmelzung'')}
\item
  \href{\#cite\_ref-9}{↑}
  \href{/Bundessatzung\#.C2.A7\_15\_-\_Partei.C3.A4mter}{§ 15
  Bundessatzung (``Parteiämter'')}
\item
  \href{\#cite\_ref-10}{↑}
  \href{/Bundessatzung\#Abschnitt\_B:\_Finanzordnung}{Abschnitt B
  Bundessatzung (``Finanzordnung'')}
\item
  \href{\#cite\_ref-11}{↑}
  \href{/Bundessatzung\#Abschnitt\_C:\_Schiedsgerichtsordnung}{Abschnitt
  C Bundessatzung (``Schiedsgerichtsordnung'')}
\item
  \href{\#cite\_ref-12}{↑} \href{/MV:Wahlordnung}{Wahl- und
  Abstimmungsordnung der Piraten in Mecklenburg-Vorpommern}
\end{enumerate}
