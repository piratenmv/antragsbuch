\section{Antrag}

Die Piratenpartei erkennt an, dass trotz der formalen Gleichberechtigung der Geschlechter in Deutschland eine faktische Gleichstellung noch nicht erreicht ist. Viele Menschen aller Geschlechter sind durch die an ihr Geschlecht geknüpften Rollenbilder und Erwartungshaltungen in ihrer individuellen Freiheit und ihren Entfaltungsmöglichkeiten eingeschränkt. Studien zeigen signifikante Lohnunterschiede zwischen Frauen und Männern, in vielen Berufen und gerade in höheren Positionen in Unternehmen sind Frauen nach wie vor stark unterrepräsentiert. Auch Parteien bevorzugen durch ihren Aufbau und ihre Strukturen eher ein Verhalten, welches klassischerweise Männern zugeordnet wird. Dominanz und Prahlerei verhelfen in Parteien besser zum Erfolg als Bescheidenheit und Umsicht, Eigenschaften, die eher Frauen zugeordnet werden.

Die Piratenpartei setzt sich sowohl nach innen als auch nach außen für Gleichstellung ein.

\section{Begründung}

Der durchschnittliche Lohnunterschied (``Gender Pay Gap'') zwischen Frauen und Männern beträgt laut einer kürzlich veröffentlichten OECD-Studie ca. 23\%. Bei gleicher Tätigkeit verdient eine Frau im Schnitt 8\% weniger als ein Mann. \\Der Frauenanteil in den Vorständen und Aufsichtsräten liegt bei 3,7 Prozent beziehungsweise 14 Prozent.

Dies hat weitreichende Folgen. So ist auch bei modernen Paaren die Entscheidung, welcher Elternteil in Elternzeit geht, nach wie vor meist die Mutter die wirtschaftlichere Wahl, da man eher auf das geringere Gehalt verzichten möchte.\\Die wichtigen, aber schlecht bezahlten sozialen Berufe werden fast ausschließlich von Frauen ergriffen. \\Auch die Politik muss sich verändern. Mit einem politischen Amt ist Familie praktisch nicht vereinbar.

\begin{itemize}
\item
  Zu anonymen Bewerbungsverfahren\footnote{\url{http://www.bundestag.de/dokumente/analysen/2012/Anonymes\_Bewerbungsverfahren.pdf}} - ``63 Prozent von ihnen haben sich demzufolge schon mal aufgrund ihres Alters, Geschlechts, ihrer Herkunft, des Familienstandes oder einer Behinderung in einem Bewerbungsverfahren benachteiligt gefühlt.''
\item
  OECD-Studie zu Gender Pay Gap\footnote{\url{http://www.zeit.de/karriere/beruf/2012-03/schlusslicht-gehaltsunterschied-deutschland}}
\item
  Zu Frauen in Führungspositionen\footnote{\url{http://www.spiegel.de/wirtschaft/unternehmen/dax-unternehmen-frauenanteil-in-fuehrungsetagen-steigt-nur-leicht-a-841579.html}}
\end{itemize}
