\section{Antrag}

Die Piratenpartei Mecklenburg-Vorpommern setzt sich dafür ein, dass ein landesweit einheitliches Rücken- bzw. Helm-Kennzeichnungssystem für Feuerwehren, Technisches Hilfswerk und den Rettungsdienst ausgearbeitet und in Dienst gestellt wird. Zweck dieser Kennzeichnung ist die Kenntlichmachung der Aufgabe und des Einsatzgebietes in einer Einsatz und/oder Krisensituation.

\section{Begründung}

Einsätze in einem Flächenland wie Mecklenburg-Vorpommern werden selten von nur einer Dienststelle angefahren. Oft werden Einsatzzüge von verschiedenen Orten herangezogen. Aus diesem Grund kennen sich die einzelnen Einsatzkräfte nicht, müssen aber in einer Krisensituation aufeinander vertrauen können. Auch darf es nicht zur Fehlkommunikation kommen. Dazu kommt, nicht nur Einsatzleitung oder Abschnittsleiter müssen oft für sie fremde Rettungskräfte ansprechen, auch die Teams untereinander müssen kommunizieren, wobei meist weniger der Name eine Rolle spielt, als das ausgefüllt Amt/die zu erfüllende Aufgabe.\\In einem Einsatz müssen Maschinisten, Melder, Angriffstrupps oder Techniker oft gezielt angesprochen und kommandiert werden. Gruppen werden gebildet und Teams zusammengestellt. Teilweise werden Rangkennzeichungen vorne am Helm getragen, Aufgabenkennzeichnung gibt es so aber nicht. Durch die Uniformen kann man immerhin Jugend-, Berufs- oder Werksfeuerwehr, Technisches Hilfswerk und Rettungsdienst unterscheiden.\\Die Idee dieses Antrags ist, das weit sichtbar, ein einheitliches Aufgabensymbol an Rücken, Brust und Helm eines jeden Einsatzteilnehmers angebracht wird. So kann jeder mit einem Blick den Aufgabenbereich des anderen erkennen und so direkt die mit der Aufgabe betreute Personen ansprechen.\\Dies würde im Krisenfall nicht nur Zeit sparen, es würde auch helfen Fehler zu verringern, Prozesse effektiver zu gestalten und am Ende, durch erfolgreichere Einsätze, Leben retten.
