\section{Antrag}

Änderung der Kommunalverfassung und Landesverfassung dahin gehend, dass es keine nicht-öffentlichen Beratungen und nicht-öffentlichen Tagesordnungspunkte bei Gemeindevertretersitzungen, Kreistagssitzungen und Landtagssitzungen geben darf. Auch die Abstimmungen der politisch Tätigen sollen öffentlich sein. Bei Personen bezogenen Daten werden Pseudonyme verwendet, um Datenschutz zu gewährleisten. Ergänzend sollen in den Haushalten Mittel bereit gestellt werden, um die technischen Voraussetzungen für ein Live-Streaming (Bild+Ton) der Sitzungen im Internet schaffen zu können. Um auch kleinen Gemeinden diese Möglichkeit einzuräumen, soll der Server der Landesregierung dafür genutzt werden dürfen, wodurch die Kosten des Streaming für die jeweiligen Gemeinden erheblich gesenkt werden.

\section{Begründung}

Nur unbedingte Transparenz verhindert Amtsmissbrauch und Korruption. Jeder Bürger muss die Möglichkeit haben, politische Entscheidungsabläufe zu verfolgen.
