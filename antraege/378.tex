\section{Vorbemerkung}

Dieser Antrag ist Teil eines Antrages, den ein Bürger bzw. eine Bürgerin über das Vorstandsportal an den Landesparteitag gerichtet hat\footnote{\url{https://redmine.piratenpartei-mv.de/redmine/issues/151}}.

\section{Antrag}

Der Landesparteitag möge beschließen, dass die folgende Aussage in das Wahlprogramm der Piratenpartei Mecklenburg-Vorpommern aufgenommen wird:

Die Piratenpartei Mecklenburg-Vorpommern setzt sich für eine institutionalisierte, konsequent demokratisch verfasste und daher selbstverwaltete Justiz in Mecklenburg-Vorpommern ein.

\section{Begründung}

Zu den wichtigsten Anliegen der Piratenpartei gehört »Mehr Demokratie«. Mit ihrem Programm erachtet die Piratenpartei Deutschland eine möglichst große und sinnvolle Gewaltenteilung im Staat als absolut notwendig:

»Gerade die Unabhängigkeit der Judikative, vor allem des Bundesverfassungsgerichtes, gilt es zu stärken und zu fördern, da es sich mehrfach als Schützer der Grundrechte der Einzelnen vor Legislative und Exekutive erwiesen hat.«

Der Deutsche Richterbund, der größte Berufsverband der Richterinnen und Richter, Staatsanwältinnen und Staatsanwälte in Deutschland, fordert bereits seit Jahren eine selbstverwaltete Justiz, wie sie in fast allen Staaten Europas schon üblich ist\footnote{\url{http://www.drb.de/cms/index.php?id=569}}.
