\section{Antrag}

In der Satzung wird folgender Absatz mit der nächst höheren freien Absatznummer des § 9b eingefügt:

\begin{quote}
Die Entscheidungen des Landesparteitags werden mit einfacher Mehrheit der abgegebenen gültigen Stimmen beschlossen. Bei Stimmengleichheit gilt ein Antrag als abgelehnt. Stimmenthaltungen werden als ungültige Stimmen gewertet.

\end{quote}
Des weiteren wird beantragt, in § 11 Absatz 1 in den Satz 1 nach dem Wort „Zweidrittelmehrheit`` folgende Worte einzufügen:

\begin{quote}
der abgegebenen gültigen Stimmen

\end{quote}
\section{Begründung}

Im Prinzip ist keine Änderung nötig, da die Rechtsprechung\footnote{\url{http://wiki.piratenpartei.de/Antragsfabrik/Beschlussfassung\_2.0\#Rechtsprechung}} durch den Bundesgerichtshof dies klar gestellt hat, allerdings gab es vor dem Bundesschiedsgericht eine Klage\footnote{\url{http://wiki.piratenpartei.de/images/5/5f/BSG\_Urteil\_BSG\_2008-05-18\_1.pdf}} und der BGH empfiehlt eine Regelung zur Klarstellung in der Satzung

Damit keine Missverständnisse aufkommen:

\begin{itemize}
\item
  hiermit werden Enthaltungen nicht ungültig, sondern werden bei einer Zählung / Berechnung ebenso wie ungültige Stimmen behandelt und haben keine Auswirkung auf das Ergebnis
\item
  allgemeine Abstimmungen werden mit mehr als 50\% Ja-Stimmen entschieden
\item
  für die Sonderfälle SÄA und PÄA sind 2/3 nötig
\end{itemize}
\section{Anmerkung}

\begin{itemize}
\item
  Ein sehr ähnlicher Antrag\footnote{\url{http://wiki.piratenpartei.de/Bundesparteitag\_2012.1/Antragsportal/Satzungsänderungsantrag\_-\_006}} wurde auf dem Bundesparteitag in Neumünster angenommen. Eine Klärung ist auch auf Landesebene wünschenswert.
\end{itemize}
