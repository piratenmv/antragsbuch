\section{Antrag}

Die Piratenpartei Mecklenburg-Vorpommern setzt sich dafür ein, dass alle Einsatzleitstellen der Rettungsdienste in Mecklenburg-Vorpommern untereinander so vernetzt werden, dass nicht mehr nur die zuständige Wache alarmiert wird, sondern vor allem die wirklich am günstigsten und nächsten gelegene Wache. Des Weiteren soll, so weit es die Informationslage ermöglicht, das gesamte, bei Alarmierung bekannte, benötigte Material ausrücken. Sollte es nicht in der alarmierten Dienstelle vorrätig ist, von der nächst möglichen, um lange Nachforderungszeiten zu verhindern.

\section{Begründung}

Die jetzige Lage in Mecklenburg-Vorpommern bei einem Notfall lässt nur die Alarmierung der zuständigen Dienststelle zu. Dies muss nicht die am nächsten oder besten gelegene Dienststelle sein. Auch muss die alarmierte Stelle das benötigte Material vorrätig haben.\\Erst an der Krisenstelle angekommen, werden nach Sichtung der Lage die wirklich benötigten Kräfte nach geordert. Dies liegt daran, das wenn eine Dienstelle unnötig Nachfordert, diese das bezahlen muss. Dies hat zur Folge, das unnötig Zeit verstreichen kann, bis das benötigte, aber noch nicht vor Ort befindliche Material ankommt. Des weiteren vergeht auch unnötiger Weise Zeit, da durch die aktuelle Regelung Zuständigkeit vor geographischer Lage geht.\\Wären die Dienstellen wie gefordert vernetzt, könnte man den Einsatz an die wirklich benötigte Dienststelle leiten, diese könnte sofort bekanntgeben, ob sie die geforderten Materialien besitzt und gegebenenfalls schon auf dieser frühen Ebene eine weitere Dienststelle anfordern.\\Dies würde im Krisenfall nicht nur Zeit sparen, es würde auch helfen Fehler zu verringern, Prozesse effektiver zu gestalten und am Ende, durch erfolgreichere Einsätze, Leben retten.
