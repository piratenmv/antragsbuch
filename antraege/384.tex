\section{Antrag}

Die Piratenpartei Mecklenburg-Vorpommern setzt sich dafür ein, dass ein auf Klassenstufen und Alter angepasstes Programm, zur Förderung von Ersthelfermaßnahmen in allgemein- und weiterbildenden Schulen ausgearbeitet, eingeführt und regelmäßig durchgeführt wird, da es nicht ausreicht, einmal im Leben an einer Schulungsmaßnahme in lebensrettenden Sofortmaßnahmen teilzunehmen.\\Die Piratenpartei Mecklenburg-Vorpommern fordert daher, dass Erste Hilfe in die Schulprogramme für Schüler ab der 5. Klasse freiwillig und ab der 7. Klasse (Wiederbelebungsunterricht\footnote{\url{http://www.aerzteblatt.de/archiv/70015/Wiederbelebungsunterricht-bei-Schuelern-Ab-der-siebten-Klasse-sinnvoll}}) verpflichtend, zum Beispiel im Rahmen von jährlichen Projekttagen oder anderen regelmäßigen Aktionen an Schulen aufgenommen und angeboten wird.\\Darüber hinaus setzen wir uns für die Einrichtung und Förderung von Schulsanitätsdiensten ein, die auf freiwilliger Basis beruhen und das Verantwortungsbewusstsein der Schüler fördern. Die bereits bestehenden Angebote der Ersten Hilfe sollen für Interessierte kostenfrei angeboten werden.

\section{Begründung}

Über 80.000 Menschen sterben in Deutschland am plötzlichen Herztod, das bedeutet statistisch betrachtet alle fünf Minuten ein Bundesbürger. 5.000 Menschen sterben jedes Jahr in Deutschland, weil nicht rechtzeitig Erste Hilfe geleistet wird. Nach einem Zusammenbruch sinkt pro Minute die Überlebensrate um 10\%, der Notarzt oder Rettungsdienst kann gar nicht so schnell vor Ort sein. So ist der plötzliche Herztod eine der häufigsten Todesursachen und eine der größten medizinischen und gesellschaftlichen Herausforderungen unserer Zeit.\\Die meisten Menschen fühlen sich aber zu unsicher, um im Notfall Erste Hilfe zu leisten. Unser Wunsch ist, dass junge Menschen selbstverständlich Erste Hilfe leisten. Erste Hilfe ist ab der 7. Klasse problemlos theoretisch erlernbar und praktisch mit Erfolg durchführbar, früher macht es aber auf jeden Fall schon Sinn (Wiederbelebungsunterricht\footnote{\url{http://www.aerzteblatt.de/archiv/70015/Wiederbelebungsunterricht-bei-Schuelern-Ab-der-siebten-Klasse-sinnvoll}}). Durch die Maßnahme könnte selbst bei vorsichtiger Schätzung eine Steigerung der Überlebensrate nach Herz-Kreislaufstillstand von 10--20\% erreicht werden.\\Die Kurse und Angebote sollen ab der 5. Klasse für interessierte Schüler, die vielleicht in ihrer Körperlichen und Geistigen Entwicklung schneller fortgeschrittenen sind zu Verfügung stehen. Schüler ab der 7. Klasse sind laut Studienlage soweit, sowohl theoretische Inhalte zum Thema Wiederbelebung zu erlernen als auch praktisch bei einem Erwachsenen die spezifischen Maßnahmen durchzuführen. Die Ergebnisse der Studie sprechen dafür, Wiederbelebungskurse als Pflichtveranstaltung in den Schulunterricht der 7. oder 8. Klasse einzuführen

\section{Hinweis}

Im Original ist dieser Antrag von Daniel Düngel und BrittaS. Dieser Antrag ist für mich ein Anliegen, da ich mich im Rahmen meines Studiums mit den Gegebenheiten von Herz-Kreislauf-Notfälle mehrere Jahre lang auseinander gesetzt habe und den Erstautor des verlinkten Beitrags gut kenne. Seine Ausführungen sind mehr als seriös und man kann mit geringen Mitteln eine Merkbare verbesserung der Überlebenschance der Bürger von MV erreichen.
