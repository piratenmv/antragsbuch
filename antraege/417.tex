\section{Antrag}

\subsection{Barrierefreier und maschinenlesbarer Haushalt}

Die Piratenpartei Mecklenburg-Vorpommern setzt sich dafür ein, dass die Haushaltsrechnungen, Haushaltsentwürfe und Unterlagen über die mittelfristige Finanzplanungen auf Landes-, Kreis-und Gemeindeebene spätestens zum Zeitpunkt der Vorlage an das zuständige Gremium und mindestens vier Wochen vor dem Termin einer beschlussrelevanten Sitzung des Gremiums nicht nur als PDF-Dokument, sondern auch in einer Weise digital veröffentlich werden (vorzugsweise Tabellendokument, OfficeOpenXML (OOXML) oder Open Document Format (ODF), die eine weitergehende Auswertung der Unterlagen durch interessierte Bürgerinnen und Bürger barrierefrei und maschinenlesbar zulässt. Die Unterlagen zur Haushaltsplanung sollen vollständig digital einsehbar sein und neben den Haushaltsansätzen des Vorjahres auch die Ergebnisse des abgelaufenen Haushaltsjahres, die Haushaltsansätze des kommenden Haushaltsjahres und auch die vollständigen Begründungen je Einzelposition enthalten. Vorbemerkungen, Erklärungen zu Deckungsfähigkeiten sowie die Anlagen zum Haushaltsplan sind ebenso digital auszuweisen.

\section{Begründung}

Nur durch die frühzeitige Darstellung der Haushaltsplanung und der eröffneten Möglichkeit, die zugehörigen Dokumente nach frei festzulegenden Kriterien zu filtern, kann bürgernahe Transparenz in Haushaltsfragen gewährleistet werden. Datenschutzrechtliche Gründe, die einer Veröffentlichung zuwider stehen, existieren nicht. Vielmehr haben die Bürger nach dem Informationsfreiheitsgesetz einen Rechtsanspruch auf diese Informationen und müssen sich zumindest darauf verlassen können, dass die Verordneten des beschlussgebenden Gremiums hinreichende Möglichkeiten zur Einsichtnahme in alle erforderlichen Unterlagen hatten.

Um einen handhabbaren Umgang mit den Datenmengen zu gewährleisten, sind die Haushaltspläne barrierefrei und maschinenlesbar zu publizieren, Beispielsweise als Tabellendokument oder ggf. einfach als HTML oder Textfile, jedoch nicht als ein gescanntes PDF. Da der Haushaltsentwurf und der anschließende Beschluss sich in jedem Fall an der bestehenden Rechtsgrundlage, der Bedarfssituation und der Entscheidungsfreiheit der Abgeordneten orientiert, ist eine rechtswidrige oder auch nur kontraproduktive Umgestaltung der Haushaltsansätze durch diese geschaffene Transparenz erschwert. Vielmehr wird einer ungewollten Manipulation der Haushaltszahlen vorgebeugt. Eine Überprüfung der Dokumente auf durchgeführte Änderungen zum vorherigen Ansatz ist jederzeit möglich. Die durch die geforderte Vorveröffentlichung geschaffene Transparenz erleichtert die Kommunikation mit den Bürgern, stärkt den beschlossenen Entwurf das zuständige Gremium und beugt einer ungewollten Einflussnahme vor.

\section{Anmerkung}

Der Antrag wurde aus dem Antragsbuch des LPT 2012.1 aus Brandenburg (Antrag WP091\footnote{\url{http://wiki.piratenbrandenburg.de/Antragsfabrik/Barrierefreier\_und\_maschinenlesbarer\_Haushalt}}) übergenommen. Dort wurde der Antrag angenommen.
