\section{Antrag}

Die Piratenpartei Mecklenburg-Vorpommern spricht sich für den flächendeckenden Erhalt der Geburtskliniken und Perinatalzentren in Mecklenburg-Vorpommern aus, um eine angemessene moderne Versorgung von Früh- und Neugeborenen und deren Müttern in unserem Land zu gewährleisten. Ein weiterer Abbau der Versorgungsqualität aus wirtschaftlichen Gründen um Kosten zu sparen, kann in einem Bundesland nicht toleriert werden, welches Gesundheitsland Nr.1 sein will.\\Wir fordern, dass das Land die Krankenhäuser in den einzelnen Kreisen durch die Absicherung der Grundausstattung für den Betrieb der Geburtskliniken unterstützt soweit der wirtschaftliche Betrieb entsprechend den gesetzlich vorgeschriebenen Qualitätsstandards für die Versorgung von Früh- und Neugeborenen nicht gewährleistet werden kann. (GNPI Strukturempfehlung\footnote{\url{http://www.dggg.de/fileadmin/public\_docs/Leitlinien/3-6-3-gba-2009-02-19.pdf}})\\Wir sind gegen die Schließung der Geburtskliniken sowie gegen das Absenken der Versorgungsqualtät in den Geburtskliniken unserer Krankenhäuser aufgrund betriebswirtschaftlicher Entscheidungen.

\section{Begründung}

Geburtskliniken und Perinatalzentren sind Kompetenzzentren in unserem Land zur flächendeckenden Sicherung der ärztlichen Versorgung von Früh- und Neugeborenen und deren Mütter. Zur Qualitätssicherung der Versorgung von Früh- und Neugeborenen sind Maßnahmen bzgl. der Struktur-, Prozess- und Ergebnisqualität festgelegt worden, mit dem Ziel die Säuglingssterblichkeit und frühkindlichen Behinderungen zu verringern (GNPI Strukturempfehlung\footnote{\url{http://www.dggg.de/fileadmin/public\_docs/Leitlinien/3-6-3-gba-2009-02-19.pdf}}). Je nach Qualitätslevel fallen unterschiedliche hohe Vorhaltekosten beispielsweise für Fachpersonal und spezielle Geräte in den Krankenhäusern an. Durch sinkende Geburtenzahlen und steigende Kosten ist der kostendeckende betriebwirtschaftliche Betrieb von Geburtkliniken je nach Region nur erschwert bzw. nicht mehr möglich. Für den flächendeckenden Erhalt der Geburtkliniken und Perinatalzentren sowie die Versorgungslevel in den Häusern sind Zuschüsse erforderlich.\\Die Geburt ist auch heute immer noch eine gefährliche Situation für Mutter und Kind. Darüber kann auch nicht die Tendenz in Richtung einer natürlichen Geburt in Geburtshäusern hinwegtäuschen. Screening und Risikoabschätzung kann einen hohen Anteil von Problemen und Gefahren bei der Geburt verhindern. Betroffenen Müttern wird in dolchen prognostizierten Fällen stark von einer Geburt außerhalb der Klinik abgeraten, oder die werdenden Mütter wird, falls noch genug Zeit vorhanden ist, frühzeitig in gut ausgestattete und mit Geburten erfahrene Kliniken überführt.\\All dies kann aber nicht die kritisch und super-kritisch gewordenen Geburtsversuche in Geburtenhäusern oder unerwartet Geburtsversuche außerhalb geschützter Orte abdecken. Dort geht es dann meist um Minuten und eine Anfahrt von bis zu 20 Minuten kann tödlich für das Ungeborene und sogar für die Mutter sein.\\Zusammenfassend sollte man bedenken, das es sich ein Bundesland, das weiterhin attraktiv für junge Paare mit Kinderwunsch sein will, nicht leisten, in der Fläche das Leben seiner ungeborenen Bürger durch eine unzureichende oder nicht vorhandene Versorgung an Geburtshilfe und Neonatologie aufs Spiel zu setzen.

\section{Hinweis}

Die Grundidee kommt von einem Antrag aus Niedersachsen. Umgebaut und ausformuliert wurde er von Jan Tamm und Klaus Klepik. Weiterer Input kam von Prof. Koepcke. Der Antrag liegt uns am Herzen, da wir so hoffen, den weiteren Kahlschlag in der Geburtshilfe zu stoppen. Ohne eine solche Unterstützung würde es sich in Zukunft auf 5--6 Geburtskliniken zuspitzen, was Mecklenburg-Vorpommern bei weitem nicht abdecken würde.``
