\section{Antrag}

\subsection{Lärmemissionen}

Lärm stellt eine der größten Umweltbelastungen in Europa dar. Die Piraten in Mecklenburg-Vorpommern erkennen Lärm als Gesundheitsrisiko an. Jeder Mensch hat das Recht auf Schutz vor Lärm. Dieser Schutz ist unter Berücksichtigung der neuesten wissenschaftlichen Erkenntnisse zu gewährleisten. Auch die Lärmbelastung von Tieren ist zu beachten und auf das mögliche Mindestmaß zu reduzieren. Aktiver Schutz (an der Quelle) ist passivem Schutz (am Wirkungsort) vorzuziehen. Lärmemissionen sind in ihrer Wirkung gesamtheitlich zu betrachten. Dabei sind z.B. wirtschaftliche Chancen den gesundheitlichen Risiken gegenüberzustellen. Zur transparenten und bürgerfreundlichen Kennzeichnung von Lärmemissionen unterstützt die Piratenpartei Mecklenburg-Vorpommern die Einführung eines Lärmlabels.

\section{Begründung}

Lärmemissionen als Gesundheitsgefährdung anzuerkennen ist noch relativ neu. Die Anerkennung fällt schwer, weil damit für die Menschen das Recht auf Schutz vor Lärm (und damit körperliche Unversehrtheit nach GG) einbezogen ist. Das Recht auf Schutz nach neuesten wissenschaftlichen Erkenntnissen führt dazu, dass im Gegensatz zur bisherigen Praxis bei der Errichtung oder Veränderungen an Anlagen, Infrastrukturen usw. Schutzmaßnahmen angewendet werden müssen. Damit wird es zukünftig unmöglich, z.B. Bahnlinien oder Straßen zu sanieren oder zu erweitern, ohne Schutzmaßnahmen durchzuführen. Der Tierschutz ergibt sich aus der Verantwortlichkeit des Menschen auf die Umwelt. Die Forderung nach Beachtung von Lärm auf Tiere erweitert ggf. den Planungsaufwand, die Forderung nach Vermeidung erhöht ggf. den ökonomischen Aufwand für Schallschutzmaßnahmen. Vorrang des aktiven vor dem passivem Schallschutz ist Grundlage, um die aktuelle Praxis des Schallschutzes umzukehren. Oft wird passiver Schutz betrieben, weil dieser auf den ersten Blick einfach billiger ist. Durch passiven Schallschutz wie. z.B. Lärmschutzfenster werden Menschen in Häuser eingesperrt, aktiver Lärmschutz z.B. nächtliche Geschwindigkeitsbeschränkungen wirken an der Quelle. Bisher werden Gesundheitskosten bei der Bewertung von Lärm nicht berücksichtigt. Das Einbeziehen der Gesundheitskosten führt direkt zur Notwendigkeit von besserem Schallschutz. Die Einführung eines Lärmlabels stellt eine Innovation in der Parteienlandschaft dar. Analog zum bekannten Energielabel für Kühlschränke, Waschmaschinen, Glühlampen, das CO2-Label für Autos usw. soll eine einfache zu erfassende Kennzeichnung von Lärmquellen für z.B. Wohn- und Arbeitsorte, Kitas, Schulen usw. geschaffen werden.

\section{Anmerkung}

Der Antrag wurde aus dem Antragsbuch des LPT 2012.1 aus Brandenburg (Antrag WP080\footnote{\url{http://wiki.piratenbrandenburg.de/Antragsfabrik/Lärmemissionen}}) übergenommen. Dort wurde der Antrag angenommen.
