\section{Antrag}

\subsection{Staat und Religion}

Freiheit und Vielfalt der kulturellen, religiösen und weltanschaulichen Einstellungen kennzeichnen die modernen Gesellschaften. Diese Freiheiten zu garantieren, ist Verpflichtung für das Staatswesen. Dabei verstehen die PIRATEN Mecklenburg-Vorpommern unter Religionsfreiheit nicht nur die Freiheit zur Ausübung einer Religion, sondern auch die Freiheit von religiöser Bevormundung. Die PIratenpartei Mecklenburg-Vorpommern erkennt und achtet die Bedeutung, die individuell gelebte Religiosität für den einzelnen Menschen erlangen kann.\\Die weltanschauliche Neutralität des Staates herzustellen, ist daher eine für die gedeihliche Entwicklung des Gemeinwesens notwendige Voraussetzung. Ein säkularer Staat erfordert die strikte Trennung von religiösen und staatlichen Belangen; finanzielle und strukturelle Privilegien einzelner Glaubensgemeinschaften, etwa im Rahmen finanzieller Alimentierung, bei der Übertragung von Aufgaben in staatlichen Institutionen und beim Betrieb von sozialen Einrichtungen, sind höchst fragwürdig und daher abzubauen. Im Sinne der Datensparsamkeit ist die Erfassung der Religionszugehörigkeit durch staatliche Stellen aufzuheben, ein staatlicher Einzug von Kirchenbeiträgen kann nicht gerechtfertigt werden.

\section{Begründung}

Der Antrag wurde aus dem Antragsbuch zum LPT2012.1 in Brandenburg übernommen (WP074). Der Antrag wurde dort nicht angenommen. Der Antrag ist eine Kopie des Berliner Wahlprogrammes.
