\section{Anträge}

\subsection{Suchtpolitik}

\subsubsection{Konsumentenjagd beenden, konsequente Vorsorgepolitik starten}

Die Piratenpartei Mecklenburg-Vorpommern will sich mit Hilfe von Modellversuchen dafür einsetzen neue drogenpolitische Lösungen für das ganze Land zu etablieren. Unser Ziel ist es, mit einer pragmatischen Suchtpolitik Schaden von der Gesellschaft abzuwenden. Die ersten Schritte dieses Weges können und werden wir in der kommenden Legislaturperiode gehen.

\subsubsection{Problembewusstsein stärken, riskanten Konsum verhindern}

Der beste Schutz vor Abhängigkeitserkrankungen ist ein intaktes soziales Umfeld. Wir wollen Eltern dabei unterstützen, ihren Kindern einen risikoarmen Umgang mit Rauschmitteln zu vermitteln. Flankierend werden wir den Unterricht an den Schulen Mecklenburg-Vorpommerns um ein Modul erweitern, das den Gebrauch bewusstseinsverändernder Substanzen im historischen und psychosozialen Kontext erarbeitet. Ziel des ``Rauschkunde''-Unterrichts ist es, Jugendlichen Werkzeuge zur Selbstkontrolle aufzuzeigen. Diese Präventionsarbeit in Schulen kann nur gelingen, wenn vom Abstinenzdogma abgerückt wird, da diese Haltung gerade für junge Menschen unglaubwürdig ist. Wir werden die Landesmittel für niedrigschwellige Hilfsangebote in der Suchthilfe deutlich aufstocken. Die therapeutische Arbeit wird dabei legale Rauschmittel und nichtstoffgebundene Süchte gleichberechtigt einschließen, da von ihnen ebenfalls große Gefahren für die Gesellschaft und den Süchtigen ausgehen.

\subsubsection{Konsumenten schützen, Gesundheitsschäden minimieren}

Wir glauben, dass eine ``drogenfreie Gesellschaft'' unmöglich ist. Statt die begrenzten Mittel für die vergebliche Jagd auf Konsumenten zu verschwenden, werden wir jene, die Rauschmittel nutzen, umfassend vor Gesundheitsrisiken schützen. Das Wissen um Wirkstoff und Beimengungen ist Grundlage risikoarmen Drogengebrauchs. Wir werden deshalb ein ``Drugchecking''-Programm etablieren, das Konsumenten mit diesen mitunter lebensrettenden Informationen versorgt. Als ersten Schritt werden wir die Resultate der Drogentests des Landeskriminalamtes in On- und Offlinedatenbanken für Jedermann verfügbar machen.\\Die PIRATEN Mecklenburg-Vorpommern setzen sich dafür ein, das Urteil des Bundesverfassungsgerichtes zur Entkriminalisierung des gelegentlichen Konsums von Drogen zu nutzen, um Polizei und Staatsanwaltschaft von zehntausenden Verfahren zu entlasten. Dazu werden wir die Regelung zur ``Geringen Menge'' von Ausnahmetatbeständen befreien und derart neu formulieren, dass Verfahren frühzeitig eingestellt werden können.\\Illegal gehandelte Cannabisprodukte sind immer häufiger mit Beimengungen verunreinigt, deren Gesundheitsgefahren die des Cannabis übersteigen. Wir werden deshalb einen Modellversuch zur legalen Eigenversorgung mit Cannabisprodukten nach dem Vorbild der spanischen ``Cannabis Social Clubs'' starten. Darüber hinaus setzen wir uns für eine bundesweite Legalisierung der Hanfpflanze ein.

\subsubsection{Bestehende Netzwerke nutzen, gemeinsam Zukunft gestalten}

Die PIRATEN Mecklenburg-Vorpommern streben die Zusammenarbeit mit allen gesellschaftlichen Gruppen an, die sich vorurteilsfrei mit dem Konsum von Genussmitteln und dessen Folgen auseinandersetzen. Gemeinsam werden wir eine Suchtpolitik erarbeiten, die riskanten Drogengebrauch verhindert, echten Jugend- und Verbraucherschutz ermöglicht und überdies die Rechte von Nichtkonsumenten schützt.

\section{Begründung}

Dieser Antrag wurde aus dem Antragsbuch zum LPT2012.1 in Brandenburg übernommen (WP098). Der Antrag wurde dort angenommen.
