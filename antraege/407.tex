\section{Antrag}

\subsection{Erfassung und Speicherung von Biometrischen Daten}

Es lässt sich derzeit der besorgniserregende Trend beobachten, dass in immer größerem Umfang die Speicherung und der automatisierte Abgleich von biometrischen Daten erfolgt. Es ist weder zu verhindern, dass die Grundrechte unschuldiger Bürger bei einem solchen Vorgehen verletzt werden, noch dass ein solches Vorgehen auf Basis existierender Daten immer häufiger angewendet wird. Daher lehnen die PIRATEN Mecklenburg-Vorpommern die Erfassung biometrischer Daten ohne Anfangsverdacht sowie deren Speicherung ohne erwiesene Straftat kategorisch ab.

Darüber hinaus lehnen die PIRATEN Mecklenburg-Vorpommern die dauerhafte Speicherung von DNA-Datensätzen von nicht belasteten Personen grundsätzlich ab. Auch persönliche Daten, die im erkennungsdienstlichen Verfahren gewonnen wurden, sind im Falle des § 170 Abs. 2 StPO oder bei Freispruch, nach Abschluss des Verfahrens unverzüglich zu löschen.

\section{Begründung}

Dieser Antrag wurde aus dem Antragsbuch des LPT2012.1 aus Brandenburg übernommen (WP006). Dort wurde der Antrag angenommen.
