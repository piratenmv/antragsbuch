\section{Antrag}

Die Piratenpartei Mecklenburg-Vorpommern setzt sich dafür ein, dass das Gesetz über den Öffentlichen Gesundheitsdienst im Land Mecklenburg-Vorpommern (Gesetz über den Öffentlichen Gesundheitsdienst - ÖGDG M-V) (ÖGDG MV\footnote{\url{http://mv.juris.de/mv/gesamt/OeGDG\_MV.htm}}) in seiner Art erhalten bleibt. Eine Pflicht des Arztes, bei einem Verdacht auf Missbrauch, Vernachlässigung oder Misshandlung des Kindes das Jugendamt zu informieren, wird abgelehnt.

\section{Begründung}

tl;dr: Änderungen im BKiSchG sind vollkommen ausreichend. Die Meldekette gut definiert. Ärzte sind an eine Berufsethik gebunden, man muss sie nicht verpflichten.\\Das am 01.01.2012 in Kraft getretene Bundeskinderschutzgesetz (BKiSchG) (BKiSchG\footnote{\url{http://www.bundesaerztekammer.de/downloads/BKiSchG\_22.12.2011\_final.pdf}}) ist für Kinder und Jugendliche, aber auch Ärzte ein Schritt in die richtige Richtung. Die wichtigsten Neuregelungen für Ärzte durch das Gesetz sind die Schaffung einer bundeseinheitlichen Befugnisnorm zur Einschaltung des Jugendamtes bei Verdacht auf Vernachlässigung oder Misshandlung eines Kindes (§ 4 KKG). Damit wird für Ärzte eine größere Rechtssicherheit im Umgang mit der ärztlichen Schweigepflicht nach § 203 StGB einerseits und der Einschaltung Dritter auf der Grundlage eines rechtfertigen Notstandes nach § 34 StGB geschaffen.\\Die Schweigepflicht ist eine essentielle Voraussetzung für den Arztberuf. Eine gesetzliche Grundlage stellt dafür die ärztliche Berufsordnung dar. Danach dürfen Ärzte ein ihnen anvertrautes Geheimnis nicht unbefugt offenbaren. In der Strafprozessordnung (StPO) und der Zivilprozessordnung (ZPO) ist zusätzlich festgelegt, dass diese Geheimnisse weder Gerichten noch der Polizei mitzuteilen sind. Sie ist auch bei Minderjährigen einzuhalten, d.h. deren Erziehungsberechtige sind nicht zu informieren. § 203 Absatz 1 Nummer 1 Strafgesetzbuch (StGB) stellt denjenigen unter Strafe, der unbefugt ein fremdes Geheimnis, namentlich ein zum persönlichen Lebensbereich gehörendes Geheimnis offenbart, das ihm als Arzt oder Zahnarzt anvertraut worden oder sonst bekanntgeworden ist.\\Der Arzt darf aber die Schweigepflicht unter folgenden Bedingungen brechen, ohne sich strafbar zu machen: a) Bestehen eines rechtfertigenden Notstandes (§ 34 StGB), b) mutmaßliche Einwilligung des Patienten. Nun ist auch eine Ausnahme von der Schweigepflicht zum Schutz von Kindern und Jugendlichen besser geregelt und bietet Rechtssicherheit für Ärzte.\\Das Bundeskinderschutzgesetz (BKiSchG) erläutert den Weg der gegangen werden kann. Zu Beginn der Kette steht ein Gespräch mit dem Kind oder Jugendlichen und dem Personensorgeberechtigten. Danach besteht ein Anspruch auf Beratung durch eine Fachkraft des Jugendamtes und die Erlaubnis zur pseudonymisierten Datenübermittlung. Zuletzt existiert die Befugnis zur Einschaltung des Jugendamtes unter Mitteilung der erforderlichen Daten.\\Es ist davon auszugehen, dass ein Arzt, der einer gewissen Berufsethik unterworfen ist, in jedem Fall diesen Weg beginnt und die zusätzliche Arbeit für das Kindeswohl auf sich nimmt. Für Mecklenburg-Vorpommern nimmt man beispielsweise im Jahre 2006 insgesamt 389 Fälle von Kindesvernachlässigung, Kindesmisshandlung und sexuellen Missbrauch an (Leitfaden MV\footnote{\url{http://www.gewalt-gegen-kinder-mv.de/images/stories/tk-leitfaden\_gewalt-gegen-kinder.pdf}}).\\Ein Patient muss sich darauf verlassen können, dass ein Geheimnis, das er dem Arzt anvertraut hat, bei diesem grundsätzlich sicher aufgehoben ist, und zwar unabhängig davon, ob das in einem Gespräch oder bei der ärztlichen Untersuchung passiert ist. Eine weitere Aushöhlung dieses Grundsatzes ist auch unter speziellen Prämisse des Schutzes von Kindern und Jugendlichen nicht akzeptabel. Deswegen sollte es keinen Zwang geben, dem Jugendamt Meldung von solchen Verdachtsfällen geben zu müssen. Die jetzige Rechtslage reicht dafür aus, um dem Arzt Handlungsalternativen zu geben und auf seine Entscheidung für oder gegen die Schweigepflicht zu vertrauen.

\section{Hinweis}

Dieser Antrag entstand durch Anregung von Stefan im Antrag ``U- und J-Untersuchungen für Kinder und Jugendliche'' und mit Texten der Webseite der Bundesärztekammer und aus Wolfgang Keils ``BASIC Rechtsmedizin''. Er soll eine Alternative darstellen und zeigen, das die aktuelle neue Gesetzgebung vollkommen ausreicht um Kinder und Jugendliche zu schützen.
