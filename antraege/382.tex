\section{Antrag}

Die Piratenpartei Mecklenburg-Vorpommern setzt sich dafür ein, dass die gesundheitliche Aufklärung und Ernährungskunde an Schulen den ihrer hohen Relevanz entsprechenden Platz einnimmt und befürwortet eine Beibehaltung des LandesschulobstprogrammMV (Landesschulobstprogramm\footnote{\url{http://service.mvnet.de/\_php/download.php?datei\_id=27160}}) für Grund- und Förderschulen und Erweiterung auf alle Schulformen in Mecklenburg-Vorpommern.\\Des Weiteren setzt sich die Piratenpartei Mecklenburg-Vorpommern dafür ein, dass möglichst von lokalen und regionalen Erzeugern, ökologisch und gesundheitlich unbedenkliche Produkte bezogen werden.

\section{Begründung}

Eine gesunde und ausgewogene Ernährung ist wichtig für den Menschen, vor allem aber für Kinder. Viele Schüler in Mecklenburg-Vorpommern kommen mit keiner ausgewogenen oder ganz ohne Pausenmahlzeiten zur Schule. Frühstücken sollten die Schüler eigentlich zu Hause, was in der Realität aber wenige tun (Nur 2/3 Kindern und Jugendliche frühstücken\footnote{\url{http://www.schulwesen-mv.de/elternportal/themenbereiche/gesundheit-und-entwicklung/nur-2-von-3i-kindern-und-jugendlichen-fruehstuecken.html}}). Insbesondere der Obst- und Gemüseanteil der Mahlzeiten ist sehr gering und damit aus ernährungsphysiologischer Sicht zu vitaminarm.\\Zudem sind viele Kinder durch falsche Ernährung und zu wenig Bewegung übergewichtig bis fettleibig. (Jedes fünfte Kind zu dick\footnote{\url{http://www.gesundheit.de/ernaehrung/essstoerungen/hintergrund/uebergewicht-jedes-fuenfte-kind-in-deutschland-ist-zu-dick}}) Übergewicht ist in Deutschland seit den 90iger Jahren bei Kindern und Jugendlichen um ca. 50\% gestiegen (KIGGS 2007) Gerade Mecklenburg-Vorpommern ist trauriger Spitzenreiter mit 12,4\% übergewichtiger und sogar 5,5 \% adipöser frisch eingeschulter Schüler. (Schuleingangsuntersuchungen 2009/10) (Adipositas bei Kindern in MV\footnote{\url{http://www.adipositas-mv.de/index.php?id=19}})\\Der Beschluss der EU-Kommission für ein europaweites Schulobstprogramm (Schulobstprogramm\footnote{\url{http://de.wikipedia.org/wiki/Schulobstprogramm}}) ist in Deutschland bis dato nur von wenigen Bundesländern (z.B. Saarland und Bremen) umgesetzt worden, vor allem scheitert es hier an Kostenübernahmen durch das Land. Die EU übernimmt 50\% der Kosten. Gesetzliche Grundlage ist hierbei das SchulObG. (SchulObG\footnote{\url{http://www.gesetze-im-internet.de/schulobg/index.html}})\\Der Grund für die schlechte Ernährung besteht unter anderem auch darin, dass schon im Kindesalter falsche Essgewohnheiten entstehen. Laut Deutscher Gesellschaft für Ernährung (DGE) essen die Deutschen z.B. 3x mehr Fleisch, als gesund für sie wäre. Auch werden Süßigkeiten als Belohnung eingesetzt. Volkskrankheiten wie Fettleibigkeit und Herzerkrankungen sind nur zwei der Folgen.\\Um dem vorzubeugen halte ich es für richtig und wichtig, den Beschluss der EU-Kommission zum Schulobstprogramm auch in Mecklenburg-Vorpommern an allen Schularten umzusetzen und das jetzt laufende Programm zu unterstützen und zu verlängern. Den Schulen ist dabei, im Rahmen regionaler Verantwortung, größtmögliche Autonomie bei der Umsetzung zu gewähren.\\Bis jetzt werden die Äpfel nur bei der ``Obst-Gemüse-Vermarktungsgesellschaft mbH Evershagen'' (Erzeugerfirma\footnote{\url{http://www.rostocker-obst.de/firma.htm)("Aktion}} Apfelkiste":http://www.schulobst-mv.de/)\footnote{\url{http://www.schulobst-mv.de/}}) bezogen. Dies könnte für ganz Mecklenburg-Vorpommern gedacht, größere Transportwege bedeuten und so der frische des Obstes abträglich sein. Des weiteren würde so die Umweltbilanz der Äpfel merklich sinken. Auch könnte dies Anreize für neu zu gründende Obstproduzenten sein, vor allem wenn das Angebot auf andere Sorten ausgedehnt werden würde.

\section{Hinweis}

Dieser Antrag wurde in leicht abgewandelter Form schon einmal im LQFB des Berliner Landesverbands gestellt und abgestimmt. Im Original ist dieser Antrag von Philipp Magalski und Manuela Schauerhammer. Dieser Antrag ist für mich ein Anliegen, da man mit wenig Kosten und Aufwand, ein bestehendes Programm weiter betreiben und erweitern könnt. Strukturen wurden schon geschaffen und können einfach nur ausgeweitet werden.\\Ich bin der Meinung, dass wir diesen Antrag auch hier im LQFB-MV ab- und vor allem zustimmen sollten. Ziel ist es im Kinder und Jugendalter wieder mehr Lust an gesundem und qualitativem regionalen Obst und Gemüse zu wecken, um nicht nur die Kindesentwicklung zu fördern und zu unterstützen, sondern auch Vorsorge für das Alter zu treffen. Gerade hier in Mecklenburg-Vorpommern.

\subsection{Anregungen}

\subsubsection{Schulobstprogramm hat so ne eine bestimmte Summe (400000€)?}

Das Programm hat drei Unterpunkte: a)Der „aid-Ernährungsführerscheins`` in Klassenstufe 3 für 37.400€ b) Bereitstellung einer ``Apfelkiste'' für die Grund- und Förderschulen des Landes, ausschließlich Äpfel aus Mecklenburg-Vorpommern für insgesamt 400.500€; c) Bewirtschaftung von Streuobstwiesen durch Schullandheime in Mecklenburg-Vorpommern unter Einbeziehung der Schülerinnen und Schüler, die sich dort aufhalten für 45.000.

\subsubsection{Macht das ganze überhaupt Sinn?}

Die Schulen berichten, dass sich das Frühstücksverhalten hinsichtlich des Verzehrs verändert hat, vor allem bei den Kindern, die bislang nicht an den regelmäßigen Verzehr von Obst gewöhnt waren. Das Projekt wirkte sich auch positiv im Unterricht aus. So wurde im Sachunterricht die Apfelkiste als Anlass genommen, Inhalte zur Ernährungsbildung und gesunder Lebensweise stärker zu behandeln sowie Apfelsorten und Apfelanbau als Unterrichtsschwerpunkt aufzunehmen. Viele Schulen nutzten die Äpfel bei der Herstellung von verschiedenen Gerichten im Hauswirtschaftsunterricht, z. B. Apfelmus, Apfelkuchen und Bratäpfel.(Zwischenergebnis 2011\footnote{\url{http://www.regierung-mv.de/cms2/Regierungsportal\_prod/Regierungsportal/de/lm/Themen/Verbraucherschutz/Verbraucherinformationen/Verbraucherinformationen\_2011/index.jsp}})

\subsubsection{Versorgernetzwerk anzulegen (Transporteffizienz)}

Die Äpfel für das Landesschulobstprogramm stammen ausschließlich aus Mecklenburg-Vorpommern. Die Produzenten sind in der Erzeugerorganisation Mecklenburger Ernte und dem Verband Mecklenburger Obst und Gemüse e.V. organisiert. Weitere Anbieter können sich grundsätzlich in das Netzwerk einsteigen, wenn gesichert ist, dass die Ware aus Mecklenburg-Vorpommern stammt. Es gibt z. Zt. 7 Lieferanten, verteilt über ganz M-V, die die Belieferung der einzelnen Schulen in ihrem Territorium vornehmen. Der Radius, in den diese Lieferanten ausliefern, wird ca. 40 km betragen, dies ist aber eine grobe Schätzung. (Inhalte aus Anfrage an die Obst- Gemüse Vermarktungsgesellschaft mbH)

\subsubsection{Wie weit und wie oft wird die Qualität der einzelnen Lieferanten/Produzenten geprüft?}

Es dürfen nur Äpfel geliefert werden, die die gesetzlichen Vorschriften der Normen für Handelsklassen erfüllen. Die Qualität der Äpfel wird im Rahmen von Eigenkontrollsystemen der Unternehmen sowie im Rahmen von amtlichen Kontrollen regelmäßig und risikoorientiert überprüft. Die belieferten Schulen sind gefordert, und dies ist zweifellos auch in ihrem eigenen Interesse, Qualitätsabweichungen auf den Lieferscheinen oder direkt mitzuteilen.(Inhalte aus Anfrage an die Obst- Gemüse Vermarktungsgesellschaft mbH)

\subsubsection{Kosten pro Schüler/Obst/Kiste? Senkbar?}

Der aktuelle Kilopreis wird mit Hilfe der Preismitteilungen der Agrar-Informationsgesellschaft (AMI) für jede Kalenderwoche ermittelt. Dazu kommen die Aufwendungen für die Verpackung und Logistik. Im Schuljahr 2011/2012 lag der Preis bei 1,65 €/kg netto (1 kg ca. 8 Äpfel). Die Bestellung ist generell kostenfrei. Aufwendungen an der Schule können zwar durch die Organisation der Annahme und Ausgabe der Äpfel anfallen. In vielen Schulen erfolgt dies durch die Schülerinnen und Schüler selbst. Die Einsparmöglichkeiten bei größeren Liefermengen (mehr Klassen, ganze Schule) sind von vornherein einkalkuliert worden. Wo immer es möglich ist, wird die Belieferung z. B. mit der Belieferung von Schulmilch verbunden. Die beteiligten Grund- und Förderschulen werden in der Regel komplett beliefert, eine Belieferung einzelner Klassen ist aus Kostengründen bisher abgelehnt worden. (Inhalte aus Anfrage an die Obst- Gemüse Vermarktungsgesellschaft mbH)
