\section{Antrag}

\subsection{Kultur darf kein Privileg für wenige sein!}

Mit den heutigen und künftigen Mitteln digitaler Techniken kann Kulturgut in Museen, Archiven, Sammlungen und Bibliotheken verstärkt flächendeckend erfasst und allgemein zugänglich gemacht und damit verbreitet werden. Gleichzeitig kann so auch langfristig Kulturgut archiviert werden –bei allen Problemen, die in diesem Bereich noch zu lösen sind. Die Piratenpartei unterstützt dementsprechend regionale, überregionale und europaweite Projekte zur Kulturgutsicherung.

Die einmalige Chance, mithilfe neuer Techniken und Medien Kunst und Kultur möglichst allen Bürgern zugänglich zu machen, sollte genutzt werden. Dabei beschränkt sich der Kulturbegriff nicht nur auf die traditionellen Sparten, sondern schließt ausdrücklich neue Bereiche wie Video- und Computerspiele als Kulturgut mit ein.

Die Piratenpartei setzt sich dafür ein, dass Einsparungen in den öffentlichen Haushalten nicht zu Lasten von Bildung und Kultur gehen.

Wie ein demokratisches Gemeinwesen verfasst ist, wird treffend durch die Worte Friedrich Schillers beschrieben: „Die Kunst ist eine Tochter der Freiheit.``

Durch die Kulturförderung werden nicht nur die Kreativen geschützt, sondern auch unsere Haltung und Freiheitsrechte. Eine verantwortliche, transparente, anregende und nachhaltig gestaltende Kulturpolitik kräftigt eine zukunftsorientierte, vielfältige und humane Gesellschaft. Diese Politik muss die notwendigen Rahmenbedingungen für eine freie Entfaltung von Kunst und Kultur schaffen – sie darf diese nicht bewerten oder vereinnahmen. Die kulturelle Freizügigkeit, der subversive Charakter und die Vielfalt sollen durch geförderten Freiraum und einer Verhältnismäßigkeit bei der Wahrung der Rechte der Anwohner verteidigt werden. Behörden sollen ihre Ermessensspielräume nutzen, um zugunsten von Kunst- und Kulturinitiativen zu entscheiden. Das Kulturleben Meck-Pomms soll sich auch als Wirtschaftsfaktor und Vernetzungsplattform lebendig weiterentwickeln.

Kulturentwicklungsplanung ist vielschichtig und muss die kulturelle Bildung, Betätigung und Mitwirkung des Bürgers sowie die Künste und die Kulturwirtschaft aufeinander abstimmen und die dafür notwendigen Ressourcen und Verfahren definieren. Die Piratenpartei ist bestrebt, die Förderstruktur von Kunst und Kultur möglichst stabil zu halten. Bei einzelnen Sparten sollte auch in Wirtschaftskrisen nicht so stark gekürzt werden, dass ihre jeweilige Existenz gefährdet ist, denn im Gegensatz zu materiellen Werten kann eine verlorene kulturelle Infrastruktur nur langsam wieder aufgebaut werden.

\subsection{Rundfunk und Medien}

\subsubsection{Rundfunkgebühren}

Es muss gewährleistet sein, dass Rundfunkbeiträge nicht gezahlt werden müssen, wenn z.B. aus benutzerspezifischen (beispielsweise Seh- oder Höreinschränkung), geographischen oder finanziellen Gründen, ein ohne erforderliche technische Hilfsmittel möglicher, empfangskostenfreier Rundfunkzugang (wie DVB-T) nicht gegeben ist. Dies gilt auch für betriebliche Gebühren.

\subsubsection{Dauerhafte Verfügbarkeit öffentlich-rechtlicher Berichterstattung}

Eine der Aufgaben des gebührenfinanzierten öffentlich-rechtlichen Rundfunks besteht in der Versorgung der Bevölkerung mit unabhängiger Berichterstattung. Die dabei erstellten Inhalte sind seit Umsetzung des 12. Rundfunkänderungsstaatsvertrags nur kurze Zeit in den Mediatheken der Rundfunkanstalten abrufbar, obwohl sie auch dauerhaft von öffentlichem Interesse sind, da sie beispielsweise als Quelle für die politische Diskussion dienen. Sie sollten deshalb zeitlich unbegrenzt zur Verfügung gestellt werden. Solange das Internetangebot der Rundfunkanstalten und die Vorhaltung der Sendebeiträge erheblich eingeschränkt ist, darf der Besitz „neuartiger Empfangsgeräte`` wie PC oder Mobilfunkgeräte keine Beitragslasten bewirken, speziell auch bei Gewerbebetrieben.

\subsubsection{Förderung des Bürgerfunks über Neue Medien}

Der Bürgerfunk soll neben dem klassischen Modell der Sendezeit auf lokalen Rundfunksendern über neue Kommunikationswege gefördert werden. Das Internet bietet eine Möglichkeit, Sendungen zu verbreiten. Sendungen des Bürgerfunks sind somit über eine weitere Quelle verfügbar und einer größeren Zielgruppe zugänglich. Die MV-Piraten wollen die Möglichkeit einer Realisierung überprüfen und bei Umsetzbarkeit eine zusätzliche Verbreitung von Bürgerfunk über das Internet anstreben. Das stellt eine Ergänzung zu den Bemühungen dar, Online-Streams anzubieten. Der Bürgerfunk erhält hierdurch eine neue Plattform, um auch Sendungen anderer Regionen zu bewerben und zu präsentieren.

\subsubsection{Förderung von Sprachkultur im Radio}

Der öffentlich-rechtliche Rundfunk hat den Auftrag, sich den Minderheiten in der Gesellschaft zu widmen. Hierzu zählen insbesondere Migranten und mehrsprachige Mitbürger. Im heutigen Angebot der Radiostationen finden sich bundesweit immer weniger mehrsprachige Programminhalte. Die MV-Piraten setzen sich dafür ein, dass der öffentlich-rechtliche Rundfunk den bisherigen Anteil an plattdeutschen und fremdsprachigen Inhalten nicht weiter einschränkt, evtl. ausbaut. So kann eine kulturelle und sprachliche Vielfalt gewährleistet werden. Diese ist in der EU-Grundrechtecharta festgelegt: ``Die Europäische Union respektiert die sprachliche Vielfalt.''

\subsection{Museen und Kunstsammlungen}

Museen und Kunstsammlungen dokumentieren in unverzichtbarer Weise unsere kulturelle Geschichte und sind elementar für den Erhalt zeitgenössischer Formen der Kunst. Die MV-Piraten treten dafür ein, dass der Betrieb von Museen und Kunstsammlungen sowie der Erhalt historischer Gebäude auch weiterhin ein Politikziel in MV bleibt. Eine lebendige Kunstszene ist ein essentieller Teil eines lebens- und liebenswerten Landes. Das Stadtbild verarmt, wo es nicht gelingt, historische Bausubstanz zu erhalten und zu restaurieren.

\subsubsection{Zugang zu Kultur erleichtern}

Museen bieten viele Möglichkeiten den eigenen kulturellen Horizont zu erweitern, Altes und Neues kennenzulernen, Spaß am Entdecken zu haben und zu lernen. Es ist daher von großer Bedeutung, dass Museen gefördert werden, da sie sowohl Bildung als auch Freizeit gestalten können. Jeder Bürger muss barrierefreien und erschwinglichen Zugang zu Museen, und damit zu Wissen, Geschichte und Kultur haben. Digitale Wanderungen durch die Museen soll ermöglicht werden.

\subsubsection{Erhaltung von Kulturgut in Museen und Kunstsammlungen}

Um die Sammlung, Vermittlung und Erhaltung von Kulturgut dauerhaft leisten zu können, ist es erforderlich, langfristig die dazu benötigten Finanzmittel zur Verfügung zu stellen.

\subsection{Bibliotheken/ Literatur}

Die MV-Piraten betrachten gedruckte Bücher als eine wertvolle Kulturform. Literatur hilft uns, die Welt aus anderen als der eigenen Perspektive zu sehen. Sach- und Fachbücher sind unverzichtbar, wenn es darum geht, Wissen zu bewahren und zu verbreiten. Der freie Zugang zu Wissen und Informationen ist ein zentraler Bestandteil unserer Politik.

\subsubsection{Zugang zu Bibliotheksmitteln erleichtern}

In Bereichen ohne direkten Zugang zu Stadt oder Stadtteilbibliotheken und in ländlichen Regionen sollen Möglichkeiten geschaffen werden, Bücher und Medien der nächstgelegenen Bücherei auf Bestellung auszuleihen und zurückzugeben. Hierzu bieten sich sowohl fahrende Bücherbusse an als auch eine digitale Ausleihe.

\subsubsection{Bessere Ausstattung von Bibliotheken}

Die MV-Piraten streben an, die Finanzmittel für Bibliotheken langfristig zu sichern und ein breiteres Spektrum an Werken bereitzustellen. Die Literatur ist eine wichtige Form der Kultur. Das kulturelle Angebot muss ständig aktualisiert und der Allgemeinheit zugänglich gemacht werden.

\subsection{Theater und Orchester}

Die Theater und Orchester in MV sind nicht nur für den Tourismus von großer Bedeutung, sondern auch als Wirtschaftsunternehmen stärken sie die Regionen. Wir haben in MV sechs Theater, vier Orchester, diverse Chöre, Film- und Kulturfeste sowie zwanzig Musikschulen. Sie alle gehören zu MV und machen aus dem Land das, was wir lebens- und liebenswert finden. Diese vielfältige Kultur muss erhalten bleiben.

Der Kultursektor ist einer der zukunftsfähigen Märkte für das Land Mecklenburg-Vorpommern, der direkt und indirekt großen Anteil an der gesamten Wirtschaftsentwicklung für das Land nimmt. Deshalb sollte in die vorhandenen Theater- und Orchesterstrukturen des Landes investiert werden, um nachhaltige Effekte für den Kulturmarkt insgesamt zu erreichen. Ohne die Investition in diese kulturellen „Leuchttürme`` gehen wichtige Strukturen und Potentiale für den Tourismus, die Gesundheitswirtschaft und die Kreativwirtschaft verloren.

Deshalb engagieren sich die Piraten MV für den Erhalt der bestehenden Theater und Orchester. Die vom Land bereit gestellten Gelder müssen als erstem, sofortigem Schritt dynamisiert werden. Ferner treten wir dafür ein, dass ein „runder Tisch Kulturland MV`` einberufen wird, in dem alle Theater und Orchester sowie alle kulturellen Strukturen gemeinsam mit interessierten BürgerInnen transparent über ein Kulturkonzept MV beraten. Als Piraten MV treten wir für die Förderung von Kooperationsmöglichkeiten zwischen Theatern und Orchestern einerseits und Bildungsinstitutionen andererseits ein. Wir versprechen uns davon, dass die gemeinsam entwickelten Angebote helfen, um mit ihren Angeboten für Demokratie und Toleranz zu werben und entsprechende Strukturen - gerade im Jugendbereich - zu fördern. Ebenso sind wir für die Entwicklung von Konzepten für eine kulturell-digitale Mobilität - Stichwort: Theater im Netz.

\subsubsection{Förderung von Laiengruppen und Nachwuchskünstlern}

In den meisten Städten und Regionen gibt es Laientheater-Spielgruppen, Nachwuchsmusiker und andere kreativ engagierte Mitbürger. Für diese ist in der Regel keine staatliche Förderung vorgesehen. Lediglich einige Leuchtturmprojekte erhalten Förderung vom Land oder den Kommunen. Förderung muss nicht zwingend über ein finanzielles Budget geschehen. Stattdessen können für den kreativen Nachwuchs Präsentationsflächen und Proberäume in staatlichen und kommunalen Immobilien zur Verfügung gestellt werden.

\subsubsection{Nachwuchsförderung}

Die Nachwuchsförderung ist die Grundlage der zukünftigen kulturellen Entwicklung. Neue Kunstformen und kulturelle Beiträge müssen umfassend gefördert und gestärkt werden. Hierbei gilt es, ein möglichst breites Spektrum zu unterstützen und neue Wege, insbesondere durch Nutzung moderner Kommunikationstechniken, zu beschreiten.

\subsubsection{Stärkung von kreativen Fähigkeiten}

Die frühzeitige Förderung von künstlerischen Interessen bei Kindern und Jugendlichen ist derzeit nur in Ansätzen vorhanden. Gerade hier müssen Fähigkeiten frühzeitig erkannt und gefördert werden. Wir setzen uns für die Verbesserung der Angebote für Kinder und Jugendliche ein. Insbesondere wollen wir die Förderung junger Talente und deren Fähigkeiten in Vereinen, Organisationen, Verbänden und Schulen verbessern.

\subsubsection{Modellversuch: Förder- \& Kulturzentren}

Im Bereich der Breitenförderung gibt es in der Kulturpolitik gravierende Defizite. Angebote an Subkulturen und Jugendliche, die den kreativen Nachwuchs stellen, werden häufig nur als Beschäftigungsangebote in sozialen Brennpunkten betrachtet. Neue Ideen gehen oft verloren, unbekannte Künstler bleiben unbekannt. Gerade in diesen Bereichen müssen Talente frühzeitig erkannt und gefördert, Möglichkeiten ausgebaut sowie Rahmenbedingungen für eine künstlerische Entfaltung geschaffen werden.

Die MV-Piraten schlagen daher ein Konzept der ``Förder- und Kulturzentren'' vor, das wir als Modellversuch umsetzen wollen. Förderzentren des Landes MV haben den Vorteil, dass sie unabhängig von der Mitgliedschaft in Vereinen oder Organisationen für jeden nutzbar sind. Die Leitung soll durch ehrenamtliche Mitarbeiter erfolgen, die die Einrichtung im Konsensprinzip führen. Die Förderzentren sollten ein Konzept nach Piratenvorbild sein: Es steht allen Interessierten offen. So ist auch eine Plattform zur Präsentation vorhanden. Außerdem können hier Treffpunkte zur Förderung von Interessenschwerpunkten, wie etwa Hackerspaces, eingerichtet werden.

\subsection{Förderung von Offenen Arbeitsstrukturen}

Co-Working-Spaces sind Orte der gemeinsamen Arbeit und Vernetzung zum Vorteil der Einzelnen und der Gemeinschaft. Sie sind offene Arbeitsräume, häufig mit Gastronomie verbunden oder auch offene Büro-WGs. Diese werden zum Beispiel für Homeworker oder Selbständige konzipiert, um durch gemeinsames Arbeiten und Netzwerken einen Mehrwert für jeden Einzelnen zu schaffen. Eine Förderung, die primär durch die Überlassung von Räumlichkeiten aus öffentlicher oder privater Hand vonstatten geht, nutzt bereits vorhandene Mittel und verlangt daher nicht nach teuren Neuinvestitionen. Die so geschaffenen Möglichkeiten bieten ein enormes Innovationspotenzial, das sich aus der Vernetzung und der gemeinsamen Arbeit an Projekten ergibt. Das gibt dem Nutzer die Möglichkeit, seine Fähigkeiten zu spezialisieren und in Zusammenarbeit mit Anderen auszubauen. So werden soziale und berufliche Fähigkeiten gestärkt und erweitert.

\subsection{Förderung von Nischenbereichen, neuen Kunstformen und jungen Künstlern}

Die Kulturpolitik dreht sich nach dem Empfinden der MV-Piraten stark um den sogenannten Mainstream. Gerade Künstler, die nicht bekannt sind oder abseits der anerkannten Kunstformen arbeiten, werden nicht ausreichend gefördert. Oft gibt es lediglich über Kunstvereine oder Mitgliedschaften in einschlägigen Organisationen Unterstützung. Neue Ideen gehen dabei verloren. Den Künstlern fehlt es nicht nur an finanziellen Mitteln, sondern auch an Möglichkeiten, praktisch zu arbeiten oder sich zu präsentieren. Auch sind Angebote für Subkulturen nicht ausreichend vorhanden. Gerade in diesen Bereichen müssen Talente frühzeitig erkannt und gefördert werden. Möglichkeiten sollen ausgebaut und somit Rahmenbedingungen für eine künstlerische Entfaltung geschaffen werden.

\subsection{Angebote für Subkulturen}

Insbesondere im Jugendbereich neigt die bisherige Politik dazu, alles in einen Topf zu stecken und Angebote auf soziale Brennpunkte oder den Mainstream auszurichten. Insbesondere die Förderung von Vereinen mit Bezug zu verschiedensten Formen von Kultur oder Subkultur muss ausgebaut werden. Als Beispiel sind hier selbstverwaltete Projekte, Jugendzentren- und Werkstätte sowie Kultureinrichtungen, die sich an junge Musiker richten, zu nennen. Auch lose Gemeinschaften mit einem gemeinsamen, kulturellen Interesse sollten durch die Schaffung von speziellen Angeboten gefördert werden.

\subsection{Freie Lizenzen fördern}

Freie Lizenzen bieten Künstlern eine alternative Möglichkeit, ihre Werke einfach, und flexibel und ohne bürokratischen oder finanziellen Aufwand nach eigenen Wünschen zu schützen. Ein gutes Beispiel hierfür ist das Creative Commons Modell, das sich zunehmender Beliebtheit erfreut. Die MV-Piraten wollen freie Lizenzen thematisieren und fördern.

\subsection{Spiele}

Spiele, ob in klassischer analoger oder in digitaler Form, sind Bestandteil unseres sozialen Zusammenlebens. Die MV-Piraten erkennen den Vorgang des Spielens als wichtigen Beitrag zur gesellschaftlichen und kulturellen Entwicklung an. Insbesondere aus dem Bereich der Jugendkultur sind moderne Spiele wie Computer- und Actionspiele nicht mehr wegzudenken. Die MV-Piraten halten es für falsch, Spieler zu kriminalisieren, statt die eigentlichen gesellschaftlichen Probleme zu lösen.

\subsubsection{Förderung von Spielen als Kulturgut}

Video- und Computerspiele, klassische Spiele wie Brett-, Karten- sowie Rollenspiele, das elektronisch unterstützte Geocaching und Sportspiele wie beispielsweise Paintball sind Kulturgüter und sollten als solche gefördert werden. Spielen fördert unabhängig vom Medium stets Lernprozesse und Kommunikation, Vernetzung und soziale Interaktion. Da sich viele Aufgaben im Spiel nur im Team lösen lassen, fördern sie mit Führungskompetenz und Teamfähigkeit die Qualitäten, die im Arbeitsleben des 21. Jahrhunderts von essentieller Bedeutung sind.

Spiele werden nicht nur von Kindern und Jugendlichen, sondern auch von Erwachsenen als Freizeitaktivität wahrgenommen. Sowohl Video- und Computerspiele als auch Actionsportarten sind längst in der Mitte der Gesellschaft angekommen. Die Nutzung moderner Medien baut soziale sowie nationale Grenzen ab und fördert mit Online-Spielen das gegenseitige Verständnis. Video- und Computerspiele ermöglichen es Künstlern, neue Ausdrucksformen jenseits der klassischen Medien zu finden. Sie bedürfen daher der Anerkennung als Kunstform. Aus diesen Gründen setzen sich die MV-Piraten für die Anerkennung und Förderung der analogen und digitalen Spielkultur ein. Zensur und Verbotsforderungen lehnen wir entschieden ab. Der verantwortungsbewusste Umgang mit dem Medium Video- und Computerspiel soll durch Aufklärung und Schaffung von Medienkompetenz und nicht durch Verbote erreicht werden. Dies gilt für Heranwachsende und für Eltern.

\subsubsection{Förderung von eSport}

eSport ist die Kurzbezeichnung für `Elektronischer Sport', einer modernen Form des sportlichen Wettkampfs, die mit Computerspielen über das Internet oder auf LAN-Turnieren ausgetragen wird. Im Zuge des weltweiten Bandbreitenausbaus hat der eSport sich zu einer Breitensportart, insbesondere der Jugendkultur, entwickelt. eSport schafft dabei ein soziales Netz für die zahlreichen, jugendlichen Konsumenten von Online-Spielen. eSport holt Jugendliche bei einer ihrer bevorzugten Freizeitaktivitäten ab. Er vermittelt die Werte von sportlicher Fairness und Teamgeist und lässt Jugendliche an sozialen Veranstaltungen teilnehmen, online wie vor Ort. Außerdem ermöglicht eSport körperlich beeinträchtigten Menschen in einem Sportverein aktiv zu werden. Die MV-Piraten engagieren sich für die Förderung von eSport sowie dessen Vernetzung mit sozialen Projekten und der Vermittlung von Medienkompetenz bei Eltern und Schülern. Dazu werden Kooperationen mit Schulen und regionalen eSport-Veranstaltern angestrebt.

\subsection{Öffentlicher Raum für alle}

Die Nutzungsmöglichkeiten des öffentlichen Raums für alle müssen verbessert werden. Die Innenstädte gehören auch spielenden Kindern und skatenden Jugendlichen. Zwischen den Interessen von Anwohnern und anderen Nutzern des öffentlichen Raumes muss immer ein gerechter Ausgleich stattfinden. Interessengruppen dürfen dabei nicht bevorzugt werden aufgrund ihrer besseren Finanzausstattung oder besserem Organisationsgrad. Wir möchten Bürgervereinigungen, Vereinen und Kulturgruppen den Gebrauch öffentlicher Gebäude einfacher machen und setzen uns für entsprechende Verbesserungen in Nutzungs- und Haftungsregelungen ein.

\subsubsection{Öffentlicher Raum in privater Hand}

Die zunehmende Privatisierung städtischer Räume durch Einkaufszentren und Einkaufsstraßen, die von privaten Wachdiensten „sauber`` gehalten werden, sehen wir sehr kritisch. Eine solche Bewirtschaftung öffentlichen Raums darf nicht dazu führen, dass politische Betätigung (z.B. Infostände) im öffentlichen Raum unmöglich werden oder Menschen, die das „Einkaufserlebnis`` trüben könnten (z.B. Obdachlose), vertrieben werden. Um das Recht aller am öffentlichen Raum zu erhalten, möchten wir auf das Problem aufmerksam machen. Die weitere Ausweitung privaten Raumes zuungunsten öffentlichen Raumes, wollen wir bremsen. Für großflächige, öffentliche Räume in privatem Besitz, wie z.B. Einkaufszentren, wollen wir einen rechtlichen Rahmen gestalten, der dem Charakter dieser Räume als öffentliche Räume, gerecht wird.
