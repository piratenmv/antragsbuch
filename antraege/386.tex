\section{Antrag}

Die PIRATEN Mecklenburg-Vorpommern setzen sich für eine landesweit einheitliche Ausstattung der Rettungsdienstfahrzeuge und Hubschrauber (Krankentransportwagen, Rettungswagen, Notarztfahrzeuge, Rettungs- und Notarzteinsatzhubschrauber) der Kommunen, Hilfsorganisationen (DRK, JUH, ASB, MHD, DLRG, u.a.) als auch der privatwirtschaftlichen Dienstleister ein.\\Die Umsetzung der DIN EN 1789 muss verbindlich durch das ``Gesetz über den Rettungsdienst für das Land Mecklenburg-Vorpommern (Rettungsdienstgesetz - RDG M-V)'' vorgeschrieben werden; ebenso Materialausstattungen, die sich aus den Anforderungen der Richtlinien der Fachgesellschaften der Ärzteschaft ergeben. Als Beispiel dient die flächendeckende Einführung des 12-Kanal-EKG auf Rettungswagen.\\Gleichermaßen muss eine Mindestausstattung an Medikamentengruppen und Wirkstoffen pro Rettungswagen festgelegt werden. Das Ziel ist eine einheitliche Mindestausstattungen zu definieren, die erstens eine hohe Qualität der Patientenversorgung garantieren und zweitens das Zusammenwirken unterschiedlicher Rettungsdienste in Mecklenburg-Vorpommern einfacher gestalten.

\section{Begründung}

Träger des Rettungsdienstes ist in Mecklenburg-Vorpommern jede kreisfreie Stadt, als auch die Landkreise (Rettungsdienstgesetz-MV\footnote{\url{http://www.landesrecht-mv.de/jportal/portal/page/bsmvprod.psml?showdoccase=1\&doc.id=jlr-RettDGMVrahmen\&doc.part=X\&doc.origin=bs\&st=lr}}) Die Ausstattung erfolgt normalerweise anhand der DIN EN 1789, die europaweit verbindliche Rettungsdienstfahrzeuge klassifiziert und deren Ausstattung festlegt. In Mecklenburg-Vorpommern finden wir je nach Stadt aber hoch unterschiedlich ausgestattete Fahrzeuge, die die DIN EN 1789 teilweise deutlich übertreffen oder erschreckend unterschreiten.\\Einige Rettungsdienstträger haben sich dazu entschieden keine oder nur sehr wenige Notfallmedikamente auf Rettungswagen vorzuhalten. Dies kann im Rahmen von Sekundärtransporten oder unerwarteten Notfällen ohne Notarztfahrzeug, zu erheblichen Versorgungsmissständen führen.\\Teilweise werden innerhalb einer Stadt vom Rettungsdienst der Kommune, der vor Ort befindlichen Hilfsorganisation (DRK, JUH, MHD, ASB, DLRG) und den privaten Rettungsdiensten komplett unterschiedliche Ausstattungen mitgeführt. Rettungswagen die im Rahmen von Sanitätsdiensten eingesetzt werden sind in vielen Fällen nicht einheitlich ausgestattet. Auch die Aufrüstung von Rettungsdienstfahrzeugen älterer Generationen, wird mit dem Schlagwort des Bestandschutzes langzeitig ausgesessen. Im Falle vom Zusammenwirken dieser unterschiedlichen Gruppen im Rahmen von Großeinsätzen oder der überörtlichen Hilfe, können so ernsthafte Strukturdefizite zum Problem für den Patienten werden.\\So ist die Qualität der Versorgung von Notfallpatienten in vielen Fällen davon abhängig, in welchen Regionen man einen Notfall erleidet.\\Die Piratenpartei Mecklenburg-Vorpommern setzt sich deswegen für eine Mecklenburg-Vorpommern weite einheitliche Ausstattung von Rettungsmitteln ein. Dies beinhaltet die einheitliche Beschreibung der Gerätefähigkeiten, die klare Ausstattungsliste von medizinischen Kleinmaterial, als auch einer Wirkstofftabelle von Medikamenten die auf Rettungswagen als Mindestausstattung mit zu führen sind.

\section{Hinweis}

Im Original ist dieser Antrag von Thomas Weijers für den AK Gesundheit als Antrag WP026 zu finden. Dieser Antrag ist für mich ein Anliegen, da man im Notfalleinsatz sich auf das Vorhandensein von Medikamenten und Einsatzmitteln verlassen könne müss und gerade in einem Flächenland wie Mecklenburg-Vorpommern die Nachforderung eins weiteren RTW oder eines NEFs zu lange dauern kann.
