\section{Antrag}

Die PIRATEN Mecklenburg-Vorpommern setzen sich für die Einordnung der Wasserrettung in die Aufgaben des Rettungsdienstes ein, was adäquat zur Gesetzgebung des Rettungsdienstgesetz in Brandenburg (RDG BB\footnote{\url{http://www.bravors.brandenburg.de/sixcms/detail.php?gsid=land\_bb\_bravors\_01.c.47078.de}}) erfolgen kann.\\Des Weiteren sollen klare Bestimmungen zu Sicherung- und Rettungsvorkehrungen an Stränden, Flüssen und Binnengewässern, Rettungsorganistations und Leistungsträger übergreifend, beschlossen werden.

\section{Begründung}

Die Wasserrettung wird in Mecklenburg-Vorpommern von drei eingetragenen Vereinen ehrenamtlich geschultert. Dies sind die Wasserwachten des Deutschen Roten Kreuzes, der Deutschen Lebens-Rettungs-Gesellschaft und des Arbeiter-Samariter-Bundes. Im Rettungsdienstgesetz M-V (RDG MV\footnote{\url{http://mv.juris.de/mv/gesamt/RettDG\_MV.htm\#RettDG\_MV\_rahmen}}) wird bis jetzt nur die Trägerschaft der Wasserrettung durch Kommunen und kreisfreie Städte geregelt.\\Bis jetzt hat jede Organisation eigene Regelungen zu Personal, Qualifikation, Anforderungen und Equipment. Dazu können die Betreiber von bewachten Badestellen und Strandabschnitten eigene Vorgaben zu den vorgehaltenen Rettungsmitteln geben. Diese Regelungen müssten an die Standards der Boden- und Luftrettung angepasst werden, nicht nur um von den dortigen Qualitätsstandards zu profitieren, sondern auch um vergleichbar und evaluierbar zu sein. Ein Schritt den Bayern vor Jahren schon bei ihrer Bergrettung getan hat.\\Eine Tatsache, die man bei der Sicherstellung der Qualität der Wasserrettung nicht außer acht lassen darf ist die Tatsache, das der reguläre Rettungsdienst wenn dann nur in zweiter Linie zum Zuge kommt. Zuerst wird Bergung, Rettung, Stabilisierung und ggf. auch Reanimation durch die Wasserrettung geleistet. Deswegen sollte hier nicht an der Sicherstellung einer gleichbleibend hohen Qualität gespart werden.\\Abgesehen von medizinischen und organisatorischen Punkten sollte man nicht vergessen, das Mecklenburg-Vorpommern durch seine Strände und Gewässer Touristen und Gäste anlockt. Solange diese Wasserflächen durch eine gute Wasserrettung sicher be- und überwacht werden, ist dies eine der besten Werbemaßnahmen um weiter Urlauber in dieses Bundesland zu locken.
